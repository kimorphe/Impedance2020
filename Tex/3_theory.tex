\chapter{電気化学インピーダンス法の基礎}
\section{インピーダンス}
電気化学インピーダンス法では試料に交流電圧を印加し,その応答として生じる電流を計測する.
電流$I$と電圧$V$の関係が線形システムとみなせるならば,両者の関係は
\begin{equation}
	V(\omega)= Z(\omega) I(\omega), \ \ (\omega=2\pi f)
	\label{eqn:I2V}
\end{equation}
と表すことができる.ここに,$\omega$と$f$はそれぞれ交流電圧の角周波数[rad/s]と周波数[Hz]を表す.
なお,時間因子$e^{i\omega t}$は両辺に共通のため省略する.
このとき,電流と電圧の比である$Z$はインピーダンスと呼ばれる.
$Z$も周波数の関数であることから,$Z$はインピーダンススペクトルとも呼ばれる.
インピーダンス$Z(\omega)$は一般に複素数で,虚数単位を$i$として
\begin{equation}
	Z(\omega)=Z'(\omega)+iZ''(\omega)
	\label{eqn:Z_cmplx}
\end{equation}
と実部,虚部を書く.$Z'$はレジスタンス,$Z''$はリアクタンスと呼ばれる.
インピーダンスは指数関数を用いて
\begin{equation}
	Z(\omega)=\left| Z \right|(\omega)e ^{i\phi(\omega)}
	\label{eqn:}
\end{equation}
と表すこともできる.ここで$\phi$は複素数$Z$の偏角を表し,
$\phi$は電流と電圧の間の位相遅れを意味する.
%
\section{インピーダンススペクトルの表示方法}
インピーダンススペクトルの表示は目的に応じて二種類の方法が用いられる.
1つめの方法では横軸に周波数を,縦軸にインピーダンスの大きさ$|Z|$をとり
両対数グラフとして表示するものである.同時に,横軸を対数軸として周波数に,
縦軸を$Z$の偏角$\phi$としたグラフを合わせて見ることで,
インピーダンススペクトルの完全な情報を示すことができる.
これらの2つのグラフによる表示をBode(ボード)線図と呼ぶ.
もう一つの方法は,横軸にレジスタンス$R'$を,縦軸をリアクタンスの符号を反転
させた$-R''$とした複素平面にインピーダンス$Z$をプロットするものである.
各周波数におけるインピーダンスをこのような複素平面上にプロットすれば,
インピーダンススペクトルが複素平面上の曲線として表される.このような
インピーダンススペクトルの表示はNyquist(ナイキスト)線図と呼ばれる.
Nyquist線図はインピーダンススペクトの特徴をしばしば直感的にわかり易く示してくれる.
ただし,周波数に対する依存性はNyquist線図上で明示的に示されない.
そのため,周波数との関係を同時に見る必要がある場合には,Bode線図も併用する
ことになる.
%
\section{等価回路}
実験やシミュレーションで得られたインピーダンススペクトルを再現できる
簡単な電気回路を,対象とする試料やモデルの等価回路(equivalent circuit)と呼ぶ.
等価回路を構成する最も基本的な回路素子には抵抗($R$),コンデンサ$(C)$,
インダクタンス$(L)$の3つがある.
回路が理想的な抵抗だけからなる場合,電流と電圧の関係はオームの法則により
\begin{equation}
	V=RI
	\label{eqn:Ohom}
\end{equation}
で与えられる.この場合,インピーダンスは$Z=R$で周波数に依らず,
ボード線図とナイキスト線図はそれぞれ図\ref{fig:fig3_1}に示したようになる.
一方,コンデンサ-だけからなる回路では,電圧と帯電量$Q$の関係:
\begin{equation}
	V=\frac{Q}{C}
	\label{eqn:Q_CV}
\end{equation}
より,
\begin{equation}
	\frac{dV}{dt}=i\omega V =\frac{1}{C}\frac{dQ}{dt}=\frac{I}{C}
	\label{eqn:}
\end{equation}
だから,インピーダンスは
\begin{equation}
	Z=\frac{1}{i\omega C}
	\label{eqn:Zc}
\end{equation}
で与えられリアクタンス成分だけを持つ.
図\ref{fig:fig3_2}はこの結果を示したナイキスト線図とボード線図である.
また,本研究で用いることは無いが,インダクタンスのみの回路については
電流と起電力の関係:
\begin{equation}
	V=-L\frac{dI}{dt}=-i\omega LI 
	\label{eqn:}
\end{equation}
より
\begin{equation}
	Z=i\omega L
	\label{eqn:}
\end{equation}
で,コンデンサと同様リアクタンス成分だけになる.
この場合のインピーダンススペクトルは図\ref{fig:fig3_3}の通りである.
なお,以上の図\ref{fig:fig3_1}-図\ref{fig:fig3_3}においてナイキスト線図
に示した青の矢印は,低周波から高周波側に進む方向を示している.
%--------------------
\begin{figure}[h]
	\begin{center}
	\includegraphics[width=0.9\linewidth]{Figs/fig3_1.eps} 
	\end{center}
	\caption{
		抵抗素子のボード線図(a),(b)とナイキスト線図(c).
	} 
	\label{fig:fig3_1}
\end{figure}
%--------------------
%--------------------
\begin{figure}[h]
	\begin{center}
	\includegraphics[width=0.9\linewidth]{Figs/fig3_2.eps} 
	\end{center}
	\caption{
		コンデンサ素子のボード線図(a),(b)とナイキスト線図(c).
	} 
	\label{fig:fig3_2}
\end{figure}
%--------------------
%--------------------
\begin{figure}[h]
	\begin{center}
	\includegraphics[width=0.9\linewidth]{Figs/fig3_3.eps} 
	\end{center}
	\caption{
		インダクター素子のボード線図(a),(b)とナイキスト線図(c).
	} 
	\label{fig:fig3_3}
\end{figure}
%--------------------
以上に述べた基本的な回路素子を組み合わせることで,より多様なインピーダンススペクトルを表現することができる.
そのような例として以下では,本研究に関連の深い5つの回路を取り上げ,そのインピーダンススペクトルを示す.
\section{合成インピーダンス}
\subsection{RC並列回路}
抵抗$R$とコンデンサ$C$を並列に接続したRC並列回路の合成インピーダンスは
\begin{equation}
	\frac{1}{Z}=\frac{1}{R} + i\omega C \ \ 
	\Rightarrow \ \ Z =\frac{R}{1+i\omega RC}
	\label{eqn:RC_para}
\end{equation}
で与えられる.従ってレジスタンス$Z'$とリアクタンス$Z''$は,それぞれ
\begin{equation}
	Z'=\frac{R}{1+(\omega RC)^2}, \ \ 
	Z''=-\frac{i\omega RC}{1+(\omega R^2C)^2} 
	\label{eqn:}
\end{equation}
となる.また,これらの式から$\omega$を消去すれば,
\begin{equation}
	\left( Z'-\frac{R}{2}\right)^2 +\left(Z''\right)^2 =\left( \frac{R}{2}\right)^2
	\label{eqn:}
\end{equation}
が得られ,式(\ref{eqn:RC_para})のナイキスト線図は中心が$\left( 0, \frac{R}{2}\right)$,
半径が$\frac{R}{2}$の半円を描くことが分かる.
以上より,RC並列回路の合成インピーダンスは図\ref{fig:fig3_4}のようになる.
RC並列回路が描くナイキスト線図の半円は容量性の半円と呼ばれ,実験で得られたスペクトルを
よく再現することがある.
例えば,電解液のインピーダンスには容量性の半円が現れることが知られている.
これは,電極表面に電気二重層が形成されてコンデンサの役割を果たすこと,
電解液から電極への電荷の移動反応に伴う抵抗が存在することの両者の効果が,
計測されるためである.
このように,等価回路の回路素子は物理的な実体として存在するわけでは無いが,
それと同様な効果をもつ電気化学的な過程が存在することを示す.
またそののような電気化学的プロセスの影響を回路定数として定量的に表現できるという
意味でも有用なものとなる.
\subsection{Cole-Coleプロット}
実際の計測データでは容量性半円が真円ではなく縦方向につぶれたような
形状のしたスペクトルが得られることがある.
そのようなナイキスト線図を再現するインピーダンスには,次の
ものが知られている.
\begin{equation}
	Z =\frac{R}{1+\left(i\omega RC\right)^p}
	\label{eqn:CC}
\end{equation}
図\ref{fig:fig3_5}は式(\ref{eqn:CC})のスペクトルを示したもので,
このナイキスト線図はCole-Coleプロットと呼ばれる.
この図には式(\ref{eqn:CC})の指数$p$が$p=0.5,0.8$および1.0の
場合を示している.$p=1.0$の場合は式(\ref{eqn:RC_para})に一致する
ため,式(\ref{eqn:CC})はRC並列回路の一般化と見ることもできる.
図\ref{fig:fig3_5}に示したように,$p$が小さくなるにつれ
半円がより扁平なものになる.
\subsection{Cole-Davidsonプロット}
ナイキスト線図が扁平な円となる別のインピーダンスには,次のものもある.
\begin{equation}
	Z =\frac{R}{\left(1+i\omega RC\right)^p}
	\label{eqn:CD}
\end{equation}
こちらも$p=1$の場合はRC並列回路となる意味で,RC並列並列回路の
一般化とみることができる.式(\ref{eqn:CD})で与えられるスペクトルを$p=0.5,0.8,1.0$の
ケースについて示すと図\ref{fig:fig3_6}のようになり,
このときのナイキスト線図はCole-Davidsonプロットと呼ばれる.
Cole-Davidsonプロットでは,$p<1$のとき半円の左側が
より大きく歪んだ形となる.
%--------------------
\begin{figure}[h]
	\begin{center}
	\includegraphics[width=0.9\linewidth]{Figs/fig3_4.eps} 
	\end{center}
	\caption{
		RC並列回路のボーデおよびナイキスト線図.
	} 
	\label{fig:fig3_4}
\end{figure}
%--------------------
%--------------------
\begin{figure}[h]
	\begin{center}
	\includegraphics[width=0.9\linewidth]{Figs/fig3_5.eps} 
	\end{center}
	\caption{
		Cole-Coleプロットとそのボード線図.
	} 
	\label{fig:fig3_5}
\end{figure}
%--------------------1
%--------------------
\begin{figure}[h]
	\begin{center}
	\includegraphics[width=0.9\linewidth]{Figs/fig3_6.eps} 
	\end{center}
	\caption{
		Cole-Davidsonプロットとそのボード線図.
	} 
	\label{fig:fig3_6}
\end{figure}
%--------------------
\subsection{Constant phase element (CPE)}
コンデンサーのインピーダンス特性を一般化したものには,
Constant phase element(CPE)がある.
これは,二つのパラメータ$T_{CPE}$と$p$をもつ,
次のようなインピーダンスを持つ回路素子として定義される.
\begin{equation}
	Z=\frac{1}{(i\omega)^pT_{CPE}}
	=\frac{1}{\omega^p T_{CPE}} 
	\left( 
		\cos\left(\frac{\pi p}{2}\right) 
		-
		i
		\sin\left(\frac{\pi p}{2}\right) 
	\right)
	=\frac{1}{\omega^p T_{CPE}}e^{-\frac{i\pi p}{2}} 
	\label{eqn:Z_CPE}
\end{equation}
CPEの位相は
\begin{equation}
	\phi=\frac{\pi p}{2}
	\label{eqn:}
\end{equation}
となり一定で周波数に依らず,ナイキスト線図は
一定の傾き$\frac{\pi p}{2}$の直線となる.
図\ref{fig:fig3_7}はこの様子を示したものである.
Cole-Coleプロット(\ref{eqn:CC})は,
抵抗とCPEの並列とみなすこともできる.
%--------------------
\begin{figure}[h]
	\begin{center}
	\includegraphics[width=0.9\linewidth]{Figs/fig3_7.eps} 
	\end{center}
	\caption{
		Constant phase element(CPE)のインピーダンススペクトル.
	} 
	\label{fig:fig3_7}
\end{figure}
%--------------------
\subsection{ワールブルグインピーダンス}
電荷移動が拡散によって律速されるとき,インピーダンススペクトルは,
実数軸に対して45$^\circ$の傾きを持つ直線を描くことが知られている.
この場合,インピーダンスはCPEの特別な場合として
\begin{equation}
	Z=\frac{\sigma}{\sqrt{i\omega}}
	=
	\frac{\sigma}{\sqrt{\omega}}e^{-\frac{i\pi}{4}}
	\label{eqn:Zb}
\end{equation}
で与えられる.式(\ref{eqn:Zb})はワールブルグ(Warburg)インピーダンスと
呼ばれ,そのスペクトルは図\ref{fig:fig3_8}のオレンジの線で示したようになる.
式(\ref{eqn:Zb})のインピーダンスとなることは,
半無限区間での1次元拡散方程式の解から得られる荷電物質の濃度変調から
電位を求め,質量フラックスから電流変調を求めて電位との関係を調べることで
得られる.
ただし,実際には拡散場は電極表面から無限遠方まで続くとは限らず,
有限な拡散層が形成される場合もある.例えば,電解質内で流れがある場合,
拡散層は無限に成長することは出来ず有限な大きさとなる.
この場合,1次元拡散方程式の境界値問題を拡散層の区間で解き,
電流と電圧変調の関係を求めれば,インピーダンスが
次のようになることが示される.
\begin{equation}
	Z=
	\frac{\sigma}{\sqrt{\omega}}e^{-\frac{i\pi}{4}}
	\tanh \left( \delta \sqrt{\frac{i\omega}{D}}\right)
	\label{eqn:ZbF}
\end{equation}
ここに,$\delta$は拡散層の厚さを,$D$は拡散係数を表す.
式(\ref{eqn:ZbF})もワールブルグインピーダンスと呼ばれ,
式(\ref{eqn:Zb})との区別が必要な場合には,有限ワールブルグインピーダンス
と呼ばることもある.
有限ワールブルグインピーダンスのスペクトルは,
図\ref{fig:fig3_8}において青の実線で示したようになる.
ナイキスト線図は,高周波側では傾きが45$^\circ$の直線に漸近し,
拡散層厚が無限大の場合の挙動に近づく.
一方,低周波側ではリアクタンス成分が次第に低下して,最終的には実数軸の点に収束する.
%--------------------
\begin{figure}[h]
	\begin{center}
	\includegraphics[width=0.9\linewidth]{Figs/fig3_8.eps} 
	\end{center}
	\caption{
		無限および有限な厚さの拡散層に対するワールブルグインピーダンス.
	} 
	\label{fig:fig3_8}
\end{figure}
%--------------------
