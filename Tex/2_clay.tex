\section{モンモリロナイトの鉱物学的な特徴}
\subsection{ベントナイト}
ベントナイトは,今から数百万年から数億年前の火山噴火によって堆積した火山灰などが
熱水などと反応し,温度や圧力による変性を受けて鉱床を形成したと考えられている。
ベントナイトは粘着性や吸水性や吸着性に優れ,建設や化学工学を始めとする各種産業分野
で利用されている。ベントナイトの主成分はモンモリロナイトであり,モンモリロナイトが
ベントナイトの性質を決定していると言ってよい.例えば,ベントナイトは膨張性や増粘性を
示す他、水中ではほぼ単結晶にまで分離して分散する.このような性質は,モンモリロナイト
表面に形成される厚い水和層に起因したものである.他にも、ベントナイトの吸着性の一部は,
モンモリロナイト層の間に存在する交換性の陽イオンによるなど,ベントナイトの特異な
性質は概ねモンモリロナイトの挙動によって生じている.このことから,ベントナイトの
性質を理解するためには,モンモリロナイト含水系の挙動を詳しく調べることが必要となる.
\subsection{モンモリロナイト}
粘土鉱物にはスメクタイトを始めとする結晶質鉱物と、イモゴライト等の非晶質鉱物がある。
結晶質の粘土は層状ケイ酸塩(フィロケイ酸塩)の一種で、SiO$^4$四面体シートと
Al(OH)$_6$八面体が積層して一つの結晶を作っている.
SiO$^4$四面体シートは,Si$^{4+}$に配位したO$^{2-}$が四面体を作り、四面体どうしは
頂点酸素を共有して六角網状につながっている.
Al(OH)$_6$八面体層はAl$^{3+}$を中心に6つのOH$^-$あるいはO$^{2-}$が八面体を作り,
八面体は稜を共有してシートを形成する。二種類のシートは、アルミナ四面体の頂点酸素を
共有することで結合するが、四面体の六角網の中心には$O^{2-}$は無いため,
ここにOH$^-$が入る.

モンモリロナイトは層状ケイ酸塩鉱物の一種であるスメクタイトに分類される粘土鉱物である。
図-2.1 に示すように結晶構造はケイ酸四面体層-アルミナ八面体層-ケイ酸四面体層の3層が
積み重なり,単位結晶は厚みが約 1nm,幅が 100-1000nm のとても薄い板状の結晶をしている。実
際は,この薄い板状の単位結晶が数枚積み重なり1つの鉱物粒子を作っている。図-2.2 に結晶構
造の模式図を示す。また,水との分散性と親和性があり,これによって膨潤性などの特徴的な性質
をもつ。アルミナ八面体層の中心原子である Al の一部が Mg に置換されることで陽電荷不足とな
り,各結晶層全体は負に帯電する。しかし結晶層間に Na + ,K + ,Ca 2+ ,Mg 2+ などの陽イオンを挟むこと
で電荷不足を中和し,モンモリロナイトは安定状態となる。この層間陽イオンは容易に交換され
る性質を持っており,水分子を容易に取り込む特性がある。そのため,モンモリロナイトは結晶層
が何重も重なり合った状態で存在しあい,層表面の負電荷及び層間陽イオンが様々な作用を起こ
すことによって,モンモリロナイトの特異的性質は発揮される。



2.1 スメクタイト族粘土鉱物の性質
 2.1.1 スメクタイト
  シート結合の型,層電荷,八面体シート型,さらに同型置換による組成の違いによって,粘土鉱物が分類される。2:1型粘土鉱物であり, 層電荷0.2-0.6の範囲のものをスメクタイトという。2八面体型スメクタイトには次の3種類に分類される。
モンモリロナイト :
バイデライト     :
ノントロナイト   :
Eは交換性陽イオンを一価として表したものである。モンモリロナイトでは四面体シートにはほとんど同型置換がなく,層電荷は八面体シートでの同型置換によって生じている。
。 
2.1.2 モンモリロナイト
ベントナイトの主成分であるモンモリロナイトは、層状ケイ酸塩鉱物の一種であるスメクタイトに分類される粘土鉱物である。図-2.1に示すように結晶構造はケイ酸四面体層-アルミナ八面体層-ケイ酸四面体層の3層が積み重なり,単位結晶は厚みが約1nm,幅が100-1000nmのとても薄い板状の結晶である。実際は,この薄い板状の単位結晶が数枚積み重なり1つの鉱物粒子をつくっている。図-2.2に模式図と層間距離を示す。また,水との分散性と親和性があり,これによって膨潤性などの特徴的な性質をもつ。
アルミナ八面体層の中心原子であるAlの一部がMgに置換されることで陽電荷不足となり,各結晶層全体は負に帯電する。しかし結晶層間に・・・などの陽イオンを挟むことで電荷不足を中和し,モンモリロナイトは安定状態となる。この層間陽イオンは容易に交換される性質を持っており,水分子を容易に取り込む特性がある。そのため,モンモリロナイトは結晶層が何重も重なり合った状態で存在しあい,層表面の負電荷及び層間陽イオンが様々な作用を起こすことによって,モンモリロナイトの特異的性質は発揮される。


2.1.3 ベントナイトの性質
 ベントナイトは,モンモリロナイトという粘土鉱物を主成分とする粘土であり,ベントナイトを利用することは,モンモリロナイトの特徴とほぼ同義である。モンモリロナイトは,その層間に入っている交換性陽イオンにより,ナトリウム型やカルシウム型などと呼ばれている。カルシウム型モンモリロナイトは吸水性に優れているものの,ナトリウム型モンモリロナイトに比べて膨潤性.増粘性,懸濁安定性の面で劣る。そこで緩衝材として膨潤性能,止水性能に優れた性質を有している面で,ナトリウム型モンモリロナイトを含有するベントナイトが地層処分のバリア材として採用が検討されている。
本研究で用いるベントナイトはモンモリロナイトを主成分とするナトリウム型ベントナイトとし,ナトリウム型ベントナイトの中でもクニミネ工業(株)製クニピアFとする。図2.3にクニピアFの物理,化学特性を示す。


