\chapter{スメクタイト族粘土とモンモリロナイト}
\section{ベントナイト}
ベントナイトは今から数百万年から数億年前の火山噴火によって堆積した火山灰などが
熱水などと反応し,温度や圧力による変性を受けて鉱床を形成したと考えられている.
ベントナイトは粘着性や吸水性や吸着性に優れ,建設や化学工学を始めとする各種産業分野
で利用されている.ベントナイトの主成分はモンモリロナイトであり,モンモリロナイトが
ベントナイトの性質を決定している.例えば,ベントナイトは膨張性や増粘性を
示す他,水中ではほぼ単結晶にまで分離して分散する.このような性質はモンモリロナイト
表面に形成される厚い水和層に起因したものである.他にも,ベントナイトの吸着性の一部は
モンモリロナイト層の間に存在する交換性の陽イオンによること等,ベントナイトの特異な
性質は概ねモンモリロナイトの挙動によって生じる.このことから,ベントナイトの性質を
理解するためには,モンモリロナイト含水系の挙動を詳しく調べることが必要となる.
\section{結晶質粘土鉱物\cite{Sudo}}
粘土鉱物にはスメクタイトを始めとする結晶質鉱物とイモゴライト等の非晶質鉱物がある.
結晶質の粘土は層状ケイ酸塩(フィロケイ酸塩)の一種で,SiO$_4$四面体シートと
Al(OH)$_6$八面体が積層して一つの結晶を作っている.
SiO$_4$四面体シートはSi$^{4+}$に配位したO$^{2-}$が四面体を作り,四面体間は
頂点酸素を共有して六角網状につながっている.
これに対しAl(OH)$_6$アルミナ八面体層は,Al$^{3+}$を中心に6つのOH$^-$あるいはO$^{2-}$が
八面体を作り,八面体どうしは稜を共有してシートを形成する.
これら二種類のシートはアルミナ四面体の頂点酸素を共有して結合するが,
四面体六角網の中心に$O^{2-}$は無いため,この位置にOH$^-$が入る.
粘土鉱物には四面体と八面体シートが1:1で結合したカオリナイトや,2つの四面体シートが
八面体シートを挟むように2:1で結合したスメクタイトのようなも2つのタイプに分けられる.
また,八面体シート内のAl$^{3+}$の一部はMg$^{2+}$に置換され,その結果粘土鉱物は負に帯電する.
このような同型置換によって粘土鉱物に生じる正の電荷を層電荷と呼ぶ.
粘土鉱物表面には層電荷を打ち消し電気的中性を保つようにK$^+$やNa$^+$,Ca$^{2+}$等の
交換性の陽イオンが配位する.粘土鉱物は交換性陽イオンを挟み込むことで積層する.
粘土鉱物は4面体層と8面体層数の比,層電荷数,同型置換による組成の違いによって分類されている.
%--------------------
\begin{figure}[h]
	\begin{center}
	\includegraphics[width=0.6\linewidth]{Figs/fig2_1.eps} 
	\end{center}
	\caption{
		スメクタイト分子の単位構造.
	} 
	\label{fig:fig2_1}
\end{figure}
%--------------------
\section{スメクタイトおよびモンモリロナイト}
2:1型の粘土鉱物で層電荷$x$が0.2$\sim$0.6の範囲にあるものをスメクタイトという.
スメクタイト属の粘土鉱物の構造式は次のように表される.
\[
	\rm 
	E_x(M_2-M_3)(OH)_2
	[ Si_\alpha (Al,Fe(III)_\alpha O_{10}))]
	\cdot nH_2O
\]
ただし,
\begin{itemize}
\item 
	$\rm E_x$は交換性陽イオンを,
\item
	$\rm M_2-M_3$は八面体席を占める2価および3価の金属イオンを,
\item
	$\alpha$は四面体席における置換の割合を,
\item
	$n$は粘土層間に水和した単位構造あたりの水分子数
\end{itemize}
を意味し,これらの違いにより表\ref{tbl:smectite}のような鉱物名が与えられている.
ちなみに雲母の構造式は
\[
	\rm KAl_2(OH)_2[SiAlO_{10}]
\]
で,層電荷は丁度1となる.
\begin{table}[h]
\begin{center}
	\caption{スメクタイト族の粘土鉱物}
\begin{tabular}{c|c|c|c|c}
	\hline \hline 
	八面体層 & 鉱物名 & 八面体席 & 四面体席 & 交換性陽イオン \\
	\hline 
	Dioctahedral & montmorillonite & Al$_{1.67}$-Mg$_{0.33}$ & Si$_4$ & E$_{0.33}$ \\
		& beiellite & Al$_{2}$ & Si$_{3.67}$-Al$_{0.33}$ & E$_{0.33}$ \\
		& nontronite & Fe(III)$_{2}$ & Si$_{3.67}$-Al$_{0.33}$ & E$_{0.33}$ \\
	\hline
	Trioctahedral & saponite & Mg$_{3}$ & Si$_{3.67}$-Al$_{0.33}$ & E$_{0.33}$ \\
	 & hectorite & Mg$_{2.67}$-Li$_{0.33}$ & Si$_{4}$ & E$_{0.33}$ \\
	 & stevensite & Mg$_{2.92}$ & Si$_{4}$ & E$_{0.16}$ \\
\hline  \hline
\end{tabular}
	\label{tbl:smectite}
\end{center}
\end{table}
スメクタイトは水分の存在下で交換性陽イオンや粘土鉱物表面に配位する形で水分子を層間に取り込む.
スメクタイトは水分子を強く水和するために顕著な膨潤を起こす\cite{Watanabe,Morodome,Sato}.
このことは,水分子が粘土層間に
強く引きつけられることを意味し,粘土鉱物は水中でよく分散する.また.粘土分子は水和した水分子を
引き連れて運動するため粘土懸濁液は高い粘性を示す.
このように,ベントナイトの特徴的な性質の多くは粘土鉱物の結晶と積層の仕方に起因して発生する.


