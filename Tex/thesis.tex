\documentclass[11pt,a4j]{mybook2}
\usepackage[top=2.5cm, bottom=2.5cm, left=2cm, right=2cm]{geometry}
\usepackage{showkeys}
%\documentclass[11pt,a4j]{jbook}
%\usepackage{graphicx,wrapfig}
\usepackage{graphicx,titlesec}
%\usepackage{tocloft} %目次の調整
%\setlength{\topmargin}{-1.5cm}
%\setlength{\textwidth}{16.5cm}
%\setlength{\textheight}{25.2cm}
\newlength{\minitwocolumn}
\setlength{\minitwocolumn}{0.5\textwidth}
\addtolength{\minitwocolumn}{-\columnsep}
%\addtolength{\baselineskip}{-0.1\baselineskip}
%
\def\Mmaru#1{{\ooalign{\hfil#1\/\hfil\crcr
\raise.167ex\hbox{\mathhexbox 20D}}}}
%
\newcommand{\fat}[1]{\mbox{\boldmath $#1$}}
\newcommand{\D}{\partial}
\newcommand{\w}{\omega}
\newcommand{\ga}{\alpha}
\newcommand{\gb}{\beta}
\newcommand{\gx}{\xi}
\newcommand{\gz}{\zeta}
\newcommand{\vhat}[1]{\hat{\fat{#1}}}
\newcommand{\spc}{\vspace{0.7\baselineskip}}
\newcommand{\halfspc}{\vspace{0.3\baselineskip}}
\bibliographystyle{unsrt}
\newcommand{\twofig}[2]
 {
   \begin{figure}
     \begin{minipage}[t]{\minitwocolumn}
         \begin{center}   #1
         \end{center}
     \end{minipage}
         \hspace{\columnsep}
     \begin{minipage}[t]{\minitwocolumn}
         \begin{center} #2
         \end{center}
     \end{minipage}
   \end{figure}
 }
%\titleformat{\chapter}[display]{\normalfont\normalsize}{\chaptertitlename \thechapter 章}{20pt}{\normalsize}
%{\normalsize}
%\vspace*{\baselineskip}
%\renewcommand{\cfttoctitlefont}{\hfill\normalsize\bfseries}
%\renewcommand{\cftaftertoctitle}{\hfill\null}
\renewcommand{\labelenumi}{(\arabic{enumi})}

\title{
\vspace{20mm}
圧縮成形された不飽和粘土の\\
電気化学インピーダンス特性に関する研究
\\
\vspace{5mm}
Study on Chemical Impedance Characteristics 
of Unsaturated Compacted Clay
\vspace{60mm}
}
%\date{\today}
\date{2021年2月9日}
\author{
	\vspace{40mm}
岡山大学環境理工学部\\
環境デザイン工学科 10429223\\
	佐々木 絢悟}

%\makeatletter
%\def\@evenfoot{\hfil -\thepage- \hfil}
%\makeatother
%\makeatletter
%\def\@oddfoot{\hfil -\thepage- \hfil}
%\makeatother
%\makeatletter
%\def\@oddeven{}
%\makeatother

\begin{document}
\maketitle
\tableofcontents
\frontmatter
\mainmatter
%%%%%%%%%%%%%%%%%%%%%%%%%%%%%%%%%%%%%%%%%%%%%%%%%%%%%%%%%%%%%%%%
\chapter{はじめに}
	\chapter{はじめに}
\section{研究の背景}
我が国は資源に乏しく,エネルギー源を海外から輸入される石油や石炭,天然ガス,ウランなどの天然資源に依存している.
東日本大震災による福島第一原子力発電所の事故が発生するまでは,電力を安定して供給するために原子力発電にも力を入れてきた.
原子力発電では少量のウラン燃料から大量のエネルギーを取り出すことができ,二酸化炭素の排出量も小さい.
また,比較的政情が安定した国から燃料となるウランを輸入することができることもメリットになる.
一方で,原子力発電や燃料加工施設からは様々な放射性廃棄物が発生することが大きな問題となる.
中でも,使用済み核燃料やその処理の結果として生じる高レベル放射性廃棄物(High-level radio active waste: HLW)は
放射能レベルが極めて高く,恒久的な処分にあたり多くの解決すべき課題が残されている.
HLWが,天然ウラン鉱石の放射線レベルと同程度まで下がるには,数万年以上の期間が必要と言われ,
このことがHLWの恒久的な処分を困難にしている\cite{Fujiie}.
HLWの処分方法としてこれまでに種々のものが検討されてきた.その中で,
地下300m以深の岩盤内に埋設する地層処分が,我が国を含め,現在国際的に最も
現実的な方法と考えられ,処分事業が計画あるいは実施されている\cite{NUMO_URL,NUMO}.
地層処分で想定する地下深部では,地表付近と比べ地質学的に安定しており地下水の動きも遅い.
そのため廃棄物を長期的に安定して保管することや,廃棄体の劣化によって漏洩した放射性物質が,
地下水の作用で生活圏に到達するまでの間に安全な放射線レベルまで下がるだけの時間を確保できると
考えられている.また,十分な深度に処分することで,テロ攻撃の対象となることや,後の人類が
偶然に掘削してしまうことを避けることもできる.\\

我が国では,原子力発電の使用済み燃料は再処理され,プルトニウム等の有用元素は
回収して発電に再利用される.使用済み燃料の再処理によって生じる高レベル放射性廃液
はガラス固化した上で金属製のキャニスターに封入される.ガラス固化体は作成当初,
非常に発熱量が高いため,地上の中間貯蔵施設で30$\sim$50年程度冷却貯蔵される.
地層処分では,冷却後,発熱量の下がったガラス固化体を炭素鋼のオーバーパックで覆い
地下300m以深の地層中に埋設処分する.このとき,オーバーパックと周辺岩盤の間には
緩衝材が充填される.緩衝材の役割には,廃棄体定置後に残る処分項内の空隙を充填する
ことに加え,岩盤からの地下水の浸透を抑制すること,廃棄対から漏洩した放射性核種を
吸着して地下水への流入を抑制すること,周辺岩盤の変形を吸収して廃棄体に加わる応力を
軽減することが挙げられる.これらの役割を果たすことのできる材料として,ベントナイト
の利用が計画されている.ベントナイトは,モンモリロナイトを主成分とする粘土で,
吸水によって膨潤するために空隙の充填性能に優れる.また,モンモリロナイトは非常に微細な
鉱物のため透水性が非常に低く,さらに陽イオン交換性があり物質の吸着性能にも優れている.
以上のような性質を持つことから,ベントナイトは地層処分における緩衝材として最適と考えられている.\\

ベントナイト緩衝材は処分場に定置される時点では不飽和状態にあり,定置後,地下水の浸潤
により次第に飽和状態へと遷移する.この過程を再冠水と呼ぶ.
緩衝材中の水分は各種の反応や物質の輸送に影響を与えるため,再冠水時の水分浸透挙動を
知ることは重要である.また,再冠水により次第に変化する水分状態に応じて
物質の輸送特性がどのように変化するかを明らかにすることも,ベントナイト緩衝材中における
放射性核種の移行挙動や緩衝材の性能評価において重要となる.
このことから,ベントナイトの物質輸送特性を調べることを目的として,
様々な核種についてこれまで物質拡散実験が行われてきた.
ベントナイト中の物質移動は非常に遅いプロセスであることから,拡散実験の実施には
多大な時間と労力を要する.実験はこれまで飽和粘土を対象として行われているが,
同様な実験を不飽粘土について行う場合には,所定の不飽和状態を長期間維持する機構が
あらたに必要となり,その実施は一層困難なものとなると予想される.
このことから,飽和,不飽和によらず,粘土における物質輸送特性をその場かつ非破壊的に
調べることのできる簡便な方法があれば非常に有用といえる.
しかしながら,そのような方法は殆ど無く,放射性核種をトレーサーとして行う拡散実験が
挙げられる程度である.この方法では,トレーサー物質からの線量を計測することで,
物質の存在量の時空間的な情報をその場で得ることができる.ただし,実験は管理区域で行う必要があり,
非放射性核種の拡散挙動は調べることができない.また,トレーサー物質の輸送状況は調べられるが,
物質輸送に影響を与える拡散媒体の構造や物性に関する情報を得ることはできない.
これに対して,物質の電気伝導度あるいは電気抵抗を計測する方法は,
電荷を帯びた物質の易動度に関する情報を比較的簡単な方法で測定することができる.
特に,交流電圧に対する応答を調べる電気化学インピーダンス法では,
周波数による応答電流の変化を見ることで,物質のもつ複数の抵抗要素を分離して調べる
ことができる.例えば,金属材料の抵抗は,粒界抵抗だけでなく分極現象も影響するが,
直流電圧に対する応答では各種の伝導度や抵抗要素を分離して評価することができない.
一方,電気化学インピーダンス法では,印加する交流電圧の周波数を掃引してインピーダンススペクトル
を計測することで,異なる抵抗要素をスペクトル上の異なる特徴として捉えることができる.
従って,要因ごとに抵抗や伝導度を評価することができ,伝導性物質の易動度に関係する
物質の構造や物性についての情報が得られる.
\section{研究の目的}
以上のことから,電気化学インピーダンス法を用いれば,粘土中の物質輸送や水分に関する情報を
得られる可能性がある.電気化学インピーダンスの計測は簡単な装置で非破壊的に行うことができ,
計測を比較的短時間で済ませることができる.そのため,この方法を不飽和粘土の水や物質輸送に
関する効率的かつ簡便な検査方法として利用できる可能性がある.しかしながら,粘土含水系の
電気化学インピーダンスを調べる研究はこれまであまり行われていない.特に,固体状の粘土試料
についてインピーダンス計測を行った報告はなく,計測方法や結果の解釈についてこれまで
明らかにされていない.そこで本研究では,固体粘土試料を対象とした電気化学インピーダンス
の計測方法を開発して計測を実施し,不飽和粘土のインピーダンス特性を明らかにする.
インピーダンスの計測は,水分量と密度の異なる複数の試料を用いて行い,
インピーダンススペクトルの特徴と変化を調べる.その結果を踏まえ,インピーダンス計測によって
不飽和粘土の物質輸送に関するどのような情報が得られるか,また,それに与える水分と密度の
影響を明らかにする.
\section{本論文の構成}
本論文の構成は以下の通りである.
本章で述べた研究背景と目的に続き,第二章では本研究で対象とする粘土鉱物であるモンモリロナイトの
鉱物学的な特徴について述べる.次に,電気化学インピーダンス法の基礎理論を第三章において示す.
ここでは,各種の等価回路に加え,結果の表示方法について述べる.
第四章では本研究で行った実験の方法を,実験供試体(不飽和粘土試料)の作成,計測セルと実験系の
構成,計測条件の順に述べた後に計測結果を示す.続く第五章では計測した結果をもとに
等価回路の推定を行い,回路素子定数の推定結果をあわせ,インピーダンススペクトルの解釈について議論する.
最終章である第六章では,本研究で得られた知見をまとめ,今後の課題と併せて結論を示す.



\chapter{スメクタイト族粘土の結晶構造,分類,特徴}
	\section{モンモリロナイトの鉱物学的な特徴}
\subsection{ベントナイト}
ベントナイトは,今から数百万年から数億年前の火山噴火によって堆積した火山灰などが
熱水などと反応し,温度や圧力による変性を受けて鉱床を形成したと考えられている。
ベントナイトは粘着性や吸水性や吸着性に優れ,建設や化学工学を始めとする各種産業分野
で利用されている。ベントナイトの主成分はモンモリロナイトであり,モンモリロナイトが
ベントナイトの性質を決定していると言ってよい.例えば,ベントナイトは膨張性や増粘性を
示す他、水中ではほぼ単結晶にまで分離して分散する.このような性質は,モンモリロナイト
表面に形成される厚い水和層に起因したものである.他にも、ベントナイトの吸着性の一部は,
モンモリロナイト層の間に存在する交換性の陽イオンによるなど,ベントナイトの特異な
性質は概ねモンモリロナイトの挙動によって生じている.このことから,ベントナイトの
性質を理解するためには,モンモリロナイト含水系の挙動を詳しく調べることが必要となる.
\subsection{モンモリロナイト}
粘土鉱物にはスメクタイトを始めとする結晶質鉱物と、イモゴライト等の非晶質鉱物がある。
結晶質の粘土は層状ケイ酸塩(フィロケイ酸塩)の一種で、SiO$^4$四面体シートと
Al(OH)$_6$八面体が積層して一つの結晶を作っている.
SiO$^4$四面体シートは,Si$^{4+}$に配位したO$^{2-}$が四面体を作り、四面体どうしは
頂点酸素を共有して六角網状につながっている.
Al(OH)$_6$八面体層はAl$^{3+}$を中心に6つのOH$^-$あるいはO$^{2-}$が八面体を作り,
八面体は稜を共有してシートを形成する。二種類のシートは、アルミナ四面体の頂点酸素を
共有することで結合するが、四面体の六角網の中心には$O^{2-}$は無いため,
ここにOH$^-$が入る.

モンモリロナイトは層状ケイ酸塩鉱物の一種であるスメクタイトに分類される粘土鉱物である。
図-2.1 に示すように結晶構造はケイ酸四面体層-アルミナ八面体層-ケイ酸四面体層の3層が
積み重なり,単位結晶は厚みが約 1nm,幅が 100-1000nm のとても薄い板状の結晶をしている。実
際は,この薄い板状の単位結晶が数枚積み重なり1つの鉱物粒子を作っている。図-2.2 に結晶構
造の模式図を示す。また,水との分散性と親和性があり,これによって膨潤性などの特徴的な性質
をもつ。アルミナ八面体層の中心原子である Al の一部が Mg に置換されることで陽電荷不足とな
り,各結晶層全体は負に帯電する。しかし結晶層間に Na + ,K + ,Ca 2+ ,Mg 2+ などの陽イオンを挟むこと
で電荷不足を中和し,モンモリロナイトは安定状態となる。この層間陽イオンは容易に交換され
る性質を持っており,水分子を容易に取り込む特性がある。そのため,モンモリロナイトは結晶層
が何重も重なり合った状態で存在しあい,層表面の負電荷及び層間陽イオンが様々な作用を起こ
すことによって,モンモリロナイトの特異的性質は発揮される。



2.1 スメクタイト族粘土鉱物の性質
 2.1.1 スメクタイト
  シート結合の型,層電荷,八面体シート型,さらに同型置換による組成の違いによって,粘土鉱物が分類される。2:1型粘土鉱物であり, 層電荷0.2-0.6の範囲のものをスメクタイトという。2八面体型スメクタイトには次の3種類に分類される。
モンモリロナイト :
バイデライト     :
ノントロナイト   :
Eは交換性陽イオンを一価として表したものである。モンモリロナイトでは四面体シートにはほとんど同型置換がなく,層電荷は八面体シートでの同型置換によって生じている。
。 
2.1.2 モンモリロナイト
ベントナイトの主成分であるモンモリロナイトは、層状ケイ酸塩鉱物の一種であるスメクタイトに分類される粘土鉱物である。図-2.1に示すように結晶構造はケイ酸四面体層-アルミナ八面体層-ケイ酸四面体層の3層が積み重なり,単位結晶は厚みが約1nm,幅が100-1000nmのとても薄い板状の結晶である。実際は,この薄い板状の単位結晶が数枚積み重なり1つの鉱物粒子をつくっている。図-2.2に模式図と層間距離を示す。また,水との分散性と親和性があり,これによって膨潤性などの特徴的な性質をもつ。
アルミナ八面体層の中心原子であるAlの一部がMgに置換されることで陽電荷不足となり,各結晶層全体は負に帯電する。しかし結晶層間に・・・などの陽イオンを挟むことで電荷不足を中和し,モンモリロナイトは安定状態となる。この層間陽イオンは容易に交換される性質を持っており,水分子を容易に取り込む特性がある。そのため,モンモリロナイトは結晶層が何重も重なり合った状態で存在しあい,層表面の負電荷及び層間陽イオンが様々な作用を起こすことによって,モンモリロナイトの特異的性質は発揮される。


2.1.3 ベントナイトの性質
 ベントナイトは,モンモリロナイトという粘土鉱物を主成分とする粘土であり,ベントナイトを利用することは,モンモリロナイトの特徴とほぼ同義である。モンモリロナイトは,その層間に入っている交換性陽イオンにより,ナトリウム型やカルシウム型などと呼ばれている。カルシウム型モンモリロナイトは吸水性に優れているものの,ナトリウム型モンモリロナイトに比べて膨潤性.増粘性,懸濁安定性の面で劣る。そこで緩衝材として膨潤性能,止水性能に優れた性質を有している面で,ナトリウム型モンモリロナイトを含有するベントナイトが地層処分のバリア材として採用が検討されている。
本研究で用いるベントナイトはモンモリロナイトを主成分とするナトリウム型ベントナイトとし,ナトリウム型ベントナイトの中でもクニミネ工業(株)製クニピアFとする。図2.3にクニピアFの物理,化学特性を示す。



\chapter{電気化学インピーダンス法の基礎}
	\chapter{電気化学インピーダンス法の基礎}
\section{インピーダンス}
電気化学インピーダンス法では試料に交流電圧を印加し,その応答として生じる電流を計測する.
電流$I$と電圧$V$の関係が線形システムとみなせるならば,両者の関係は
\begin{equation}
	V(\omega)= Z(\omega) I(\omega), \ \ (\omega=2\pi f)
	\label{eqn:I2V}
\end{equation}
と表すことができる.ここに,$\omega$と$f$はそれぞれ交流電圧の角周波数[rad/s]と周波数[Hz]を表す.
なお,時間因子$e^{i\omega t}$は両辺に共通のため省略する.
このとき,電流と電圧の比である$Z$はインピーダンスと呼ばれる.
$Z$も周波数の関数であることから,$Z$はインピーダンススペクトルとも呼ばれる.
インピーダンス$Z(\omega)$は一般に複素数で,虚数単位を$i$として
\begin{equation}
	Z(\omega)=Z'(\omega)+iZ''(\omega)
	\label{eqn:Z_cmplx}
\end{equation}
と実部,虚部を書く.$Z'$はレジスタンス,$Z''$はリアクタンスと呼ばれる.
インピーダンスは指数関数を用いて
\begin{equation}
	Z(\omega)=\left| Z \right|(\omega)e ^{i\phi(\omega)}
	\label{eqn:}
\end{equation}
と表すこともできる.ここで$\phi$は複素数$Z$の偏角を表し,
$\phi$は電流と電圧の間の位相遅れを意味する.
%
\section{インピーダンススペクトルの表示方法}
インピーダンススペクトルの表示は目的に応じて二種類の方法が用いられる.
1つめの方法では横軸に周波数を,縦軸にインピーダンスの大きさ$|Z|$をとり
両対数グラフとして表示するものである.同時に,横軸を対数軸として周波数に,
縦軸を$Z$の偏角$\phi$としたグラフを合わせて見ることで,
インピーダンススペクトルの完全な情報を示すことができる.
これらの2つのグラフによる表示をBode(ボード)線図と呼ぶ.
もう一つの方法は,横軸にレジスタンス$R'$を,縦軸をリアクタンスの符号を反転
させた$-R''$とした複素平面にインピーダンス$Z$をプロットするものである.
各周波数におけるインピーダンスをこのような複素平面上にプロットすれば,
インピーダンススペクトルが複素平面上の曲線として表される.このような
インピーダンススペクトルの表示はNyquist(ナイキスト)線図と呼ばれる.
Nyquist線図はインピーダンススペクトの特徴をしばしば直感的にわかり易く示してくれる.
ただし,周波数に対する依存性はNyquist線図上で明示的に示されない.
そのため,周波数との関係を同時に見る必要がある場合には,Bode線図も併用する
ことになる.
%
\section{等価回路}
実験やシミュレーションで得られたインピーダンススペクトルを再現できる
簡単な電気回路を,対象とする試料やモデルの等価回路(equivalent circuit)と呼ぶ.
等価回路を構成する最も基本的な回路素子には抵抗($R$),コンデンサ$(C)$,
インダクタンス$(L)$の3つがある.
回路が理想的な抵抗だけからなる場合,電流と電圧の関係はオームの法則により
\begin{equation}
	V=RI
	\label{eqn:Ohom}
\end{equation}
で与えられる.この場合,インピーダンスは$Z=R$で周波数に依らず,
ボード線図とナイキスト線図はそれぞれ図\ref{fig:fig3_1}に示したようになる.
一方,コンデンサ-だけからなる回路では,電圧と帯電量$Q$の関係:
\begin{equation}
	V=\frac{Q}{C}
	\label{eqn:Q_CV}
\end{equation}
より,
\begin{equation}
	\frac{dV}{dt}=i\omega V =\frac{1}{C}\frac{dQ}{dt}=\frac{I}{C}
	\label{eqn:}
\end{equation}
だから,インピーダンスは
\begin{equation}
	Z=\frac{1}{i\omega C}
	\label{eqn:Zc}
\end{equation}
で与えられリアクタンス成分だけを持つ.
図\ref{fig:fig3_2}はこの結果を示したナイキスト線図とボード線図である.
また,本研究で用いることは無いが,インダクタンスのみの回路については
電流と起電力の関係:
\begin{equation}
	V=-L\frac{dI}{dt}=-i\omega LI 
	\label{eqn:}
\end{equation}
より
\begin{equation}
	Z=i\omega L
	\label{eqn:}
\end{equation}
で,コンデンサと同様リアクタンス成分だけになる.
この場合のインピーダンススペクトルは図\ref{fig:fig3_3}の通りである.
なお,以上の図\ref{fig:fig3_1}-図\ref{fig:fig3_3}においてナイキスト線図
に示した青の矢印は,低周波から高周波側に進む方向を示している.
%--------------------
\begin{figure}[h]
	\begin{center}
	\includegraphics[width=0.9\linewidth]{Figs/fig3_1.eps} 
	\end{center}
	\caption{
		抵抗素子のボード線図(a),(b)とナイキスト線図(c).
	} 
	\label{fig:fig3_1}
\end{figure}
%--------------------
%--------------------
\begin{figure}[h]
	\begin{center}
	\includegraphics[width=0.9\linewidth]{Figs/fig3_2.eps} 
	\end{center}
	\caption{
		コンデンサ素子のボード線図(a),(b)とナイキスト線図(c).
	} 
	\label{fig:fig3_2}
\end{figure}
%--------------------
%--------------------
\begin{figure}[h]
	\begin{center}
	\includegraphics[width=0.9\linewidth]{Figs/fig3_3.eps} 
	\end{center}
	\caption{
		インダクター素子のボード線図(a),(b)とナイキスト線図(c).
	} 
	\label{fig:fig3_3}
\end{figure}
%--------------------
以上に述べた基本的な回路素子を組み合わせることで,より多様なインピーダンススペクトルを表現することができる.
そのような例として以下では,本研究に関連の深い5つの回路を取り上げ,そのインピーダンススペクトルを示す.
\section{合成インピーダンス}
\subsection{RC並列回路}
抵抗$R$とコンデンサ$C$を並列に接続したRC並列回路の合成インピーダンスは
\begin{equation}
	\frac{1}{Z}=\frac{1}{R} + i\omega C \ \ 
	\Rightarrow \ \ Z =\frac{R}{1+i\omega RC}
	\label{eqn:RC_para}
\end{equation}
で与えられる.従ってレジスタンス$Z'$とリアクタンス$Z''$は,それぞれ
\begin{equation}
	Z'=\frac{R}{1+(\omega RC)^2}, \ \ 
	Z''=-\frac{i\omega RC}{1+(\omega R^2C)^2} 
	\label{eqn:}
\end{equation}
となる.また,これらの式から$\omega$を消去すれば,
\begin{equation}
	\left( Z'-\frac{R}{2}\right)^2 +\left(Z''\right)^2 =\left( \frac{R}{2}\right)^2
	\label{eqn:}
\end{equation}
が得られ,式(\ref{eqn:RC_para})のナイキスト線図は中心が$\left( 0, \frac{R}{2}\right)$,
半径が$\frac{R}{2}$の半円を描くことが分かる.
以上より,RC並列回路の合成インピーダンスは図\ref{fig:fig3_4}のようになる.
RC並列回路が描くナイキスト線図の半円は容量性の半円と呼ばれ,実験で得られたスペクトルを
よく再現することがある.
例えば,電解液のインピーダンスには容量性の半円が現れることが知られている.
これは,電極表面に電気二重層が形成されてコンデンサの役割を果たすこと,
電解液から電極への電荷の移動反応に伴う抵抗が存在することの両者の効果が,
計測されるためである.
このように,等価回路の回路素子は物理的な実体として存在するわけでは無いが,
それと同様な効果をもつ電気化学的な過程が存在することを示す.
またそののような電気化学的プロセスの影響を回路定数として定量的に表現できるという
意味でも有用なものとなる.
\subsection{Cole-Coleプロット}
実際の計測データでは容量性半円が真円ではなく縦方向につぶれたような
形状のしたスペクトルが得られることがある.
そのようなナイキスト線図を再現するインピーダンスには,次の
ものが知られている.
\begin{equation}
	Z =\frac{R}{1+\left(i\omega RC\right)^p}
	\label{eqn:CC}
\end{equation}
図\ref{fig:fig3_5}は式(\ref{eqn:CC})のスペクトルを示したもので,
このナイキスト線図はCole-Coleプロットと呼ばれる.
この図には式(\ref{eqn:CC})の指数$p$が$p=0.5,0.8$および1.0の
場合を示している.$p=1.0$の場合は式(\ref{eqn:RC_para})に一致する
ため,式(\ref{eqn:CC})はRC並列回路の一般化と見ることもできる.
図\ref{fig:fig3_5}に示したように,$p$が小さくなるにつれ
半円がより扁平なものになる.
\subsection{Cole-Davidsonプロット}
ナイキスト線図が扁平な円となる別のインピーダンスには,次のものもある.
\begin{equation}
	Z =\frac{R}{\left(1+i\omega RC\right)^p}
	\label{eqn:CD}
\end{equation}
こちらも$p=1$の場合はRC並列回路となる意味で,RC並列並列回路の
一般化とみることができる.式(\ref{eqn:CD})で与えられるスペクトルを$p=0.5,0.8,1.0$の
ケースについて示すと図\ref{fig:fig3_6}のようになり,
このときのナイキスト線図はCole-Davidsonプロットと呼ばれる.
Cole-Davidsonプロットでは,$p<1$のとき半円の左側が
より大きく歪んだ形となる.
%--------------------
\begin{figure}[h]
	\begin{center}
	\includegraphics[width=0.9\linewidth]{Figs/fig3_4.eps} 
	\end{center}
	\caption{
		RC並列回路のボーデおよびナイキスト線図.
	} 
	\label{fig:fig3_4}
\end{figure}
%--------------------
%--------------------
\begin{figure}[h]
	\begin{center}
	\includegraphics[width=0.9\linewidth]{Figs/fig3_5.eps} 
	\end{center}
	\caption{
		Cole-Coleプロットとそのボード線図.
	} 
	\label{fig:fig3_5}
\end{figure}
%--------------------1
%--------------------
\begin{figure}[h]
	\begin{center}
	\includegraphics[width=0.9\linewidth]{Figs/fig3_6.eps} 
	\end{center}
	\caption{
		Cole-Davidsonプロットとそのボード線図.
	} 
	\label{fig:fig3_6}
\end{figure}
%--------------------
\subsection{Constant phase element (CPE)}
コンデンサーのインピーダンス特性を一般化したものには,
Constant phase element(CPE)がある.
これは,二つのパラメータ$T_{CPE}$と$p$をもつ,
次のようなインピーダンスを持つ回路素子として定義される.
\begin{equation}
	Z=\frac{1}{(i\omega)^pT_{CPE}}
	=\frac{1}{\omega^p T_{CPE}} 
	\left( 
		\cos\left(\frac{\pi p}{2}\right) 
		-
		i
		\sin\left(\frac{\pi p}{2}\right) 
	\right)
	=\frac{1}{\omega^p T_{CPE}}e^{-\frac{i\pi p}{2}} 
	\label{eqn:Z_CPE}
\end{equation}
CPEの位相は
\begin{equation}
	\phi=\frac{\pi p}{2}
	\label{eqn:}
\end{equation}
となり一定で周波数に依らず,ナイキスト線図は
一定の傾き$\frac{\pi p}{2}$の直線となる.
図\ref{fig:fig3_7}はこの様子を示したものである.
Cole-Coleプロット(\ref{eqn:CC})は,
抵抗とCPEの並列とみなすこともできる.
%--------------------
\begin{figure}[h]
	\begin{center}
	\includegraphics[width=0.9\linewidth]{Figs/fig3_7.eps} 
	\end{center}
	\caption{
		Constant phase element(CPE)のインピーダンススペクトル.
	} 
	\label{fig:fig3_7}
\end{figure}
%--------------------
\subsection{ワールブルグインピーダンス}
電荷移動が拡散によって律速されるとき,インピーダンススペクトルは,
実数軸に対して45$^\circ$の傾きを持つ直線を描くことが知られている.
この場合,インピーダンスはCPEの特別な場合として
\begin{equation}
	Z=\frac{\sigma}{\sqrt{i\omega}}
	=
	\frac{\sigma}{\sqrt{\omega}}e^{-\frac{i\pi}{4}}
	\label{eqn:Zb}
\end{equation}
で与えられる.式(\ref{eqn:Zb})はワールブルグ(Warburg)インピーダンスと
呼ばれ,そのスペクトルは図\ref{fig:fig3_8}のオレンジの線で示したようになる.
式(\ref{eqn:Zb})のインピーダンスとなることは,
半無限区間での1次元拡散方程式の解から得られる荷電物質の濃度変調から
電位を求め,質量フラックスから電流変調を求めて電位との関係を調べることで
得られる.
ただし,実際には拡散場は電極表面から無限遠方まで続くとは限らず,
有限な拡散層が形成される場合もある.例えば,電解質内で流れがある場合,
拡散層は無限に成長することは出来ず有限な大きさとなる.
この場合,1次元拡散方程式の境界値問題を拡散層の区間で解き,
電流と電圧変調の関係を求めれば,インピーダンスが
次のようになることが示される.
\begin{equation}
	Z=
	\frac{\sigma}{\sqrt{\omega}}e^{-\frac{i\pi}{4}}
	\tanh \left( \delta \sqrt{\frac{i\omega}{D}}\right)
	\label{eqn:ZbF}
\end{equation}
ここに,$\delta$は拡散層の厚さを,$D$は拡散係数を表す.
式(\ref{eqn:ZbF})もワールブルグインピーダンスと呼ばれ,
式(\ref{eqn:Zb})との区別が必要な場合には,有限ワールブルグインピーダンス
と呼ばることもある.
有限ワールブルグインピーダンスのスペクトルは,
図\ref{fig:fig3_8}において青の実線で示したようになる.
ナイキスト線図は,高周波側では傾きが45$^\circ$の直線に漸近し,
拡散層厚が無限大の場合の挙動に近づく.
一方,低周波側ではリアクタンス成分が次第に低下して,最終的には実数軸の点に収束する.
%--------------------
\begin{figure}[h]
	\begin{center}
	\includegraphics[width=0.9\linewidth]{Figs/fig3_8.eps} 
	\end{center}
	\caption{
		無限および有限な厚さの拡散層に対するワールブルグインピーダンス.
	} 
	\label{fig:fig3_8}
\end{figure}
%--------------------

\chapter{電気化学インピーダンスの測定方法と結果}
	\chapter{電気化学インピーダンス測定}
本研究では圧縮成形したペレット状の不飽和粘土を実験供試体に
用い,電気化学インピーダンスの測定を行う.
本節では供試体の作成方法についてはじめに述べ,次に,電気化学インピーダンス
の測定方法を示す.最後に,実験で得られたインピーダンススペクトルをまとめて
示す.これをもとに,次章において等価回路の推定と実験で得られたインピーダンススペクトルの解釈を行う.
\section{供試体の作成}
供試体の材料にはクニミネ工業株式会社製のクニピアFを用いる.
クニピアFは不純物を取り除いたほぼ純粋なナトリウム型モンモリロナイトで,
粉末の状態で提供されている.クニピアFの主たる物理,化学的な特性は
表\ref{tbl:kunipia}のようである.
\begin{table}[h]
\begin{center}
\caption{クニピアFの主な物理,化学特性.}
\begin{tabular}{|c||c|}
	\hline
	モンモリロナイト含有量 & 98.5\% 以上\\
	\hline
	真比重 & 2.6 \\
	\hline
	かさ密度 [g/cm$^3$]& 0.3$\sim$ 0.4 \\
	\hline 
	液性限界[\%] & 993 \\
	\hline 
	塑性限界[\%] & 42 \\
	\hline 
	塑性指数[\%] & 951 \\
	\hline 
	陽イオン交換量 [meq/100g] &  117\\
	\hline 
	浸出陽イオン [meq/100g] &  \\
	\hline 
	Na$^+$ & 114.9\\
	\hline 
	K$^+$ & 1.1\\
	\hline 
	Ca$^{2+}$ & 20.6\\
	\hline 
	Mg$^{2+}$ & 2.6 \\
	\hline 
\end{tabular}
\label{tbl:kunipia}
\end{center}
\end{table}
供試体はクニピアFと純水を所定の含水比となる比率で混合して
モールドに封入し,油圧プレスで圧縮してペレット状に成形する.
後述するように,圧縮成形に用いるモールドとピストンは電気化学インピーダンス
計測のためのセルと電極をそれぞれ兼ねる.インピーダンスの計測は油圧プレスによって
供試体を段階的に圧縮し,各圧縮段階において行う.
材料の混合と圧縮成形までの具体的な手順は以下の通りである.
\begin{enumerate}
\item
	粉末状のモンモリロナイト粘土(クニピアF)を恒温乾燥炉に入れ,110$^\circ$Cで24時間乾燥させて絶乾状態にする.
\item
	乾燥した粉末粘土($m_s=$10[g])に所定の含水比$w$とするための純水$(m_w=w m_s)$[g]を加えて十分に混合する.
	混合には小型粉砕機を用い,シリンジを用いて純水を少量ずつ投入してその都度撹拌する.このようにすることで,
	水分が粉末粘土と均等に混ざり合うようにする.
\item
	純水で混練した粘土粉末を$m_s+m_w$[g]正確に計量してモールドに入れ,ピストンを油圧プレスで
	押し込み所定の厚さ$h$まで圧縮する.
\item
	ピストンの押し込み量を保持し,供試体厚さ$h$を一定に保ったままの状態で電気化学インピーダンスの測定を行う.
\item
	インピーダンス計測の終了後,供試体をモールドから取り出し,直径$D$,厚さ$h$,湿潤質量$m_{wet}$
	を計測する.
\item
	供試体を恒温乾燥炉により24時間,110$^\circ$Cで乾燥し,供試体の乾燥質量$m_{dry}$を計測する.
\end{enumerate}
以上の手順で作成した供試体の外観を図\ref{fig:fig4_1}に示す.
%--------------------
\begin{figure}[h]
	\begin{center}
	\includegraphics[width=0.6\linewidth]{Figs/fig4_1.eps} 
	\end{center}
	\caption{
		粘土供試体の外観.含水比が(a) 17\%程度と(b)22\%程度の場合.
		含水比が低い場合の方が明るい白色を呈する.
	} 
	\label{fig:fig4_1}
\end{figure}
%--------------------
なお,供試体の正確な含水比や間隙率等の諸量は,以下のようにして算出する.\\

供試体に含まれる水分量$m_w$を
\begin{equation}
	m_w=m_{wet}-m_{dry}
	\label{eqn:}
\end{equation}
から求め,供試体の正確な含水比
\begin{equation}
	w=\frac{m_{wet}}{m_{dry}}
	\label{eqn:water_content}
\end{equation}
を得る.次に,湿潤密度$\rho_{wet}$と乾燥密度$\rho_{dry}$を供試体体積
\begin{equation}
	V=\frac{\pi D^2h}{4}
	\label{eqn:}
\end{equation}
を使い,
\begin{equation}
	\rho_{wet}=\frac{m_{wet}}{V}, \ \ 
	\rho_{dry}=\frac{m_{dry}}{V}
	\label{eqn:}
\end{equation}
の式に従ってそれぞれ求める.間隙比$e$と間隙率$n$は以上の結果から
\begin{equation}
	e=\frac{\rho_{wet}}{\rho_{dry}} -1
	\label{eqn:}
\end{equation}
と
\begin{equation}
	n=\frac{e}{1+e}
	\label{eqn:}
\end{equation}
で算出することができる.また,飽和度$S_r$も上で求めた量から
\begin{equation}
	S_r=\frac{w\rho_{clay}}{e\rho_{water}}
	\label{eqn:}
\end{equation}
によって決定することができる.ただし,$\rho_{water}$と$\rho_{clay}$は水と
モンモリロナイトの質量密度で,各々以下の数値を用いる.
\begin{equation}
	\rho_{water}=1.0{\rm [g/cm^3]}, \ \ \rho_{clay}=2.7 {\rm [g/cm^3]}
	\label{eqn:}
\end{equation}
以上のような方法で,目標含水比を
\[
	w=15,\, 17.5,\, 20,\, 22,\, 24\%
\]
の供試体5体を作成したところ,供試体の直径は29.3[mm],
厚さは9[mm], 質量12[g]程度となった.
以下では,表\ref{tbl:samples}に示すように,これらの供試体を"w15", "w17"供試体等と称する.
\begin{table}[h]
\begin{center}
\caption{供試体の名称と含水比}
	\label{tbl:samples}
\begin{tabular}{c||c|c|c|c|c}
\hline
	名称 & w15 & w17 & w20 & w22 & w24 \\
\hline
\hline
	目標含水比[\%] & 15.0 & 17.5 & 20.0 & 22.0 & 24.0 \\
\hline
	実際の含水比[\%] &  14.3 & 16.5 & 18.9 & 21.3 & 22.5  \\
\hline 
\end{tabular}
\end{center}
\end{table}
\section{インピーダンス測定方法}
電気化学反応による電流と電圧の関係は一般に非線形だが,印加交流電圧を十分に小さくすれば両者の
関係は近似的に線形とみなせる.その場合,応答関数であるインピーダンススペクトルによって
電流-電圧の関係を完全に記述できる.
ただし,ここでは物質の輸送に関する電気化学的な特性が興味の対象であるため,全周波数範囲で
インピーダンス計測を行う必要はない.物質輸送が関与する電気化学的な現象は緩和が遅いため,
その特性はインピーダンススペクトルの低周波側に現れる.そこで,100kHzまでの
インピーダンススペクトルを印加する交流電圧の周波数を掃引して計測する.
図\ref{fig:fig4_2}にインピーダンス計測実験の装置構成を示す.
%--------------------
\begin{figure}[h]
	\begin{center}
	\includegraphics[width=0.8\linewidth]{Figs/fig4_2.eps} 
	\end{center}
	\caption{
		インピーダンス測定のための実験装置の構成.
		(a)計測セル(圧縮成形のためのモールドを兼ねる), (b)粘土供試体,(c)油圧プレス, 
		(d)ケミカルインピーダンスアナライザと(e)制御およびデータ記録用PC.
	} 
	\label{fig:fig4_2}
\end{figure}
%--------------------
実験装置は,油圧プレス,計測セルとインピーダンスアナライザおよび制御PCで構成されている.
油圧プレスは供試体を所定の厚さまで圧縮成形するためのものである.
%ここで用いた油圧プレスは最大10tまでの荷重を加えることができるが,
%本研究で用いる粘土供試体は小型のものであるため,実際には300〜500[kgf]程度の荷重を
%加えるために利用する.
インピーダンス計測は供試体をモールドに入れた状態で行うため,モールドは計測セルを兼ねたものとなっている.
また,圧縮のためのピストンはSUS316Lのステンレスネジを旋盤加工して作成したもので,これをインピーダンス計測の電極としても用いる.
インピーダンスの計測は,同一の供試体を段階的に圧縮し,厚さの異なる状態において行う.
具体的には,供試体の厚さを最初に$h=10.5$mmまで圧縮してこれを初期状態とする.
その後0.5mmずつ9.0mmとなるまで供試体を圧縮し,各圧縮段階でインピダンスを3回以上計測する.
このようにすることで,含水比が一定で乾燥密度が異なる場合のインピーダンスが得られる.
なお,各圧縮段階における試料の厚さは,ダイヤルゲージでピストンの押し込み量を計測しておき,
最終的に供試体をセルから取り出したときの厚みと押し込み量からインピーダンス計測時の厚みを逆算する.
インピーダンススペクトルの測定は,市販のケミカルインピーダンスアナライザ
(HIOKI,IM3590)を用いる.周波数の掃引範囲は0.05[Hz]から100[kHz]とし,
その間を対数軸上で等間隔に分割した251の周波数でインピーダンスを測定した.
低周波数域において計測を行う場合,試料と電極の間に生じる分極が試料のインピーダンスに影響することが指摘されている.
そのため測定器と測定プローブは,測定電流により発生する磁界の影響を低減できる4端子法構造を採用した(図\ref{fig:fig4_3}).
アナライザーはノートPCから制御し,計測条件の設定と計測データの取得と保存を行う.
%--------------------
\begin{figure}[h]
	\begin{center}
	\includegraphics[width=0.6\linewidth]{Figs/fig4_3.eps} 
	\end{center}
	\caption{
		四端子法による計測のための測定端子構成.
	} 
	\label{fig:fig4_3}
\end{figure}
%--------------------
以上の計測手順をまとめると,次に示すようになる.
\begin{enumerate}
\item
	所定の含水比に調整した粘土粉末をモールドに封入してピストン/電極をとりつけて
	油圧プレスに設置する.
\item
	油圧プレスでピストン/電極を押しこみ供試体を圧縮成形する.
	このとき,供試体の厚さはおよそ10.5mmとなるようにプレス量を調整する.
\item
	ダイヤルゲージを取り付ける.このときのゲージの読みを初期値として記録する.
\item
	ピストン/電極とインピーダンスアナライザを結線する.
	印加電圧(5V),周波数掃引条件(0.1Hz$\sim$100kHz,対数軸上を等間隔に251点)を
	設定してインピーダンスを計測する.
\item
	インピーダンス計測時のダイヤルゲージの読みと,温度,湿度,参照試験体の重量を記録する.
\item
	ダイヤルゲージの値に変化が無くなった段階で,インピーダンスを3回以上,一定の時間間隔を空けて
	計測する.本研究では,予備実験の結果,1時間程度でゲージの読みに変化がなくなることが分かったため,
	圧縮後1時間経過後に30分間隔でインピーダンスの計測を行った.
\item
	ピストン/電極を0.5mm押し込み,インピダンス計測を同じ要領で行う.
 	これを所定の供試体の厚さ(9.0mm)となるまで繰り返す.
\item
	油圧プレスの荷重を解放し,ピストン/電極のリバウンド量をダイヤルゲージで測定する.
\item
	無応力状態での供試体厚さを計測する.これに,リバウンド量とダイヤルゲージの値の記録を用いて
	各圧縮段階における供試体の実際の厚さを求める.
\item
	供試体の直径,厚さ,質量を記録した後,供試体を炉乾燥する.
\item
	完全に乾燥した試料の質量を測り,乾燥密度や間隙比等,必要となる量を求める.		
\end{enumerate}
\section{実験結果}
\subsection{供試体の密度と含水比}
作成した供試体の含水比と乾燥密度を図\ref{fig:fig4_4}に示す.
ここに示した供試体の含水比と乾燥密度は,試料を乾燥した後に最終的に求められた正確なものである.
各供試体は,段階的な圧縮により乾燥密度が変化するため,この図にはインピーダンスを計測した
時点でのデータがプロットされている.
ダイヤルゲージを設置した時点を初期状態として,
そこから0.5mmずつピストンを押し込み,供試体を圧縮している.
そのため,初期状態での乾燥密度が最も低く,圧縮終了時点が最も高い値を示している.
%供試体w20, w22およびw25では,圧縮の最終段階ではほぼ飽和状態にあることが分かる.
なお,乾燥密度が供試体によって異なる理由は,
ピストンの押し込量はダイヤルゲージの読みから正確に制御できるが,
ピストンの初期位置には$\pm$0.1mm程度の誤差が生じていたことによる.
実際,乾燥密度の圧縮開始時と終了時の差はいずれの供試体でも約0.3g/cm$^3$でほぼ一致している.
供試体の実際の含水比が試料混合時の設計値より一貫して低い理由は,
混合とモールドへの封入までの間に乾燥が進むためと考えられる.
なお,モールドで圧縮している間に乾燥による質量変化がほとんど無いことは,別途作成した
参照試験片の質量を経時的に記録した結果によって確認している.
具体的には,粘土供試体をモールドに入れてピストンを取り付けた状態に
しておけば,計測期間中に質量変化はおよそ$\pm$0.01gに収まることを確認している.
%--------------------
\begin{figure}[h]
	\begin{center}
	\includegraphics[width=0.7\linewidth]{Figs/fig4_4.eps} 
	\end{center}
	\caption{
		インピーダンス測定時の供試体含水比と乾燥密度.
		破線は飽和度が一定の曲線を示す.
		} 
	\label{fig:fig4_4}
\end{figure}
%--------------------
\subsection{インピーダンススペクトル}
図\ref{fig:fig4_5}-図\ref{fig:fig4_9}に,計測で得られたインピーダンススペクトル
をナイキスト線図として示す.これらはw15からw24までの供試体に対する結果を
順に示したもので,各々の図には各圧縮段階における計測結果を全て示している.
これらの図でインピーダンススペクトルは,圧縮の進展に伴い概ね実数軸上を左方向に移動する.
ナイキスト線図の縦軸横軸の比率は等しく,グラフの傾きは正確に描画されている.
また,縦軸の表示範囲は全ての図で共通だが,含水比の低下にともないインピーダンスの
抵抗成分の値が大きくなるため,横軸の表示範囲はそれぞれの図で異なったものとなっている.
次章では,これらの結果の解釈について議論を行う.
%--------------------
\begin{figure}[h]
	\begin{center}
	\includegraphics[width=0.8\linewidth]{Figs/fig4_5.eps} 
	\end{center}
	\caption{
		供試体w15に対して計測されたインピーダンススペクトルのナイキスト線図.
	} 
	\label{fig:fig4_5}
\end{figure}
%--------------------
%--------------------
\begin{figure}[h]
	\begin{center}
	\includegraphics[width=0.8\linewidth]{Figs/fig4_6.eps} 
	\end{center}
	\caption{
		供試体w17に対して計測されたインピーダンススペクトルのナイキスト線図.
	} 
	\label{fig:fig4_6}
\end{figure}
%--------------------
%--------------------
\begin{figure}[h]
	\begin{center}
	\includegraphics[width=0.8\linewidth]{Figs/fig4_7.eps} 
	\end{center}
	\caption{
		供試体w20に対して計測されたインピーダンススペクトルのナイキスト線図.
	} 
	\label{fig:fig4_7}
\end{figure}
%--------------------
%--------------------
\begin{figure}[h]
	\begin{center}
	\includegraphics[width=0.8\linewidth]{Figs/fig4_8.eps} 
	\end{center}
	\caption{
		供試体w22に対して計測されたインピーダンススペクトルのナイキスト線図.
	} 
	\label{fig:fig4_8}
\end{figure}
%--------------------
%--------------------
\begin{figure}[h]
	\begin{center}
	\includegraphics[width=0.8\linewidth]{Figs/fig4_9.eps} 
	\end{center}
	\caption{
		供試体w24に対して計測されたインピーダンススペクトルのナイキスト線図.
	} 
	\label{fig:fig4_9}
\end{figure}
%--------------------

\chapter{考察}
	\chapter{考察}
\section{等価回路の推定}
図\ref{fig:fig5_1}に本研究で得られた典型的なナイキスト線図を示す.
黒の実線で示した計測結果には,低周波側に大きな容量性の半円(i)のおよそ半分が現れている.
これに加え,高周波側にも容量性半円の一部と思われる箇所がわずかに見られるため,
図には青の破線で想定される半円を描き込んである.
実際,100KHZ以上の周波数範囲まで計測を行うと,この部分にもはっきりとした半円が現れることを
別途実験を行い確認している.ただしここでは,物質移動に関係する長い緩和時間を持つ現象に興味があるため,
低周波側の半円(i)の挙動を表す等価回路を推定する.\\

円弧状の半円を示す等価回路の一つはRC並列回路である.
ただしRC並列回路のナイキスト線図は完全な半円を描く.
一方,図\ref{fig:fig5_1}を始めとする本研究での計測結果は,
横に扁平な半円となっているためRC並列回路でのフィッティグはできない.
そのため,Cole-ColeプロットやCole-Davidsonプロットの使用がより適切と考えられる.
いずれのプロットを選択するかにあたり半円(i)の高周波側での挙動を見ると,
図\ref{fig:fig5_1}に緑の実線で示した箇所で直線的な変化をする箇所があることに気付く.
この部分の実数軸に対する傾き$\alpha$はおよそ60$\sim$70度になっている.
直線的なインピーダンススペクトルはCPE素子で表現することができるため,
直線部分の存在はCPE素子を含む等価回路を想定すべきであることを示唆する.
Cole-ColeプロットはCPEと抵抗の並列接続で与えられることからこの条件を満足する.
一方,Cole-Davidsonプロットは,Cole-Coleプロットと類似したナイキスト線図を与えるものの,
有限個のCPE素子では表現することができない.
Cole-Davidonプロットの特徴は半円の左側半分(高周波側)が右側に比べてより歪んだ形となることである.
しかしながら,Cole-Coleプロットとの差が明らかになるのは高周波側での傾きが図\ref{fig:fig5_1}
の$\alpha$よりもかなり小さくなってからである.
以上のことから,ここではCole-Coleプロットを用いて実験で得られたインピーダンス
スペクトルのフィッティングを行う.
なお,ナイキスト線図上の直線的な変化はワールブルグインピーダンスにも現れることを
第3章において既に述べた通りで,等価回路の候補となるように思われる.
しかしながら,ワールブルグインピーダンスの直線部分の傾きは45$^\circ$のため,
ワールブルグインピーダンスだけでは今回の実験結果を再現することはできない.

Cole-Coleプロットのインピーダンスは第3章で述べたように式(\ref{eqn:CC})で与えられる.
これに,抵抗を直列に接続すれば,実数軸上の任意の位置にある半円状のスペクトルを
フィッティングすることができる.従って,ここで用いる等価回路のインピーダンスは
\begin{equation}
	Z=R_0 +\frac{R_{ct}}{1+\left( i\omega R_{ct}C\right)^p}
	\label{eqn:CC_R0}
\end{equation}
と表すことができる.式(\ref{eqn:CC_R0})において$R_0$は直列接続された抵抗成分を,
$R_{ct}$と$C$はRC並列回路の抵抗とコンデンサ成分の静電容量をそれぞれ表す.
式(\ref{eqn:CC_R0})は,CPEのインピーダンス(\ref{eqn:Z_CPE})を用い,次のように
書き直すことができる.
\begin{equation}
	Z=R_0 +\frac{R_{ct}}{1+\left( i\omega \right)^pT_{CPE}}
	\label{eqn:R_CPE}
\end{equation}
ただし,
\begin{equation}
	T_{CPE}=R_{ct}^{p-1}C^{p}
	\label{eqn:T_CPE}
\end{equation}
である.式(\ref{eqn:R_CPE})は図\ref{fig:fig5_2}の回路の合成インピーダンス
であることから,この図に示した抵抗aと抵抗(b)-CPE(c)の直-並列回路が
本研究で用いる等価回路となる.なお,$T_{CPE}$はCPE素子の時定数で,その次元は
\begin{equation}
	[T_{CPE}] =[ {\rm F\cdot s}^{p-1}]
	\label{eqn:}
\end{equation}
で,指数$p$が1のときコンデンサーの静電容量と単位も含めて一致する.
これら回路素子に対応する電気化学的な現象は,$R_0$は試料内の
電荷移動抵抗,$R_{ct}$と$T_{CPE}$はそれぞれ電極や物質の界面における
電荷移動抵抗と電気二重層の形成による分極であると言われている.
%--------------------
\begin{figure}[h]
	\begin{center}
	\includegraphics[width=0.8\linewidth]{Figs/fig5_1.eps} 
	\end{center}
	\caption{
		計測で得られたインピーダンススペクトルの特徴.	
	} 
	\label{fig:fig5_1}
\end{figure}
%--------------------
\begin{figure}[h]
	\begin{center}
	\includegraphics[width=0.4\linewidth]{Figs/fig5_2.eps} 
	\end{center}
	\caption{
		インピーダンススペクトルのフィッティングに用いる等価回路.
		$R_0,R_{ct}$および$T_{CPE}$はそれぞれの回路素子の素子定数を表す.
	} 
	\label{fig:fig5_2}
\end{figure}
%--------------------
\section{回路定数の推定方法}
図\ref{fig:fig5_2}に示した等価回路の素子定数$R_0,R_{ct},C$および$p$を
最小二乗法は用いて決定する.そこで,実験で得られたインピーダンスを$Z(\omega)$,
回帰式として用いる式(\ref{eqn:R_CPE})のインピーダンスを$\tilde{Z}(\omega)$
と書き,残差$r$を次のように定義する.
\begin{equation}
	r\left(R_0,R_{ct},C,p \right):= \frac{1}{2}\int _{W} \left| Z(\omega)-\tilde{Z}(\omega)\right|^2 d\omega
	\label{eqn:}
\end{equation}
残差$r$を最小化することで素子定数を決定する.すなわち
\begin{equation}
	\left( R_0, R_{ct}, C,p \right)= {\rm argmin} \left\{ r\left(R_0,R_{ct},C,p \right) \right\}
	\label{eqn:LSprb}
\end{equation}
で$R_0,R_{ct},C$および$p$を決定する.ここで,回帰式$\tilde Z(\omega)$は,
\begin{equation}
	\tilde Z (\omega) =R_0+\frac{R_{ct}}{g(C,p)}
	\label{eqn:}
\end{equation}
の形をしていることから,$R_0$と$R_{ct}$について線形である. 一方,$g(C,p)$の項は
\begin{equation}
	g(C,p)= 1+(i\omega)^pT_{CPE} 
	\label{eqn:}
\end{equation}
だから,$p$と$C$あるいは$T_{CPE}$について$\tilde {Z}(\omega)$は
非線形な複素数値関数となる.
以上より$(p,C)$が与えられたときであれば,$r$を最小化する$R_0$と$R_{ct}$を厳密に求めることができる.
この点を利用し本研究では,次のような繰り返し計算により,式(\ref{eqn:LSprb})の
最小二乗問題の近似解を求めた.
\begin{enumerate}
\item
	$(p,C)$の初期値を設定する.
\item
	与えられた$(p,C)$に対して$r$を最小化する$(R_0, R_{ct})$を求める.
\item
	ステップ(2)で求めた$(R_0,R_{ct})$と$C$に対し,$r$を最小化する$p$を探索して
	$p$の値を更新する.
\item
	ステップ(2)で求めた$(R_0,R_{ct})$とステップ(3)で更新した$p$に対し,$r$を最小化する$T_{CPE}$を探索して
	$T_{CPE}$の値を更新する.
\item
	現在の$(R_0,R_{ct},p,C)$に対する残差$r$が十分小さくなるか,今以上に減少しなくなるまでステップ(2)から(4)の更新を繰り返す.
\end{enumerate}
なお,ステップ(3)と(4)の探索には1次元最小化のどのようなアルゴリズムを用いてもよい.
ここでは最も単純な方法として,指定された区間内部を一定の間隔で
探索することで最小値の近似値を求めた.
ただし,探索区間は(4)のステップが終了するたびに縮小し,探索の解像度が次第に高くなるようにしている.
探索区間の初期値は$p$については$[0.4,1.0]$として区間内を50分割し,
$T_{CPE}$については$ [10^{-6}, 10^{6}]$において
対数軸上で等間隔に区間分割して最小値を求めた.
%いずれも,反復計算1回毎に区間を0.95倍に縮小した.
\section{回路定数の推定結果と考察}
実験結果から推定した等価回路の素子定数を,図\ref{fig:fig5_3}−図\ref{fig:fig5_7}に示す.
各々の図に示した2つのグラフは,(a)横軸を乾燥密度,(b)横軸を含水比とした
ものの二通りの形式で同じ量の変化を示している.
例えば,図\ref{fig:fig5_3}では,直列接続された抵抗$R_0$を
単位面積あたりの抵抗として(a)では乾燥密度との関係を,
(b)では含水比との関係を示している.
なお,同一色のドットは同じ試験体に対する結果を示している.
一つの試験体に対し,各圧縮段階で3回以上のインピーダンス計測を
行っているため,それぞれの供試体に対し15から20点程度の推定値が示されている.
以下,これらの結果について順に検討する.\\

はじめに,図\ref{fig:fig5_1}の$R_0$についてみると,直列抵抗成分は含水比の増加につれ
抵抗値を下げることが分かる.これは,水分中の荷電物質が水分量の増加につれて移動し易くなる
ことを示すものである.なお,供試体作成には純水を用い,粘土の含有量も一定であるため,
供試体ごとに電荷を持つ物質の総量は同じであることから,この結果は固体粘土の物性としての
電気伝導性を示していると考えて良い.乾燥密度との関係について言えば,含水比の低いw15やw17の
供試体では,乾燥密度が増加するについれて抵抗は下がっている.一方,含水比が相対的に高い
w20,w22,w24の場合,乾燥密度の増加による抵抗の減少幅は小さい.
このことは,電気伝導経路となる水分のネットワークが低含水比の供試体では圧縮によって
連結性が増すが,高含水比のものは当初から水分の連結性がよく,密度の増加が伝導率の増加に
大きく寄与しないことを示している.\\

%--------------------
\begin{figure}[h]
	\begin{center}
	\includegraphics[width=0.8\linewidth]{Figs/fig5_3.eps} 
	\end{center}
	\caption{
		等価回路に含まれる直列抵抗成分$R_0$.
	} 
	\label{fig:fig5_3}
\end{figure}
%--------------------
図\ref{fig:fig5_4}は,CPEと並列接続された抵抗$R_{ct}$の推定結果を示したものである.
$R_{ct}$は乾燥密度や含水比との明瞭な相関がみられない.
特徴的な点を強いてあげるとすれば,水分量の少ないw15供試体で他よりも大きな値を
取るという点がある.なお,値の大小はあるものの水分量にも密度にも$R_{ct}$
はっきりした相関を示さないということは,物質界面における電荷移動抵抗は
水分量や密度に影響を受けにくいという解釈が成り立つ.
\begin{figure}[h]
	\begin{center}
	\includegraphics[width=0.8\linewidth]{Figs/fig5_4.eps} 
	\end{center}
	\caption{
		等価回路に含まれる並列抵抗成分$R_{ct}$.
	} 
	\label{fig:fig5_4}
\end{figure}
これに対して,$R_{ct}$と対をなす量としてインピーダンスに現れる容量成分$C$の
推定結果は図\ref{fig:fig5_5}のようでなる.
こちらも水分,密度との相関は明瞭では無いが,相対的に水分の多いw22とw25供試体では
乾燥密度に対して明らかに増加する傾向がある.
これは,容量成分の発現には水分で満たされた間隙のサイズが関係することを示唆している.
ただし,$C$は$p=1$の場合を除き単一のコンデンサを表現するものでなく,
RC並列回路からのずれが大きいとき,$C$自体に回路素子としての意味をもたせることや,
具体的な電気化学的プロセスに対応付けることは難しい.\\
%--------------------
\begin{figure}[h]
	\begin{center}
	\includegraphics[width=0.8\linewidth]{Figs/fig5_5.eps} 
	\end{center}
	\caption{
		Cole-Coleプロットにおけるコンデンサの静電容量$C$.
	} 
	\label{fig:fig5_5}
\end{figure}

図\ref{fig:fig5_7}に$T_{CPE}$の推定結果を示す.
この図の(a)は横軸を乾燥密度に,(b)は含水比にとって$T_{CPE}$を
プロットしたものである.
この図から明らかなように,同じ含水比であれば乾燥密度が高い程,
$T_{CPE}$も大きな値となる.また,同じ乾燥密度でみたときには,
含水比が高い程$T_{CPE}$の値も大きい.
$T_{CPE}$はコンデンサの静電容量とは異なる次元を持つ量であるが,
蓄電量と電圧の関係を表す実数値の係数であることから,
蓄電容量を表す定数という点は同じである.
従って,図\ref{fig:fig5_7}の結果は,水分が多い程,また,
乾燥密度が大きい程,蓄電容量が増すことを意味している.
蓄電は粘土内の局所的な分極(正負電荷の分離)が生じて電気二重層を形成することによる.
このように考えると,水分の増加に伴い$T_{CPE}$の値が大きくなることは,
局所的な分極が間隙水の内部あるいは表面で発生していると判断できる.
一方,乾燥密度の増加による$T_{CPE}$の増加は,局所的な分極が
より狭い領域内に密集して生じるためと解釈できる.
同じことは,乾燥密度の増加に伴い,分極電荷の空間的な密度も高まり,蓄電容量の増加となって
現れたという言い方もできる.\\
\begin{figure}[h]
	\begin{center}
	\includegraphics[width=0.8\linewidth]{Figs/fig5_7.eps} 
	\end{center}
	\caption{
		等価回路に含まれるCPE素子の時定数$T_{CPE}$.
	} 
	\label{fig:fig5_7}
\end{figure}

最後に,CPEの指数$p$と含水比,乾燥密度との関係を図\ref{fig:fig5_6}に示す.
この図に示した含水比と$p$の関係(b)を見ると,水分の増加に対し$p$の変化は
単調でなく,含水比20\%程度で極大となっている.
次に,乾燥密度との関係(a)を含水比毎に見ると,乾燥密度に対して増加
するケースと低下するケースが混在している.
含水比が20\%以下のw15,w17およびw20では,乾燥密度の増加に応じて$p$も増加するのに対し,
高含水比側(w22とw24)では,これと逆の傾向を示している.
この結果は,含水比20\%前後では,指数$p$を決める要因となる
電気化学的現象において特別な状態が発生していることを示唆する.
そこで,指数$p$の意味についてより詳しく検討する.
CPEの電流と電圧の関係は
\begin{equation}
	V=\frac{I}{(i\omega )^pT_{CPE}}
	\label{eqn:}
\end{equation}
である.帯電量$Q$は電流を時間に関して一回積分することで得られるため,交流回路では
\begin{equation}
	Q=\frac{I}{i \omega}=(i\omega)^{p-1}T_{CPE}V=T_{CPE}Ve^{-i\frac{\pi}{2}(1-p)}
	\label{eqn:}
\end{equation}
となる.これは,帯電量$Q$は印加電圧に対して位相が
$\phi_p=\frac{\pi}{2}(1-p)$だけ遅れることを意味する.
$p=1$の場合は理想的なコンデンサを意味し,このとき$\phi_p=0$で
帯電は電圧と完全に同期する.一方,$p<1$の場合,印加した電圧よりも帯電が遅れて生じ,
そのラグは指数$p$が小さい程大きくなる.
なお,分極を生じさせる電荷移動が拡散に支配されるときは$p=0.5$となり,
このときのCPEはワールブルグインピーダンスと呼ばれている.
つまり,$0.5<p<1$では,拡散支配のもとで生じる場合分極程に遅くは無いが,
電圧変化に追従できるほど速やかなものではないことを意味する.
このことを踏まえれば,図\ref{fig:fig5_6}において$p$の下限値が概ね0.5程度で拡散
に支配され,上限値でも0.75程度で明らかな位相遅れを伴うということを示しており,
このような$p$の推定結果は合理的と考えられる.
なお,低含水比の場合に,乾燥密度に対して$p$が増加することは,
乾燥密度の上昇に伴い分極が生じやすくなると言い換えることができる.
分極が生じるためには電荷の移動が必要で,電荷の移動は間隙水を経路として起きる.
そのため,間隙水量と,間隙水ネットワークの連結性や屈曲は,分極の位相遅れ,
すなわち指数$p$の値に反映される.以上のことを踏まえれば,低含水比側での乾燥密度に
対する$p$の増加は,乾燥密度が増すことによって間隙水の連結性がよくなることを示すと解釈できる.
一方,高含水比側での乾燥密度に対する$p$の減少は,間隙水の連結性が向上する効果を,
圧縮によって生じる間隙水ネットワークの屈曲や閉塞の効果が上回り,
結果的には電荷移動が抑制されるためと考えることができる.
図\ref{fig:fig5_6}はこれら相反する効果が,含水比20\%程度でバランスすることで,
乾燥密度に対する$p$の変化挙動が切り替わったことを示している.
%--------------------
\begin{figure}[h]
	\begin{center}
	\includegraphics[width=0.8\linewidth]{Figs/fig5_6.eps} 
	\end{center}
	\caption{
		等価回路に含まれるCPE素子の指数$p$.
	} 
	\label{fig:fig5_6}
\end{figure}
%図\ref{fig:fig5_6}はCPEの指数$p$の推定結果を示したもので,最も興味深い挙動が現れている.
%まず,含水比との関係を見ると,水分の増加に対し$p$の応答は単調でなく,w20で極大となる結果を示している.
%含水比が同じ(すなわち同じ供試体)で指数$p$には変動が有り,この点について乾燥密度の関係をみれば,w15,w17,w20の
%低含水比側では乾燥密度の増加に応じて$p$も増加するが,高含水比ではその反対の傾向となることが示されている.
%この結果は,含水比20\%前後が,指数$p$を決める要因の特別な状態が現れていることを意味する.
%指数$p$は,理想的コンデンサーからのずれを意味し,$p=1$に近いほど理想的なコンデンサーとみなしうる.
%逆に,$p$が小さい場合は,単一のコンデンサーで挙動を説明できないことを意味する.特に$p=0.5$の場合はワールブルグインピーダンスの場合に
%相当し,荷電物質の拡散が微視的な電気二重層の形成を律速する.つまり,$p$が大きいときには一定の外部電場に対して充電が速やかかつ
%効率的に行われるが,$p$が小さい場合には,電荷の移動が拡散に支配される結果,理想的なコンデンサーのようには充電
%すなわち分極が怒らない.分極を妨げる要素には,水分が少なく伝導経路が十分に確保されないことと,
%間隙が複雑に屈曲して電場へ追従した電荷の移動が制限を受けることに2つが考えられる.
%w22やw25の指数が小さいことは,間隙の屈曲によるものと考えられる.このように考えると,乾燥密度が大きくなっても,屈曲率が増すために,
%分極率には寄与しないことも説明がつく.一方,w15やw17では水分が少なくもともと水分ネットワークの繋がりが
%良くないことが$p$が小さくなることが一つの理由と考えられる.そのため,乾燥密度が増すと,ネットワークの連結性は向上し,
%指数$p$は若干大きな値を持つようになる.しかしながら,間隙の屈曲は次第に大きくなるため,指数$p$の増加は
%頭打ちになり,水分が多い場合ほどの値にはならない.これに対して,w20は適切な水分量で,無理なく水和粘土が
%締め固められたために,元々間隙の屈曲が小さく圧縮されることによって荷電物質の移動する一定の直線距離を移動するための経路長が
%より短くなることで,分極が起こりやすくなると考えられる.これは巨視的には最適含水比に近い状態で締め固められたことによると言うこともできる.
%--------------------
%最後に,CPEの時定数$T_{CPE}$の推定結果を図\ref{fig:fig5_9}に示す.この結果は,水分,密度と最も相関の高い挙動を示す.
%乾燥密度,含水比のいずれに対しても$T_{CPE}$はほぼ単調に増加している.すなわち,乾燥密度が同じであれば,水分が多い方が,
%水分量が同じならば乾燥密度が高い方が$T_{CPE}$の値が大きくなっている.$T_{CPE}$は一般化されたコンデンサーの容量を表す.
%従って,水分と密度に応じて,コンデンサー成分が支配的になることを意味する・また,微視的なコンデンサーの総量が,密度や水分といったマクロ量で
%スケールし,微視的な構造に依らないことを示すという点で非常に興味深い結果と言える.
%なお,$p$もCPEの性質を決めるパラメータだが,$T_{CPE}$は誘電体の分極率にあたり,$p$は分極の緩和時間に関する
%パラメータであるため,両者が水分量と密度に対して同じように変化する必然性は無い.ここでの結果は,$T_{CPE}$は微視構造をあまり反映せず,
%一方$p$は間隙ネットワークの屈曲率に影響されることを示唆するという点でで微視的構造を反映するパラメータということができる.


\end{document}
%\begin{figure}[here]
\begin{figure}
	\vspace{-3mm}
	\begin{center}
%	\includegraphics[width=0.45\linewidth]{fig1.eps} 
	\end{center}
	\vspace{-5mm}
	\caption{一端を固定壁に支持された棒部材とそのひずみ分布.} 
	\label{fig:fig1}
\end{figure}
%%%%%%%%%%%%%%%%%%%%%%%%%%%%%%%%%%%%%%%%%%%%

