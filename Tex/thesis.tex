\documentclass[11pt,a4j]{mybook2}
\usepackage[top=2.5cm, bottom=2.5cm, left=2cm, right=2cm]{geometry}
%\documentclass[11pt,a4j]{jbook}
%\usepackage{graphicx,wrapfig}
\usepackage{graphicx,titlesec}
%\usepackage{tocloft} %目次の調整
%\setlength{\topmargin}{-1.5cm}
%\setlength{\textwidth}{16.5cm}
%\setlength{\textheight}{25.2cm}
\newlength{\minitwocolumn}
\setlength{\minitwocolumn}{0.5\textwidth}
\addtolength{\minitwocolumn}{-\columnsep}
%\addtolength{\baselineskip}{-0.1\baselineskip}
%
\def\Mmaru#1{{\ooalign{\hfil#1\/\hfil\crcr
\raise.167ex\hbox{\mathhexbox 20D}}}}
%
\newcommand{\fat}[1]{\mbox{\boldmath $#1$}}
\newcommand{\D}{\partial}
\newcommand{\w}{\omega}
\newcommand{\ga}{\alpha}
\newcommand{\gb}{\beta}
\newcommand{\gx}{\xi}
\newcommand{\gz}{\zeta}
\newcommand{\vhat}[1]{\hat{\fat{#1}}}
\newcommand{\spc}{\vspace{0.7\baselineskip}}
\newcommand{\halfspc}{\vspace{0.3\baselineskip}}
\bibliographystyle{unsrt}
\newcommand{\twofig}[2]
 {
   \begin{figure}
     \begin{minipage}[t]{\minitwocolumn}
         \begin{center}   #1
         \end{center}
     \end{minipage}
         \hspace{\columnsep}
     \begin{minipage}[t]{\minitwocolumn}
         \begin{center} #2
         \end{center}
     \end{minipage}
   \end{figure}
 }
%\titleformat{\chapter}[display]{\normalfont\normalsize}{\chaptertitlename \thechapter 章}{20pt}{\normalsize}
%{\normalsize}
%\vspace*{\baselineskip}
%\renewcommand{\cfttoctitlefont}{\hfill\normalsize\bfseries}
%\renewcommand{\cftaftertoctitle}{\hfill\null}
\renewcommand{\labelenumi}{(\arabic{enumi})}

\title{
\vspace{20mm}
圧縮成形された不飽和粘土の\\
電気化学インピーダンス特性に関する研究
\\
\vspace{5mm}
Study on Chemical Impedance Characteristics 
of Unsaturated Compacted Clay
\vspace{60mm}
}
%\date{\today}
\date{2021年2月9日}
\author{
	\vspace{40mm}
岡山大学環境理工学部\\
環境デザイン工学科 10429223\\
	佐々木 絢悟}

%\makeatletter
%\def\@evenfoot{\hfil -\thepage- \hfil}
%\makeatother
%\makeatletter
%\def\@oddfoot{\hfil -\thepage- \hfil}
%\makeatother
%\makeatletter
%\def\@oddeven{}
%\makeatother

\begin{document}
\maketitle
\tableofcontents
\frontmatter
\mainmatter
%%%%%%%%%%%%%%%%%%%%%%%%%%%%%%%%%%%%%%%%%%%%%%%%%%%%%%%%%%%%%%%%
\chapter{はじめに}
%\makeatletter
%\def\@oddhead{\hfil -\thepage- \hfil}
%\makeatother
%\makeatletter
%\def\@oddfoot{\hfil -\thepage- \hfil}
%\makeatother
%\section{変形に関する量}
\chapter{はじめに}
\section{研究の背景}
我が国は資源に乏しく,エネルギー源を海外から輸入される石油や石炭,天然ガス,ウランなどの天然資源に依存している.
東日本大震災による福島第一原子力発電所の事故が発生するまでは,電力を安定して供給するために原子力発電にも力を入れてきた.
原子力発電では少量のウラン燃料から大量のエネルギーを取り出すことができ,二酸化炭素の排出量も小さい.
また,比較的政情が安定した国から燃料となるウランを輸入することができることもメリットになる.
一方で,原子力発電や燃料加工施設からは様々な放射性廃棄物が発生することが大きな問題となる.
中でも,使用済み核燃料やその処理の結果として生じる高レベル放射性廃棄物(High-level radio active waste: HLW)は
放射能レベルが極めて高く,恒久的な処分にあたり多くの解決すべき課題が残されている.
HLWが,天然ウラン鉱石の放射線レベルと同程度まで下がるには,数万年以上の期間が必要と言われ,
このことがHLWの恒久的な処分を困難にしている\cite{Fujiie}.
HLWの処分方法としてこれまでに種々のものが検討されてきた.その中で,
地下300m以深の岩盤内に埋設する地層処分が,我が国を含め,現在国際的に最も
現実的な方法と考えられ,処分事業が計画あるいは実施されている\cite{NUMO_URL,NUMO}.
地層処分で想定する地下深部では,地表付近と比べ地質学的に安定しており地下水の動きも遅い.
そのため廃棄物を長期的に安定して保管することや,廃棄体の劣化によって漏洩した放射性物質が,
地下水の作用で生活圏に到達するまでの間に安全な放射線レベルまで下がるだけの時間を確保できると
考えられている.また,十分な深度に処分することで,テロ攻撃の対象となることや,後の人類が
偶然に掘削してしまうことを避けることもできる.\\

我が国では,原子力発電の使用済み燃料は再処理され,プルトニウム等の有用元素は
回収して発電に再利用される.使用済み燃料の再処理によって生じる高レベル放射性廃液
はガラス固化した上で金属製のキャニスターに封入される.ガラス固化体は作成当初,
非常に発熱量が高いため,地上の中間貯蔵施設で30$\sim$50年程度冷却貯蔵される.
地層処分では,冷却後,発熱量の下がったガラス固化体を炭素鋼のオーバーパックで覆い
地下300m以深の地層中に埋設処分する.このとき,オーバーパックと周辺岩盤の間には
緩衝材が充填される.緩衝材の役割には,廃棄体定置後に残る処分項内の空隙を充填する
ことに加え,岩盤からの地下水の浸透を抑制すること,廃棄対から漏洩した放射性核種を
吸着して地下水への流入を抑制すること,周辺岩盤の変形を吸収して廃棄体に加わる応力を
軽減することが挙げられる.これらの役割を果たすことのできる材料として,ベントナイト
の利用が計画されている.ベントナイトは,モンモリロナイトを主成分とする粘土で,
吸水によって膨潤するために空隙の充填性能に優れる.また,モンモリロナイトは非常に微細な
鉱物のため透水性が非常に低く,さらに陽イオン交換性があり物質の吸着性能にも優れている.
以上のような性質を持つことから,ベントナイトは地層処分における緩衝材として最適と考えられている.\\

ベントナイト緩衝材は処分場に定置される時点では不飽和状態にあり,定置後,地下水の浸潤
により次第に飽和状態へと遷移する.この過程を再冠水と呼ぶ.
緩衝材中の水分は各種の反応や物質の輸送に影響を与えるため,再冠水時の水分浸透挙動を
知ることは重要である.また,再冠水により次第に変化する水分状態に応じて
物質の輸送特性がどのように変化するかを明らかにすることも,ベントナイト緩衝材中における
放射性核種の移行挙動や緩衝材の性能評価において重要となる.
このことから,ベントナイトの物質輸送特性を調べることを目的として,
様々な核種についてこれまで物質拡散実験が行われてきた.
ベントナイト中の物質移動は非常に遅いプロセスであることから,拡散実験の実施には
多大な時間と労力を要する.実験はこれまで飽和粘土を対象として行われているが,
同様な実験を不飽粘土について行う場合には,所定の不飽和状態を長期間維持する機構が
あらたに必要となり,その実施は一層困難なものとなると予想される.
このことから,飽和,不飽和によらず,粘土における物質輸送特性をその場かつ非破壊的に
調べることのできる簡便な方法があれば非常に有用といえる.
しかしながら,そのような方法は殆ど無く,放射性核種をトレーサーとして行う拡散実験が
挙げられる程度である.この方法では,トレーサー物質からの線量を計測することで,
物質の存在量の時空間的な情報をその場で得ることができる.ただし,実験は管理区域で行う必要があり,
非放射性核種の拡散挙動は調べることができない.また,トレーサー物質の輸送状況は調べられるが,
物質輸送に影響を与える拡散媒体の構造や物性に関する情報を得ることはできない.
これに対して,物質の電気伝導度あるいは電気抵抗を計測する方法は,
電荷を帯びた物質の易動度に関する情報を比較的簡単な方法で測定することができる.
特に,交流電圧に対する応答を調べる電気化学インピーダンス法では,
周波数による応答電流の変化を見ることで,物質のもつ複数の抵抗要素を分離して調べる
ことができる.例えば,金属材料の抵抗は,粒界抵抗だけでなく分極現象も影響するが,
直流電圧に対する応答では各種の伝導度や抵抗要素を分離して評価することができない.
一方,電気化学インピーダンス法では,印加する交流電圧の周波数を掃引してインピーダンススペクトル
を計測することで,異なる抵抗要素をスペクトル上の異なる特徴として捉えることができる.
従って,要因ごとに抵抗や伝導度を評価することができ,伝導性物質の易動度に関係する
物質の構造や物性についての情報が得られる.
\section{研究の目的}
以上のことから,電気化学インピーダンス法を用いれば,粘土中の物質輸送や水分に関する情報を
得られる可能性がある.電気化学インピーダンスの計測は簡単な装置で非破壊的に行うことができ,
計測を比較的短時間で済ませることができる.そのため,この方法を不飽和粘土の水や物質輸送に
関する効率的かつ簡便な検査方法として利用できる可能性がある.しかしながら,粘土含水系の
電気化学インピーダンスを調べる研究はこれまであまり行われていない.特に,固体状の粘土試料
についてインピーダンス計測を行った報告はなく,計測方法や結果の解釈についてこれまで
明らかにされていない.そこで本研究では,固体粘土試料を対象とした電気化学インピーダンス
の計測方法を開発して計測を実施し,不飽和粘土のインピーダンス特性を明らかにする.
インピーダンスの計測は,水分量と密度の異なる複数の試料を用いて行い,
インピーダンススペクトルの特徴と変化を調べる.その結果を踏まえ,インピーダンス計測によって
不飽和粘土の物質輸送に関するどのような情報が得られるか,また,それに与える水分と密度の
影響を明らかにする.
\section{本論文の構成}
本論文の構成は以下の通りである.
本章で述べた研究背景と目的に続き,第二章では本研究で対象とする粘土鉱物であるモンモリロナイトの
鉱物学的な特徴について述べる.次に,電気化学インピーダンス法の基礎理論を第三章において示す.
ここでは,各種の等価回路に加え,結果の表示方法について述べる.
第四章では本研究で行った実験の方法を,実験供試体(不飽和粘土試料)の作成,計測セルと実験系の
構成,計測条件の順に述べた後に計測結果を示す.続く第五章では計測した結果をもとに
等価回路の推定を行い,回路素子定数の推定結果をあわせ,インピーダンススペクトルの解釈について議論する.
最終章である第六章では,本研究で得られた知見をまとめ,今後の課題と併せて結論を示す.



\chapter{鉱物学的にみた粘土結晶}
\section{モンモリロナイトの鉱物学的な特徴}
\subsection{ベントナイト}
ベントナイトは,今から数百万年から数億年前の火山噴火によって堆積した火山灰などが
熱水などと反応し,温度や圧力による変性を受けて鉱床を形成したと考えられている。
ベントナイトは粘着性や吸水性や吸着性に優れ,建設や化学工学を始めとする各種産業分野
で利用されている。ベントナイトの主成分はモンモリロナイトであり,モンモリロナイトが
ベントナイトの性質を決定していると言ってよい.例えば,ベントナイトは膨張性や増粘性を
示す他、水中ではほぼ単結晶にまで分離して分散する.このような性質は,モンモリロナイト
表面に形成される厚い水和層に起因したものである.他にも、ベントナイトの吸着性の一部は,
モンモリロナイト層の間に存在する交換性の陽イオンによるなど,ベントナイトの特異な
性質は概ねモンモリロナイトの挙動によって生じている.このことから,ベントナイトの
性質を理解するためには,モンモリロナイト含水系の挙動を詳しく調べることが必要となる.
\subsection{モンモリロナイト}
粘土鉱物にはスメクタイトを始めとする結晶質鉱物と、イモゴライト等の非晶質鉱物がある。
結晶質の粘土は層状ケイ酸塩(フィロケイ酸塩)の一種で、SiO$^4$四面体シートと
Al(OH)$_6$八面体が積層して一つの結晶を作っている.
SiO$^4$四面体シートは,Si$^{4+}$に配位したO$^{2-}$が四面体を作り、四面体どうしは
頂点酸素を共有して六角網状につながっている.
Al(OH)$_6$八面体層はAl$^{3+}$を中心に6つのOH$^-$あるいはO$^{2-}$が八面体を作り,
八面体は稜を共有してシートを形成する。二種類のシートは、アルミナ四面体の頂点酸素を
共有することで結合するが、四面体の六角網の中心には$O^{2-}$は無いため,
ここにOH$^-$が入る.

モンモリロナイトは層状ケイ酸塩鉱物の一種であるスメクタイトに分類される粘土鉱物である。
図-2.1 に示すように結晶構造はケイ酸四面体層-アルミナ八面体層-ケイ酸四面体層の3層が
積み重なり,単位結晶は厚みが約 1nm,幅が 100-1000nm のとても薄い板状の結晶をしている。実
際は,この薄い板状の単位結晶が数枚積み重なり1つの鉱物粒子を作っている。図-2.2 に結晶構
造の模式図を示す。また,水との分散性と親和性があり,これによって膨潤性などの特徴的な性質
をもつ。アルミナ八面体層の中心原子である Al の一部が Mg に置換されることで陽電荷不足とな
り,各結晶層全体は負に帯電する。しかし結晶層間に Na + ,K + ,Ca 2+ ,Mg 2+ などの陽イオンを挟むこと
で電荷不足を中和し,モンモリロナイトは安定状態となる。この層間陽イオンは容易に交換され
る性質を持っており,水分子を容易に取り込む特性がある。そのため,モンモリロナイトは結晶層
が何重も重なり合った状態で存在しあい,層表面の負電荷及び層間陽イオンが様々な作用を起こ
すことによって,モンモリロナイトの特異的性質は発揮される。



2.1 スメクタイト族粘土鉱物の性質
 2.1.1 スメクタイト
  シート結合の型,層電荷,八面体シート型,さらに同型置換による組成の違いによって,粘土鉱物が分類される。2:1型粘土鉱物であり, 層電荷0.2-0.6の範囲のものをスメクタイトという。2八面体型スメクタイトには次の3種類に分類される。
モンモリロナイト :
バイデライト     :
ノントロナイト   :
Eは交換性陽イオンを一価として表したものである。モンモリロナイトでは四面体シートにはほとんど同型置換がなく,層電荷は八面体シートでの同型置換によって生じている。
。 
2.1.2 モンモリロナイト
ベントナイトの主成分であるモンモリロナイトは、層状ケイ酸塩鉱物の一種であるスメクタイトに分類される粘土鉱物である。図-2.1に示すように結晶構造はケイ酸四面体層-アルミナ八面体層-ケイ酸四面体層の3層が積み重なり,単位結晶は厚みが約1nm,幅が100-1000nmのとても薄い板状の結晶である。実際は,この薄い板状の単位結晶が数枚積み重なり1つの鉱物粒子をつくっている。図-2.2に模式図と層間距離を示す。また,水との分散性と親和性があり,これによって膨潤性などの特徴的な性質をもつ。
アルミナ八面体層の中心原子であるAlの一部がMgに置換されることで陽電荷不足となり,各結晶層全体は負に帯電する。しかし結晶層間に・・・などの陽イオンを挟むことで電荷不足を中和し,モンモリロナイトは安定状態となる。この層間陽イオンは容易に交換される性質を持っており,水分子を容易に取り込む特性がある。そのため,モンモリロナイトは結晶層が何重も重なり合った状態で存在しあい,層表面の負電荷及び層間陽イオンが様々な作用を起こすことによって,モンモリロナイトの特異的性質は発揮される。


2.1.3 ベントナイトの性質
 ベントナイトは,モンモリロナイトという粘土鉱物を主成分とする粘土であり,ベントナイトを利用することは,モンモリロナイトの特徴とほぼ同義である。モンモリロナイトは,その層間に入っている交換性陽イオンにより,ナトリウム型やカルシウム型などと呼ばれている。カルシウム型モンモリロナイトは吸水性に優れているものの,ナトリウム型モンモリロナイトに比べて膨潤性.増粘性,懸濁安定性の面で劣る。そこで緩衝材として膨潤性能,止水性能に優れた性質を有している面で,ナトリウム型モンモリロナイトを含有するベントナイトが地層処分のバリア材として採用が検討されている。
本研究で用いるベントナイトはモンモリロナイトを主成分とするナトリウム型ベントナイトとし,ナトリウム型ベントナイトの中でもクニミネ工業(株)製クニピアFとする。図2.3にクニピアFの物理,化学特性を示す。



\end{document}
%\begin{figure}[here]
\begin{figure}
	\vspace{-3mm}
	\begin{center}
%	\includegraphics[width=0.45\linewidth]{fig1.eps} 
	\end{center}
	\vspace{-5mm}
	\caption{一端を固定壁に支持された棒部材とそのひずみ分布.} 
	\label{fig:fig1}
\end{figure}
%%%%%%%%%%%%%%%%%%%%%%%%%%%%%%%%%%%%%%%%%%%%

