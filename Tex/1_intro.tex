\section{研究の背景}
我が国はエネルギー資源に乏しく,海外から輸入される石油や石炭,天然ガス,ウランなどの天然資源に大きく依存している.
東日本大震災による福島第一原子力発電所の事故が発生するまでは,電力を安定的に供給するために,原子力発電にも力を入れてきた.
原子力発電では,少量のウラン燃料から大量のエネルギーを取り出すことができること,二酸化炭素の排出力が小さいこと,
比較的政情が安定した国から燃料となるウランを輸入することができるなどのメリットがある。
%使い終えた燃料を再処理し,回収したウランやプルトニウムを再び燃料に加工して利用することができるため,
%発電コストが安定している。またエネルギー率の向上及びエネルギーの安定供給に貢献するとともに,
%二酸化炭素排出量の低減に大きく寄与している。そのため開発・利用が積極的に進められてきた。
一方で,原子力発電や燃料加工施設からは様々な放射性廃棄物が発生する点が大きな問題となる.
中でも,使用済み核燃料の処理から生じる高レベル放射性廃棄物(High-level radio active waste: HLW)は,
放射能レベルが極めて高く,天然のウラン鉱石と同程度まで放射線レベルが下がる
には数万年以上が必要と言われている.
HLWの処分方法には様々なものが検討されてきたが,地下300m以深の岩盤内に恒久的に
埋設する方法である地層処分が,我が国を含め,現在国際的に最も現実的な方法と
考えられている.地下深部は地表付近と比べ地質学的に安定しており地下水の動きも遅い.
そのため廃棄物を長期的に安定して保管でき,さらに,廃棄体が劣化して放射性物質が
漏洩した後も,地下水によって生活圏に到達するまでの間に安全な放射線レベル
になるための時間が確保できると考えられる.
また,十分な深度に処分することで,テロ攻撃の対象となることや,後の人類が
偶然に掘削してしまう可能性も小さくすることができる.\\

我が国では,使用済み燃料は再処理され,プルトニウム等,有用な元素は回収して再利用される.
再処理の結果として残る,高レベルの放射性廃液はガラス固化して金属製のキャニスターに
封入し,地上の中間貯蔵施設で30$\sim$50年程度冷却のために貯蔵する.
冷却されて発熱量の小さくなったガラス固化体は,キャニスターに加えて炭素鋼のオーバーパックで覆われ,
地下300m以深の地層中に埋設処分される.
このとき,処分坑とオーバーパックの間には緩衝材が充填される.
緩衝材の役割には,廃棄体を定置した後の空隙を充填することに加え,
地下水の浸透を抑制すること,漏洩した放射性核種を吸着すること,
周辺岩盤の変形を吸収し,廃棄体に加わる応力を軽減することが挙げられる.
これらの役割を果たすことのできる材料として,ベントナイトの利用が
計画されている.ベントナイトは,モンモリロナイトを主成分とする粘土
で,吸水によって膨潤するため,空隙の充填性能に優れる.
また,モンモリロナイトは非常に微細な鉱物で,透水性が非常に低く,
さらに陽イオン交換性があり物質の吸着性能にも優れていることから,
地層処分における緩衝材として最適と考えられている.\\

ベントナイト緩衝材は,処分場に定置される時点では不飽和状態にあり,
その後の地下水の浸潤により飽和状態へと遷移する.この過程を再冠水と呼ぶ。
緩衝材中の水分は各種の反応や物質の輸送に影響を与えるため,
再冠水時の水分浸透挙動を知ることは重要である.また,水分の状態に応じて
物質の輸送特性がどのように変化するかを明らかにすることも,緩衝材中での
放射性核種の移行挙動を調べ,緩衝材の性能評価を行う上で重要となる.
しかしながら,モンモリロナイトは非常に微細な鉱物であり間隙水を含め
物質の移動速度は極めて遅く,物質の輸送状況を調べるための拡散実験には
には多大な労力が必要となる.
不飽和状態での拡散実験では,拡散係数は飽和時よりも小さくなると予想され、
さらに所定の不飽和状態を長期間維持する機構が必要となり,実験は一層困難なものとなる.
このことから,不飽和粘土における物質輸送特性を,その場かつ非破壊的に調べることのできる
簡便な方法があれば非常に有用といえる.しかしながら,そのような方法は殆ど無く,
放射性核種をトレーサーとして行う拡散実験が挙げられる程度である.
この方法では,放射性のトレーサー物質による線量を計測することで,
物質の存在量の時空間的な情報を得ることができる.ただし,実験は管理区域で行う必要
があり,非放射性核種の拡散挙動は調べることができない.
また,拡散状況は調べられるが,拡散媒体となる物質の構造や物性関する情報は得られない.
これに対して,物質の電気伝導度あるいは電気抵抗を計測する方法は,
電荷を帯びた物質の易動度に関する情報を比較的簡単な方法で測定することができる.
特に,交流電圧の印加に対する応答を調べる電気化学インピーダンス法では,
周波数による応答の違いを見ることで,物質のもつ複数の抵抗要素を分離して調べる
ことができる。
例えば、金属材料では,粒界での抵抗だけでなく分極現象も考慮する必要があるが、
直流電圧に対する応答では、各々の伝導度や抵抗を分離して評価することができない。
これに対して電気化学インピーダンス法では,印加交流電圧の周波数を掃引して
インピーダンスの周波数スペクトルを計測することで,異なる抵抗要素を
インピーダンススペクトル上の異なる特徴として捉えることができる.
これにより,物質の構造や物性と計測された伝導度の相関を見ることができるように
なり、内部構造や伝導性物質の輸送についてより詳しい情報を得ることができる。
\section{研究の目的}


処分場のバリア材における水の存在は,様々な反応に影響を与えるため,
地下水涵養の過程を調べるためには土壌水分挙動を把握することが必要である。
このため,緩衝材を構成する膨潤性粘土と間隙水分の相互作用について理解し,
所定の温度,圧力下で圧縮粘土内部に存在する間隙がどのような幾何学的形状や
構造等を有しているかを知る必要がある。

また従来, 野外での不飽和水分研究の大半は高周波誘電率法を用いて解析されている。
土壌における各誘電率の中で,水における誘電率は土粒子や空気に比べて極めて大きく,
土壌のもつ誘電率を求めることでその土壌の水分量を求めることが可能である。
その中で基本的な測定方法としてTDR法が挙げられる。
土壌の誘電率を直接測定し,それと土壌水分の関係を実験的に求めた
同一周波数に対してのキャリブレーション式を用いて水分量を得るものであり,
誘電率と体積水分量の関係が3次回帰式で占められていると明示されている。
しかし粘土鉱物においては水が吸着することで膨潤性や粘性を示すようになるため,
精確な水分量を調整して行うことは非常に難しい。
また粘土鉱物の間隙構造や間隙水の移動や分布には不明確な点が多い。
このため以上のような測定方法では測定が不可能となる。

このことを踏まえて本研究では粘土試料中の伝導現象を調べるため,
不飽和粘土において電気伝導度測定を行う。固体の伝導率を評価するには,
試料内部の伝導のみならず粒界や分極現象等も詳細に検討しなければならず,
伝導経路や伝導種を特定することが困難である。
しかし本研究で用いる電気化学インピーダンス法は,電極またはセルの
伝達関数としてインピーダンスまたはアドミッタンスを求め,電極・溶媒の電気的特性を評価する方法である。
溶液内での物質移動や電荷移動反応機構,電極の形状などさまざまな要素から伝達関数が決定され,
電極反応の解析などに対し有効な手段とされている。この方法を用い,複数の周波数でインピーダンス測定を行うことで,
適切な等価回路を仮定し,等価回路に含まれる各要素を分離,それぞれの伝導現象について評価することが可能となる。
インピーダンス測定による不飽和粘土の水分挙動計測を実現するためには,含水量とインピーダンスの関係を正確に知る必要がある。
そのためには,測定する不飽和圧縮ベントナイトの水分量を調整した試料を用いて供試体を作成することが重要である。
以上を踏まえて,インピーダンス計測により不飽和粘土内の水分挙動を逆推定するための基礎研究として,所定の含水比において
不飽和粘土のインピーダンス測定を行い,Cole-Coleプロットのデータを利用して不飽和粘土における等価回路を考え,
水分量と電気化学インピーダンスの関係について調べる。
