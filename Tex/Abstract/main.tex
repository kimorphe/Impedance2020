%%#!platex
%
% Example of Japanese Paper of JSCE
% for LaTeX2e users
%
% revised on 4/25/2014
%
%%%%%%%%%%%%%%%%%%%%%%%%%%%%%%%%%%%%%%%%%%%
%
% もし jis フォントメトリックを使う場合は,以下をアンコメントしてください.
% \DeclareFontShape{JY1}{mc}{m}{n}{<-> s * jis}{}
% \DeclareFontShape{JY1}{gt}{m}{n}{<-> s * jisg}{}
%
\documentclass{jsce}
%
\usepackage{epic,eepic,eepicsup}
%\usepackage{graphicx,multicol}
\usepackage{graphicx}
\usepackage{multicol}
\usepackage{amsmath}
%\usepackage{showkeys}
\usepackage{setspace}
%  amsを使う方は以下をアンコメントしてください.
%\usepackage{amssymb,amsmath}
% 英語はサポートしているかどうか不明
% \inenglish
% 学会サンプルに times とあるので指定しておきます
\usepackage{times}
%
\finalversion
%\pagestyle{empty}
%
\title{
圧縮成形された不飽和粘土の\\
電気化学インピーダンス特性に関する研究
}%
\endtitle{
STUDY ON CHEMICAL IMPEDANCE CHARACTERISTICS OF UNSATURATED COMPACTED CLAY
}
%
% emailアドレスのフォントをタイプライター体にしたい方は次行をアンコメント
% \emailstyle{\ttfamily}
% emailアドレスを公開される方は,
%% \thanks{○○○○○○\email{your_name@foo.ac.jp}}のようにしてください.
%
\author{
10429223 佐々木 絢悟
\thanks{岡山大学環境理工学部・環境デザイン工学科 (〒700-8530 岡山県岡山市北区3丁目1番地1号)}
}
\endauthor{Kengo SASAKI}
%
\abstract{
\small
本研究は圧縮成形された不飽和粘土の電気化学インピーダンス特性を実験によって調べたものである.
実験には,Na型モンモリロナイトと純水を混合して圧縮し,ペレット状に成形した供試体を用いた.
供試体は5つの異なる含水比で作成し,各々,圧縮途上で所定の厚みにおいてインピーダンスを計測する
ことで水分量と乾燥密度の影響を調べた. 実験の結果,不飽和粘土のインピーダンススペクトルは
Cole-Coleプロットで近似できることが分かった.
Cole-Coleプロットの等価回路には,互いに並列接続された抵抗とCPE(constant phase element)が
含まれ,CPEは2つの素子定数を持つ.そこで,これら2つの定数をカーブフィッテイングを行って
求めたところ,いずれの定数も含水比と乾燥密度に関して明確な相関を示すことが明らかとなった.
特に,CPE定数の一つである指数$p$は特異な挙動を示し,含水比が$20\%$を超えるときには乾燥密度
と正の相関を持つが,それ以下では負の相関を示すことが分かった.このことは,
指数$p$は不飽和粘土の間隙や間隙水の配置に関する微視的な構造変化を捉える指標に
なりうるという意味で重要な知見と言える.
}
%
\keywords{compacted clay, montmorillonite, electrochemical impedance, water content, bulk dry density}
%
\endabstract{% Yes blank line
\normalsize
This study investigates the propagation characteristics of surface wave traveling in a random heterogeneous medium. 
For this purpose, ultrasonic measurements are performed on a coarse-grained granite block as a typical randomly heterogeneous medium. In the ultrasonic testing, a line-focus transducer is used to excite ultrasonic waves, whereas a laser Doppler vibrometer is used to pickup the ultrasonic motion on the surface of the granite block. The measured waveforms are analysed in the frequency domain to evaluate the travel-time for each measurement point based on the Fermat's principle. From the ensemble of travel-times obtained thus, 
the probability distribution of the travel-time is established as a function of travel-distance. The uncertainty of the travel-time and its spatial evolution are then investigated using the standard deviation of the travel-time as a measure of the uncertainty. As a result, it was found that the uncertainty is 
approximately proportional to the mean travel-time divided by the square root travel-distance.This is a finding that would be useful for stochastic modeling of the waves in random heterogeneous media. 
}
%
% \titlepagecontrol{1}
%
%\receivedate{2019.7.19}
% \receivedate{January 15, 1991}
%
% \def\theenumi{\alph{enumi}}  % もし enumerate 最初の箇条を (a) と
% \def\labelenumi{(\theenumi)} % したい場合・・・
%
\begin{document}
\maketitle
%%%%%%%%%%%%%%%%%%%%%%%%%%%%%%%%%%%%%%%%%%%%%%%%%%%%%%%%%%%%%%%%%%%%%%%%%%%%%
\section{はじめに}
高レベル放射性廃棄物の地層処分には,ベントナイト緩衝材の利用が計画されている.
ベントナイト緩衝材には,処分坑の空隙充填や,地下水の浸透抑制,
放射性核種の吸着や移行遅延といった役割を果たすことが期待されている.
そのため,ベントナイトの膨潤性や透水性,物質輸送特性を正確に理解することは,
緩衝材の設計や施工を適切に行う上で重要である.
これらベントナイトの特性は間隙水分の状態や,間隙の量と構造によって変化する.
従って、ベントナイトの間隙構造や水分状態の把握は、膨潤や透水、物質輸送メカニズム
の理解にあたり本質的な問題となる.しかしながら,ベントナイトの主成分である
モンモリロナイトは、ナノメートルスケールの微細鉱物で、間隙や水分状態を実験で
直接観測することは困難である.このことから,X線回折試験で粘土分子の積層構造を
調べることや,分子動力学法等の数値シミュレーションによって粘土層間水の物性
を調べるといった取り組みが行われてきた.\\
\hspace{\parindent}物質の内部構造を調べる方法は種々存在するが,中でも簡便な方法の一つに
電気化学インピーダンス法がある.この方法は,試料に交流電圧を印加したときの
応答電流を計測して対象物のインピーダンスを求める.
スペクトルの特徴から物質の内部構造を推定は,
印加電圧の周波数を掃引して取得したスペクトルの特徴に基づいて行われ,
実験の実施が容易であることが利点の一つである.
インピーダンスはイオン等の帯電した物質の易動度に関する量であることから,
インピーダンスを介して試料内部構造を反映した物質の移動に関する情報が得られる.
粘土の場合について言えば,物質は間隙と間隙水のネットワーク
を経由して移動するため,インピーダンス法は間隙や間隙水の構造について
の有用な情報を取得できる可能性がある。しかしながら、固体状の粘土試料の
インピーダンス特性はこれまでほとんど明らかにされていない.
そこで本研究では、不飽和粘土のインピーダンス特性を把握することを目的として,
圧縮成形した粘土供試体のインピーダンス計測を行った.
特に,水分とかさ密度の影響を把握すべく,含水比と乾燥密度を変化させたときの
インピーダンスの挙動を調べた.以下,その結果について報告を行う.
\section{インピーダンス計測試験}
\subsection{粘土供試体の作成}
実験供試体はモンモリロナイト(クニミネ工業社製,クニピアF)と純水を所定の
含水比となるように混合し,油圧プレスで圧縮してペレット状に成形して作成した.
圧縮は粘土粉末を直径29.4mmのモールドに入れて行い,厚みがおよそ
9$\sim$10mmの円盤状ペレットとなるようにした.
供試体は含水比の異なる計5体を作成した.
これらの供試体の正確な含水比は実験終了後に、炉乾燥して乾燥重量を求めて決定した.
表-\ref{tbl:samples}は,これら5体の名称,目標および出来上がり含水比を示す.
\begin{table}[h]
\begin{center}
\caption{供試体の名称と含水比}
	\label{tbl:samples}
\begin{tabular}{c||c|c|c|c|c}
\hline
	名称 & w15 & w17 & w20 & w22 & w24 \\
\hline
\hline
	目標含水比[\%] & 15.0 & 17.5 & 20.0 & 22.0 & 24.0 \\
\hline
	実際の含水比[\%] &  14.3 & 16.5 & 18.9 & 21.3 & 22.5  \\
\hline 
\end{tabular}
\end{center}
\end{table}
いずれの供試体も実際の含水比は目標値を下回っている.
これは,粘土の混合時に一部の水分が失われたためと考えられる.
%
\subsection{インピーダンスの計測方法}
図\ref{fig:fig2}にインピーダンス計測のための装置構成を示す.
実験装置は油圧プレス,計測セル,インピーダンスアナライザおよび制御PCで構成されている.
油圧プレスは供試体を所定の厚さまで圧縮するために用いる.
インピーダンス計測は供試体をモールドに入れた状態で行うため,
モールドは計測セルを兼ねたものとなっている.
また,圧縮のためのピストンはSUS316Lのステンレスネジを
旋盤加工して作成したもので,これをインピーダンス計測の電極としても用いる.
インピーダンスの計測は,同一の供試体を段階的に圧縮し,厚さの異なる状態
において行う.具体的には,供試体の厚さを最初に$h=10.5$mmまで圧縮して
これを初期状態とする.その後0.5mmずつ9.0mmとなるまで供試体を圧縮し,
各圧縮段階でインピ-ダンスを3回以上計測する.このようにすることで,
含水比が一定で乾燥密度が異なる場合のインピーダンスが得られる.
インピーダンススペクトルの測定は,市販のケミカルインピーダンスアナライザ
(HIOKI,IM3590)を用いた.周波数の掃引範囲は0.05[Hz]から100[kHz]とし,
その間を対数軸上で等間隔に分割した251の周波数でインピーダンスを測定した.
%--------------------
\begin{figure}[h]
	\begin{center}
	\includegraphics[width=1.0\linewidth]{Figs/fig1.eps} 
	\end{center}
	\caption{
		インピーダンス計測のための実験装置.
	} 
	\label{fig:fig1}
\end{figure}
%--------------------
図\ref{fig:fig2}に,インピーダンス計測時の供試体の含水比と乾燥密度を示す.
この図には,各試験体について15点程度の点がプロットされている.
これらのデータは,インピーダンス計測時に記録した供試体の厚さから、
実験後に算出した乾燥密度である.
なお,乾燥密度が供試体によって異なる理由は,試験開始時のピストン位置に
$\pm$0.1mm程度の誤差が生じていたことによると考えられる.
\begin{figure}[h]
	\begin{center}
	\includegraphics[width=0.9\linewidth]{Figs/fig2.eps} 
	\end{center}
	\caption{
		インピーダンス計測時の乾燥密度と含水比.
		破線は飽和度一定の曲線を表す.
	} 
	\label{fig:fig2}
\end{figure}
%--------------------
\subsection{実験結果}
図\ref{fig:fig3}に,計測で得られたインピーダンススペクトルを示す.
このグラフは,横軸をインピーダンスの実部,縦軸を虚部の符号を反転
させた複素平面にインピーダンスを表示したもので,このようなグラフは
ナイキスト線図と呼ばれる.
図\ref{fig:fig3}はw17供試体に対する結果全てを示したもので、実軸上で
左側にあるものほど供試体厚さが小さく、かさ密度が高い場合の結果に相当する.
次節では,これらの結果の解釈について述べる.
\begin{figure}[h]
	\begin{center}
	\includegraphics[width=1.0\linewidth]{Figs/fig3.eps} 
	\end{center}
	\caption{
		インピーダンススペクトル(w17供試体に対する計測結果).
	} 
	\label{fig:fig3}
\end{figure}
%--------------------
\section{等価回路の推定\cite{Itagaki}}
供試体に印加した交流電圧を$V(\omega)$,角周波数を$\omega$,
応答電流を$I(\omega)$とするとき,インピーダンス$Z(\omega)$は
\begin{equation}
	Z(\omega)=\frac{V(\omega)}{I(\omega)}
	\label{eqn:def_Z}
\end{equation}
で与えられ,一般に複素数となる.実験で得られたインピーダンスを再現する
電気回路は等価回路と呼ばれる.電気化学インピーダンス法では,
計測結果から等価回路を推定し,各回路素子に対応した電気化学的
プロセスに関する情報を取得する.本節では,計測で得られたインピーダンスの
特徴を調べ,不飽和粘土の等価回路を推定する.\\
\hspace{\parindent}
図\ref{fig:fig4}は今回の実験で得られた典型的なナイキスト線図の
特徴を示したものである.黒の実線で示した計測結果には,低周波側に大きな
半円(i)のおよそ半分が現れている.
高周波側にも半円(ii)の一部がわずかに見られ,図には想定される
半円を青の破線で示している。物質移動に関係した長い緩和時間を持つ現象に
興味があるため,低周波側の半円(i)について調べる.
ナイキスト線図で半円を示す最も簡単な回路は、抵抗RとコンデンサCの
並列回路(RC並列回路)である.ただし,RC並列回路は完全な半円を描く.
一方,ここでの計測結果は横長で扁平な円弧であるため,RC並列回路でなく
Cole-ColeプロットやCole-Davidsonプロットの使用が必要となる.
ここで、半円(i)の高周波側での挙動を見ると、図\ref{fig:fig4}に緑の実線で示す
箇所で直線的な変化を示す箇所があることに気付く.
直線的なインピーダンススペクトルはコンデンサを一般化した回路素子
であるCPE(constant phase element)で表現できる.
このことはCPE素子を含む等価回路を想定すべきであることを示唆し、
Cole-Coleプロットはこの条件を満足する.
なお,CPE単体でのインピーダンス$Z_{CPE}$は、次式で与えられる.
\begin{equation}
	Z_{CPE}=\frac{1}{(i\omega )^pT_{CPE}}
\end{equation}
ここに,$T_{CPE}$はCPEの時定数を,$p$は位相遅れを決める指数で、これら2つがCPEの素子定数である.
CPEを用いれば,Cole-Coleプロットを与える等価回路を図\ref{fig:fig5}のように表すことができる.
この回路の合成インピーダンスは
次式で与えられる.
\begin{equation}
        Z=R_0 +\frac{R_{ct}}{1+\left( i\omega \right)^pT_{CPE}}
        \label{eqn:R_CPE}
\end{equation}
ここに,$R_0$と$R_{ct}$はそれぞれ直列および並列に接続された抵抗成分を表す.
本研究ではこれを粘土供試体の等価回路として用いる.
なお,$T_{CPE}$の次元はF$\cdot$s$^{p-1}$で、指数$p$が1のとき
$T_{CPE}$はコンデンサーの静電容量と単位も含めて一致する.
また,等価回路の素子定数は、実験結果に式(\ref{eqn:R_CPE})を最小二乗法で
フィッティングすることで決定する.次節では、その結果と考察を、
最も重要な回路素子であるCPEについて示す.
%--------------------
\begin{figure}[h]
	\begin{center}
	\includegraphics[width=1.0\linewidth]{Figs/fig4.eps} 
	\end{center}
	\caption{
		計測されたインピーダンススペクトルの特徴.
	} 
	\label{fig:fig4}
\end{figure}
%--------------------
\begin{figure}[h]
	\begin{center}
	\includegraphics[width=0.7\linewidth]{Figs/fig5.eps} 
	\end{center}
	\caption{
		不飽和粘土供試体の等価回路(Cole-Coleプロット).
	} 
	\label{fig:fig5}
\end{figure}
%--------------------
\section{回路素子定数の推定結果と考察}
図\ref{fig:fig6}に$T_{CPE}$の推定結果を示す.
この図の(a)は横軸を乾燥密度に,(b)は含水比にとって$T_{CPE}$を
プロットしたものである.これらのプロットにおいて、同一色のドットは
同じ試験体に対する結果,すなわち,同一含水比での結果を示している.
この図から明らかなように,同じ含水比であれば乾燥密度が高い程、
$T_{CPE}$も大きな値となる.また,同じ乾燥密度でみたときには,
含水比が高い程$T_{CPE}$の値も大きい.
$T_{CPE}$はコンデンサの静電容量とは異なる次元を持つ量であるが,
蓄電量と電圧の関係を表す実数値の係数であることから,
蓄電容量を表す定数という点は同じである.
従って、図\ref{fig:fig6}の結果は,水分が多い程、また,
乾燥密度が大きい程、蓄電容量が増すことを意味している.
蓄電は粘土内の局所的な分極(正負電荷の分離)が生じて電気二重層を形成することによる。
このように考えると,水分の増加に伴い$T_{CPE}$の値が大きくなることは,
局所的な分極が間隙水の内部あるいは表面で発生していると判断できる.
一方,乾燥密度の増加による$T_{CPE}$の増加は,局所的な分極が
より狭い領域内に密集して生じるためと解釈できる.
同じことは,乾燥密度の増加に伴い,分極電荷の空間的な密度も高まり,蓄電容量の増加となって
現れたという言い方もできる.\\
\hspace{\parindent}
最後に,CPEの指数$p$と含水比,乾燥密度との関係を図\ref{fig:fig7}に示す.
この図に示した含水比と$p$の関係(b)を見ると,水分の増加に対し$p$の変化は
単調でなく、含水比20\%程度で極大となっている.
次に,乾燥密度との関係(a)を含水比毎に見ると,乾燥密度に対して増加
するケースと低下するケースが混在している.
含水比が20\%以下のw15,w17およびw20では,乾燥密度の増加に応じて$p$も増加するのに対し、
高含水比側(w22とw24)では,これと逆の傾向を示している.
この結果は,含水比20\%前後では,指数$p$を決める要因となる
電気化学的現象において特別な状態が発生していることを示唆する.
そこで、指数$p$の意味についてより詳しく検討する.
CPEの電流と電圧の関係は
\begin{equation}
	V=\frac{I}{(i\omega )^pT_{CPE}}
	\label{eqn:}
\end{equation}
である.帯電量$Q$は電流を時間に関して一回積分することで得られるため,交流回路では
\begin{equation}
	Q=\frac{I}{i \omega}=(i\omega)^{p-1}T_{CPE}V=T_{CPE}Ve^{-i\frac{\pi}{2}(1-p)}
	\label{eqn:}
\end{equation}
となる.これは,帯電量$Q$は印加電圧に対して位相が
$\phi_p=\frac{\pi}{2}(1-p)$だけ遅れることを意味する.
$p=1$の場合は理想的なコンデンサを意味し,このとき$\phi_p=0$で
帯電は電圧と完全に同期する.一方、$p<1$の場合,印加した電圧よりも帯電が遅れて生じ,
そのラグは指数$p$が小さい程大きくなる.
なお,分極を生じさせる電荷移動が拡散に支配されるときは$p=0.5$となり,
このときのCPEはワールブルグインピーダンスと呼ばれている.
つまり,$0.5<p<1$では,拡散支配のもとで生じる場合分極程に遅くは無いが,
電圧変化に追従できるほど速やかなものではないことを意味する.
このことを踏まえれば,図\ref{fig:fig7}において$p$の下限値が概ね0.5程度で拡散
に支配され,上限値でも0.75程度で明らかな位相遅れを伴うということを示しており,
このような$p$の推定結果は合理的と考えられる.
なお,低含水比の場合に、乾燥密度に対して$p$が増加することは,
乾燥密度の上昇に伴い分極が生じやすくなると言い換えることができる。
分極が生じるためには電荷の移動が必要で,電荷の移動は間隙水を経路として起きる.
そのため,間隙水量と,間隙水ネットワークの連結性や屈曲は、分極の位相遅れ、
すなわち指数$p$の値に反映される.以上のことを踏まえれば,低含水比側での乾燥密度に
対する$p$の増加は,乾燥密度が増すことによって間隙水の連結性がよくなることを示すと解釈できる。
一方,高含水比側での乾燥密度に対する$p$の減少は,間隙水の連結性が向上する効果を,
圧縮によって生じる間隙水ネットワークの屈曲や閉塞の効果が上回り,
結果的には電荷移動が抑制されるためと考えることができる.
図\ref{fig:fig7}はこれら相反する効果が,含水比20\%程度でバランスすることで,
乾燥密度に対する$p$の変化挙動が切り替わったことを示している.
%--------------------
\begin{figure}[h]
	\begin{center}
	\includegraphics[width=1.0\linewidth]{Figs/fig6.eps} 
	\end{center}
\vspace{-3mm}
	 \caption{
		実験結果から推定したCPEの時定数$T_{CPE}$.
	} 
	\label{fig:fig6}
\end{figure}
%--------------------
\begin{figure}[h]
	\begin{center}
	\includegraphics[width=1.0\linewidth]{Figs/fig7.eps} 
	\end{center}
\vspace{-3mm}
	\caption{
		実験結果から推定したCPEの指数$p$.
	} 
	\label{fig:fig7}
\end{figure}
%--------------------
\vspace{-4mm}
\section{まとめ}
本研究では,不飽和モンモリロナイト供試体を用いて電気化学インピーダンススペクトル
の計測を行い、不飽和粘土のインピーダンスの等価回路はCPE(constant phase element)を
含む直−並列回路となることを示した.また、CPEの時定数は含水比、乾燥密度ともに正の相関があり、
CPEは間隙水中の電荷移動に対応する回路素子であることが明らかとなった。一方、CPEの指数$p$は、
印加電圧に対する分極(蓄電量)の位相遅れを表す定数で、間隙や間隙水分布の変化に反応する
指標となることが分かった.今後は、等価回路の各素子に対応する電気化学プロセスを正確に特定し、
微視的な構造と回路定数の定量的な関係をつけることが重要な課題となる。
%%%%%%%%%%%%%%%%%%%%%%%%%%%%%%%%%%%%%%%%%%%%%%%%%%%%%%%%%%%%%%%%%%%%%%%%%%%%%
%{\gt 謝辞:}
%本実験に用いた花崗岩供試体は浮田石材店代表浮田隆司氏に提供頂いた.
%また本研究の推進には,科学研究費補助金(基盤研究©課題番号\#18K04334)の補助を受けた.
%併せて謝意を表す.
%%%%%%%%%%%%%%%%%%%%%%%%%%%%%%%%%%%%%%%%%%%%%%%%%%%%%%%%%%%%%%%%%%%%%%%%%%%%%
%\newpage
%\lastpagecontrol[2cm]{13.7cm}
\begin{thebibliography}{99}
\begin{spacing}{1.175}
\bibitem{Itagaki}
	板垣 昌幸:電気化学インピーダンス法 第二版 原理・測定・解析,丸善出版, 2011. 
\end{spacing}
\end{thebibliography}
%\begin{flushright}
%	\small
%	\bf{ (Received June 24, 2020)\\
%	(Accepted November 19, 2020)}
%\end{flushright}
\end{document}

%\lastpagesettings
%\begin{minipage}[c]{13.7cm}
%\end{minipage}
%\lastpagecontrol[0cm]{13.7cm}
%\begin{multicols}{1}
%-------------------------------------------------
%-------------------------------------------------
%\end{multicols}

