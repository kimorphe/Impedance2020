\chapter{電気化学インピーダンス測定}
本研究では圧縮成形したペレット状の不飽和粘土を実験供試体に
用い,電気化学インピーダンスの測定を行う.
本節では供試体の作成方法についてはじめに述べ,次に,電気化学インピーダンス
の測定方法を示す.最後に,実験で得られたインピーダンススペクトルをまとめて
示す.これをもとに,次章において等価回路の推定と実験で得られたインピーダンススペクトルの解釈を行う.
\section{供試体の作成}
供試体の材料にはクニミネ工業株式会社製のクニピアFを用いる.
クニピアFは不純物を取り除いたほぼ純粋なナトリウム型モンモリロナイトで,
粉末の状態で提供されている.クニピアFの主たる物理,化学的な特性は
表\ref{tbl:kunipia}のようである.
\begin{table}[h]
\begin{center}
\caption{クニピアFの主な物理,化学特性.}
\begin{tabular}{|c||c|}
	\hline
	モンモリロナイト含有量 & 98.5\% 以上\\
	\hline
	真比重 & 2.6 \\
	\hline
	かさ密度 [g/cm$^3$]& 0.3$\sim$ 0.4 \\
	\hline 
	液性限界[\%] & 993 \\
	\hline 
	塑性限界[\%] & 42 \\
	\hline 
	塑性指数[\%] & 951 \\
	\hline 
	陽イオン交換量 [meq/100g] &  117\\
	\hline 
	浸出陽イオン [meq/100g] &  \\
	\hline 
	Na$^+$ & 114.9\\
	\hline 
	K$^+$ & 1.1\\
	\hline 
	Ca$^{2+}$ & 20.6\\
	\hline 
	Mg$^{2+}$ & 2.6 \\
	\hline 
\end{tabular}
\label{tbl:kunipia}
\end{center}
\end{table}
供試体はクニピアFと純水を所定の含水比となる比率で混合して
モールドに封入し,油圧プレスで圧縮してペレット状に成形する.
後述するように,圧縮成形に用いるモールドとピストンは電気化学インピーダンス
計測のためのセルと電極をそれぞれ兼ねる.インピーダンスの計測は油圧プレスによって
供試体を段階的に圧縮し,各圧縮段階において行う.
材料の混合と圧縮成形までの具体的な手順は以下の通りである.
\begin{enumerate}
\item
	粉末状のモンモリロナイト粘土(クニピアF)を恒温乾燥炉に入れ,110$^\circ$Cで24時間乾燥させて絶乾状態にする.
\item
	乾燥した粉末粘土($m_s=$10[g])に所定の含水比$w$とするための純水$(m_w=w m_s)$[g]を加えて十分に混合する.
	混合には小型粉砕機を用い,シリンジを用いて純水を少量ずつ投入してその都度撹拌する.このようにすることで,
	水分が粉末粘土と均等に混ざり合うようにする.
\item
	純水で混練した粘土粉末を$m_s+m_w$[g]正確に計量してモールドに入れ,ピストンを油圧プレスで
	押し込み所定の厚さ$h$まで圧縮する.
\item
	ピストンの押し込み量を保持し,供試体厚さ$h$を一定に保ったままの状態で電気化学インピーダンスの測定を行う.
\item
	インピーダンス計測の終了後,供試体をモールドから取り出し,直径$D$,厚さ$h$,湿潤質量$m_{wet}$
	を計測する.
\item
	供試体を恒温乾燥炉により24時間,110$^\circ$Cで乾燥し,供試体の乾燥質量$m_{dry}$を計測する.
\end{enumerate}
以上の手順で作成した供試体の外観を図\ref{fig:fig4_1}に示す.
%--------------------
\begin{figure}[h]
	\begin{center}
	\includegraphics[width=0.6\linewidth]{Figs/fig4_1.eps} 
	\end{center}
	\caption{
		粘土供試体の外観.含水比が(a) 17\%程度と(b)22\%程度の場合.
		含水比が低い場合の方が明るい白色を呈する.
	} 
	\label{fig:fig4_1}
\end{figure}
%--------------------
なお,供試体の正確な含水比や間隙率等の諸量は,以下のようにして算出する.\\

供試体に含まれる水分量$m_w$を
\begin{equation}
	m_w=m_{wet}-m_{dry}
	\label{eqn:}
\end{equation}
から求め,供試体の正確な含水比
\begin{equation}
	w=\frac{m_{wet}}{m_{dry}}
	\label{eqn:water_content}
\end{equation}
を得る.次に,湿潤密度$\rho_{wet}$と乾燥密度$\rho_{dry}$を供試体体積
\begin{equation}
	V=\frac{\pi D^2h}{4}
	\label{eqn:}
\end{equation}
を使い,
\begin{equation}
	\rho_{wet}=\frac{m_{wet}}{V}, \ \ 
	\rho_{dry}=\frac{m_{dry}}{V}
	\label{eqn:}
\end{equation}
の式に従ってそれぞれ求める.間隙比$e$と間隙率$n$は以上の結果から
\begin{equation}
	e=\frac{\rho_{wet}}{\rho_{dry}} -1
	\label{eqn:}
\end{equation}
と
\begin{equation}
	n=\frac{e}{1+e}
	\label{eqn:}
\end{equation}
で算出することができる.また,飽和度$S_r$も上で求めた量から
\begin{equation}
	S_r=\frac{w\rho_{clay}}{e\rho_{water}}
	\label{eqn:}
\end{equation}
によって決定することができる.ただし,$\rho_{water}$と$\rho_{clay}$は水と
モンモリロナイトの質量密度で,各々以下の数値を用いる.
\begin{equation}
	\rho_{water}=1.0{\rm [g/cm^3]}, \ \ \rho_{clay}=2.7 {\rm [g/cm^3]}
	\label{eqn:}
\end{equation}
以上のような方法で,目標含水比を
\[
	w=15,\, 17.5,\, 20,\, 22,\, 24\%
\]
の供試体5体を作成したところ,供試体の直径は29.3[mm],
厚さは9[mm], 質量12[g]程度となった.
以下では,表\ref{tbl:samples}に示すように,これらの供試体を"w15", "w17"供試体等と称する.
\begin{table}[h]
\begin{center}
\caption{供試体の名称と含水比}
	\label{tbl:samples}
\begin{tabular}{c||c|c|c|c|c}
\hline
	名称 & w15 & w17 & w20 & w22 & w24 \\
\hline
\hline
	目標含水比[\%] & 15.0 & 17.5 & 20.0 & 22.0 & 24.0 \\
\hline
	実際の含水比[\%] &  14.3 & 16.5 & 18.9 & 21.3 & 22.5  \\
\hline 
\end{tabular}
\end{center}
\end{table}
\section{インピーダンス測定方法}
電気化学反応による電流と電圧の関係は一般に非線形だが,印加交流電圧を十分に小さくすれば両者の
関係は近似的に線形とみなせる.その場合,応答関数であるインピーダンススペクトルによって
電流-電圧の関係を完全に記述できる.
ただし,ここでは物質の輸送に関する電気化学的な特性が興味の対象であるため,全周波数範囲で
インピーダンス計測を行う必要はない.物質輸送が関与する電気化学的な現象は緩和が遅いため,
その特性はインピーダンススペクトルの低周波側に現れる.そこで,100kHzまでの
インピーダンススペクトルを印加する交流電圧の周波数を掃引して計測する.
図\ref{fig:fig4_2}にインピーダンス計測実験の装置構成を示す.
%--------------------
\begin{figure}[h]
	\begin{center}
	\includegraphics[width=0.8\linewidth]{Figs/fig4_2.eps} 
	\end{center}
	\caption{
		インピーダンス測定のための実験装置の構成.
		(a)計測セル(圧縮成形のためのモールドを兼ねる), (b)粘土供試体,(c)油圧プレス, 
		(d)ケミカルインピーダンスアナライザと(e)制御およびデータ記録用PC.
	} 
	\label{fig:fig4_2}
\end{figure}
%--------------------
実験装置は,油圧プレス,計測セルとインピーダンスアナライザおよび制御PCで構成されている.
油圧プレスは供試体を所定の厚さまで圧縮成形するためのものである.
%ここで用いた油圧プレスは最大10tまでの荷重を加えることができるが,
%本研究で用いる粘土供試体は小型のものであるため,実際には300〜500[kgf]程度の荷重を
%加えるために利用する.
インピーダンス計測は供試体をモールドに入れた状態で行うため,モールドは計測セルを兼ねたものとなっている.
また,圧縮のためのピストンはSUS316Lのステンレスネジを旋盤加工して作成したもので,これをインピーダンス計測の電極としても用いる.
インピーダンスの計測は,同一の供試体を段階的に圧縮し,厚さの異なる状態において行う.
具体的には,供試体の厚さを最初に$h=10.5$mmまで圧縮してこれを初期状態とする.
その後0.5mmずつ9.0mmとなるまで供試体を圧縮し,各圧縮段階でインピダンスを3回以上計測する.
このようにすることで,含水比が一定で乾燥密度が異なる場合のインピーダンスが得られる.
なお,各圧縮段階における試料の厚さは,ダイヤルゲージでピストンの押し込み量を計測しておき,
最終的に供試体をセルから取り出したときの厚みと押し込み量からインピーダンス計測時の厚みを逆算する.
インピーダンススペクトルの測定は,市販のケミカルインピーダンスアナライザ
(HIOKI,IM3590)を用いる.周波数の掃引範囲は0.05[Hz]から100[kHz]とし,
その間を対数軸上で等間隔に分割した251の周波数でインピーダンスを測定した.
低周波数域において計測を行う場合,試料と電極の間に生じる分極が試料のインピーダンスに影響することが指摘されている.
そのため測定器と測定プローブは,測定電流により発生する磁界の影響を低減できる4端子法構造を採用した(図\ref{fig:fig4_3}).
アナライザーはノートPCから制御し,計測条件の設定と計測データの取得と保存を行う.
%--------------------
\begin{figure}[h]
	\begin{center}
	\includegraphics[width=0.6\linewidth]{Figs/fig4_3.eps} 
	\end{center}
	\caption{
		四端子法による計測のための測定端子構成.
	} 
	\label{fig:fig4_3}
\end{figure}
%--------------------
以上の計測手順をまとめると,次に示すようになる.
\begin{enumerate}
\item
	所定の含水比に調整した粘土粉末をモールドに封入してピストン/電極をとりつけて
	油圧プレスに設置する.
\item
	油圧プレスでピストン/電極を押しこみ供試体を圧縮成形する.
	このとき,供試体の厚さはおよそ10.5mmとなるようにプレス量を調整する.
\item
	ダイヤルゲージを取り付ける.このときのゲージの読みを初期値として記録する.
\item
	ピストン/電極とインピーダンスアナライザを結線する.
	印加電圧(5V),周波数掃引条件(0.1Hz$\sim$100kHz,対数軸上を等間隔に251点)を
	設定してインピーダンスを計測する.
\item
	インピーダンス計測時のダイヤルゲージの読みと,温度,湿度,参照試験体の重量を記録する.
\item
	ダイヤルゲージの値に変化が無くなった段階で,インピーダンスを3回以上,一定の時間間隔を空けて
	計測する.本研究では,予備実験の結果,1時間程度でゲージの読みに変化がなくなることが分かったため,
	圧縮後1時間経過後に30分間隔でインピーダンスの計測を行った.
\item
	ピストン/電極を0.5mm押し込み,インピダンス計測を同じ要領で行う.
 	これを所定の供試体の厚さ(9.0mm)となるまで繰り返す.
\item
	油圧プレスの荷重を解放し,ピストン/電極のリバウンド量をダイヤルゲージで測定する.
\item
	無応力状態での供試体厚さを計測する.これに,リバウンド量とダイヤルゲージの値の記録を用いて
	各圧縮段階における供試体の実際の厚さを求める.
\item
	供試体の直径,厚さ,質量を記録した後,供試体を炉乾燥する.
\item
	完全に乾燥した試料の質量を測り,乾燥密度や間隙比等,必要となる量を求める.		
\end{enumerate}
\section{実験結果}
\subsection{供試体の密度と含水比}
作成した供試体の含水比と乾燥密度を図\ref{fig:fig4_4}に示す.
ここに示した供試体の含水比と乾燥密度は,試料を乾燥した後に最終的に求められた正確なものである.
各供試体は,段階的な圧縮により乾燥密度が変化するため,この図にはインピーダンスを計測した
時点でのデータがプロットされている.
ダイヤルゲージを設置した時点を初期状態として,
そこから0.5mmずつピストンを押し込み,供試体を圧縮している.
そのため,初期状態での乾燥密度が最も低く,圧縮終了時点が最も高い値を示している.
%供試体w20, w22およびw25では,圧縮の最終段階ではほぼ飽和状態にあることが分かる.
なお,乾燥密度が供試体によって異なる理由は,
ピストンの押し込量はダイヤルゲージの読みから正確に制御できるが,
ピストンの初期位置には$\pm$0.1mm程度の誤差が生じていたことによる.
実際,乾燥密度の圧縮開始時と終了時の差はいずれの供試体でも約0.3g/cm$^3$でほぼ一致している.
供試体の実際の含水比が試料混合時の設計値より一貫して低い理由は,
混合とモールドへの封入までの間に乾燥が進むためと考えられる.
なお,モールドで圧縮している間に乾燥による質量変化がほとんど無いことは,別途作成した
参照試験片の質量を経時的に記録した結果によって確認している.
具体的には,粘土供試体をモールドに入れてピストンを取り付けた状態に
しておけば,計測期間中に質量変化はおよそ$\pm$0.01gに収まることを確認している.
%--------------------
\begin{figure}[h]
	\begin{center}
	\includegraphics[width=0.7\linewidth]{Figs/fig4_4.eps} 
	\end{center}
	\caption{
		インピーダンス測定時の供試体含水比と乾燥密度.
		破線は飽和度が一定の曲線を示す.
		} 
	\label{fig:fig4_4}
\end{figure}
%--------------------
\subsection{インピーダンススペクトル}
図\ref{fig:fig4_5}-図\ref{fig:fig4_9}に,計測で得られたインピーダンススペクトル
をナイキスト線図として示す.これらはw15からw24までの供試体に対する結果を
順に示したもので,各々の図には各圧縮段階における計測結果を全て示している.
これらの図でインピーダンススペクトルは,圧縮の進展に伴い概ね実数軸上を左方向に移動する.
ナイキスト線図の縦軸横軸の比率は等しく,グラフの傾きは正確に描画されている.
また,縦軸の表示範囲は全ての図で共通だが,含水比の低下にともないインピーダンスの
抵抗成分の値が大きくなるため,横軸の表示範囲はそれぞれの図で異なったものとなっている.
次章では,これらの結果の解釈について議論を行う.
%--------------------
\begin{figure}[h]
	\begin{center}
	\includegraphics[width=0.7\linewidth]{Figs/fig4_5.eps} 
	\end{center}
	\caption{
		供試体w15に対して計測されたインピーダンススペクトルのナイキスト線図.
	} 
	\label{fig:fig4_5}
\end{figure}
%--------------------
%--------------------
\begin{figure}[h]
	\begin{center}
	\includegraphics[width=0.7\linewidth]{Figs/fig4_6.eps} 
	\end{center}
	\caption{
		供試体w17に対して計測されたインピーダンススペクトルのナイキスト線図.
	} 
	\label{fig:fig4_6}
\end{figure}
%--------------------
%--------------------
\begin{figure}[h]
	\begin{center}
	\includegraphics[width=0.7\linewidth]{Figs/fig4_7.eps} 
	\end{center}
	\caption{
		供試体w20に対して計測されたインピーダンススペクトルのナイキスト線図.
	} 
	\label{fig:fig4_7}
\end{figure}
%--------------------
%--------------------
\begin{figure}[h]
	\begin{center}
	\includegraphics[width=0.7\linewidth]{Figs/fig4_8.eps} 
	\end{center}
	\caption{
		供試体w22に対して計測されたインピーダンススペクトルのナイキスト線図.
	} 
	\label{fig:fig4_8}
\end{figure}
%--------------------
%--------------------
\begin{figure}[h]
	\begin{center}
	\includegraphics[width=0.7\linewidth]{Figs/fig4_9.eps} 
	\end{center}
	\caption{
		供試体w24に対して計測されたインピーダンススペクトルのナイキスト線図.
	} 
	\label{fig:fig4_9}
\end{figure}
%--------------------
