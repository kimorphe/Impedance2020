本研究では圧縮成形によって作成した,ペレット状の不飽和粘土を実験供試体に
用いて電気化学インピーダンスの測定を行う.
本節では,供試体の作成方法についてはじめに述べ,次に,電気化学インピーダンス
の測定方法を示す.最後に,実験で得られたインピーダンススペクトルをまとめて
示す.これをもとに,計測結果の解釈と議論を次の章において行う.
\section{供試体の作成}
供試体の材料には,クニミネ工業株式会社製のクニピアFを用いる.
クニピアFは,不純物を取り除いたほぼ純粋なナトリウム型モンモリロナイトで,
乾燥して粉末状にしたものとして提供されている.クニピアFの
主たる物理,化学的な特性は表\ref{tbl:kunipia}のようである.
\begin{table}[h]
\begin{center}
\caption{クニピアFの主な物理,化学特性.}
\begin{tabular}{|c||c|}
	\hline
	モンモリロナイト含有量 & 98.5\% 以上\\
	\hline
	真比重 & 2.6 \\
	\hline
	かさ密度 [g/cm$^3$]& 0.3$\sim$ 0.4 \\
	\hline 
	液性限界[\%] & 993 \\
	\hline 
	塑性限界[\%] & 42 \\
	\hline 
	塑性指数[\%] & 951 \\
	\hline 
	陽イオン交換量 [meq/100g] &  117\\
	\hline 
	浸出陽イオン [meq/100g] &  \\
	\hline 
	Na$^+$ & 114.9\\
	\hline 
	K$^+$ & 1.1\\
	\hline 
	Ca$^{2+}$ & 20.6\\
	\hline 
	Mg$^{2+}$ & 2.6 \\
	\hline 
\end{tabular}
\label{tbl:kunipia}
\end{center}
\end{table}
供試体は,クニピアFと純水を所定の含水比となる比率で混合して
モールドに封入し,油圧プレスで圧縮してペレット状に成形する.
後述するように,圧縮成形に用いるモールドとピストンは電気化学インピーダンス
計測のためのセルを兼ね,インピーダンス計測は油圧プレスによる圧縮荷重を除荷せず,
供試体の厚みを一定に保った状態で行う.
材料の混合と圧縮成形までの手順は以下のようにして行った.
\begin{enumerate}
\item
	粉末状の粘土(クニピアF)を恒温乾燥炉に入れ,110$^\circ$Cで24時間乾燥させて絶乾状態にする.
\item
	乾燥した粉末粘土($m_s=$10[g])に所定の含水比$w$とするために必要な純水$(m_w=w m_s)$[g]を加え十分に混合する.
	混合には小型粉砕機を用い,純水はシリンジを用いて少しずつ投入してその都度十分に撹拌することで,
	水分が粉末粘土と均等に混ざり合うようにする.
\item
	純水で混練した粘土粉末を$m_s+m_w$[g]正確に計量してモールドに入れ,ピストンを油圧プレスで
	押し込み所定の厚さ$h$まで圧縮する.
\item
	ピストンの押し込み量を保持し,供試体厚さ$h$を一定に保ったままの状態で電気化学インピーダンスの測定を行う.
\item
	インピーダンス計測の終了後,供試体をモールドから取り出し,直径$D$,厚さ$h$,湿潤質量$m_{wet}$
	を計測する.
\item
	供試体を恒温乾燥炉により24時間,110$^\circ$Cで乾燥し,供試体の乾燥質量$m_{dry}$を計測する.
\end{enumerate}
以上の手順で作成した供試体の外観を図\ref{fig:fig4_1}に示す.
%--------------------
\begin{figure}[h]
	\begin{center}
	\includegraphics[width=0.6\linewidth]{Figs/fig4_1.eps} 
	\end{center}
	\caption{
		粘土供試体の外観.含水比が(a) 17\%程度と(b)22\%程度の場合.
		含水比が低い場合の方が明るい白色を呈する.
	} 
	\label{fig:fig4_1}
\end{figure}
%--------------------
なお,供試体の正確な含水比や間隙率等の諸量は,以下のようにして算出する.\\

はじめに、供試体に含まれる水分量$m_w$を
\begin{equation}
	m_w=m_{wet}-m_{dry}
	\label{eqn:}
\end{equation}
から求め,供試体の正確な含水比
\begin{equation}
	w=\frac{m_{wet}}{m_{dry}}
	\label{eqn:water_content}
\end{equation}
を得る.次に,湿潤密度$\rho_{wet}$と乾燥密度$\rho_{dry}$を供試体体積
\begin{equation}
	V=\frac{\pi D^2h}{4}
	\label{eqn:}
\end{equation}
を使って,
\begin{equation}
	\rho_{wet}=\frac{m_{wet}}{V}, \ \ 
	\rho_{dry}=\frac{m_{dry}}{V}
	\label{eqn:}
\end{equation}
で,それぞれ求める.間隙比$e$と間隙率は以上の結果から
\begin{equation}
	e=\frac{\rho_{wet}}{\rho_{dry}} -1
	\label{eqn:}
\end{equation}
\begin{equation}
	n=\frac{e}{1+e}
	\label{eqn:}
\end{equation}
によって与えられ,飽和度$S_r$も上で求めた量を使って
\begin{equation}
	S_r=\frac{w\rho_{clay}}{e\rho_{water}}
	\label{eqn:}
\end{equation}
と決定できる.ただし,$\rho_{water}$と$\rho_{clay}$は水と
モンモリロナイトの質量密度で,各々以下の通りである.
\begin{equation}
	\rho_{water}=1.0, \ \ \rho_{clay}=2.7 {\rm [g/cm^3]}
	\label{eqn:}
\end{equation}
以上のような方法で,目標含水比を
\[
	w=15,\, 17.5,\, 20,\, 22,\, 24\%
\]
の供試体5体を作成したところ,供試体の直径は29.3[mm],
厚さは9[mm], 質量12[g]程度となった.
以下では,これらの供試体を"w15", "w17"供試体等と称する.
\section{インピーダンス測定方法}
電気化学反応による電流と電圧の関係は一般に非線形だが,印加交流電圧を十分に小さくすれば両者は線形な
関係とみなせ,応答関数であるインピーダンススペクトルによって電流-電圧の関係を完全に記述できる.
ただし,ここでは物質の輸送に関する電気化学的な特性が興味の対象であるため、全周波数範囲で
インピーダンス計測を行う必要はない.物質輸送が関与する電気化学的な現象は緩和が遅いため、
その特性はインピーダンススペクトルの低周波側に現れる.そこで,100kHzまでの
インピーダンススペクトルを印加する交流電圧の周波数を掃引して計測する.
図\ref{fig:fig4_2}にインピーダンス計測実験の装置構成を示す.
%--------------------
\begin{figure}[h]
	\begin{center}
	\includegraphics[width=0.8\linewidth]{Figs/fig4_2.eps} 
	\end{center}
	\caption{
		インピーダンス測定のための実験装置の構成.
		(a)計測セル(圧縮成形のためのモールドを兼ねる), (b)粘土供試体,(c)油圧プレス, 
		(d)ケミカルインピーダンスアナライザと(e)制御およびデータ記録用PC.
	} 
	\label{fig:fig4_2}
\end{figure}
%--------------------
実験装置は、油圧プレス、計測セルとインピーダンスアナライザおよび制御PCで構成されている。
油圧プレスは,供試体を所定の厚さまで圧縮成形するためのものである。
ここで用いた油圧プレスは最大10tまでの荷重を加えることができるが、
本研究で用いる粘土供試体は小型のものであるため、実際には300〜500[kgf]程度の荷重を
加えるために利用する。供試体は圧縮成形のためのモールドに入れた状態で
インピーダンスを計測するため、計測セルを兼ねる。
そのため、圧縮を行う際に押し込むピストンをSUS316Lのステンレスネジ(M20)を
旋盤加工して電極としても利用できるようにしてある.
ピストンすなわち電極の直径は29.3[mm]で、これはアクリルパイプで作成した
モールドの直径に一致させている。
インピーダンスの計測は、同一の供試体を段階的に圧縮し、その過程で異なる厚さに
おけるインピーダンススペクトルを取得する。
具体的には、供試体の厚さを最初に$h=10.5$[mm]程度まで圧縮し、
0.5mmずつ追加で圧縮を行い,最終的には9.0[mm]程度まで供試体を圧縮する。
このようにすることで、含水比が一定で、乾燥密度の異なる場合にインピーダンスに生じる変化を調べることができる。
各圧縮段階における試料の厚さは、ダイヤルゲージでピストンの押し込み量を計測しておき、
最終的に供試体をセルから取り出したときの厚みと押し込み量からインピーダンス計測時の厚みを逆算する。
インピーダンススペクトルの測定は、市販のケミカルインピーダンスアナライザ(HIOKI,IM3590)を用いる。
なお、低周波数域において計測を行う場合,試料と電極の間に生じる分極が試料のインピーダンスに影響することが指摘されている。
そのため測定器と測定プローブは,測定電流により発生する磁界の影響を低減できる4端子法構造を採用した
アナライザーを選択した。
アナライザーはノートPCから制御し、計測条件の設定と計測データの取得と保存を行う。
%--------------------
\begin{figure}[h]
	\begin{center}
	\includegraphics[width=0.6\linewidth]{Figs/fig4_3.eps} 
	\end{center}
	\caption{
		四端子法による計測のための測定端子構成.
	} 
	\label{fig:fig4_3}
\end{figure}
%--------------------
以上の計測を行うための手順は次のようである。
\begin{enumerate}
\item
	所定の含水比に調整した含水粘土粉末を納めたモールドにピストン兼電圧をとりつけ,
	油圧プレスに設置する。
\item
	油圧プレスでピストンを押しこみ供試体を圧縮成形する。
	このとき,供試体の厚さはおよそ10.5mmとなるようにプレス量を調整する.
\item
	ダイヤルゲージを取り付け,このときのゲージの読みを初期値として記録する。
\item
	ピストン/電極とインピーダンスアナライザを接続する.
	印加電圧を5[V]に設定し、周波数0.1[Hz]から100[kHz]の周波数範囲の
	インピーダンスを計測する。
	計測する周波数は、この範囲で対数軸上で等間隔に251点とる.
\item
	インピーダンス計測時の、ダイヤルゲージの読みと室温、湿度、参照試験体の質量を記録する。
\item
	ダイヤルゲージの値に変化が無くなった後、インピーダンス計測を3回、一定の時間間隔を空けて
	行う.実験の結果、1時間程度でゲージの読みに変化がなくなることが分かったため、
	圧縮の段階から1時間経過後、30分おきにインピーダンスの計測を行った。
\item
	ピストン/電極を0.5mm押し込み、インピダンス計測を同じ要領で行う。
 	これを所定の供試体の厚さとなるまで繰り返す。
\item
	油圧プレスの荷重を解法し、ピストン/電極のリバウンド量をダイヤルゲージで測定する.
\item
	無応力状態での供試体厚さを計測し、リバウンド量も用いて各圧縮段階における
	供試体の実際の厚さを求める.
\item
	供試体の直径、厚さ、質量を記録し、供試体を炉乾燥する。
\item
	完全に乾燥した試料の質量を測り、乾燥密度や間隙比等の量を求める。		
\end{enumerate}\section{実験結果}
%--------------------
\begin{figure}[h]
	\begin{center}
	\includegraphics[width=0.7\linewidth]{Figs/fig4_4.eps} 
	\end{center}
	\caption{
	} 
	\label{fig:fig4_4}
\end{figure}
%--------------------
