\chapter{考察}
\section{等価回路の推定}
図\ref{fig:fig5_1}に本研究で得られた典型的なナイキスト線図を示す.
黒の実線で示した計測結果には,低周波側に大きな容量性の半円(i)のおよそ半分が現れている.
これに加え,高周波側にも容量性半円の一部と思われる箇所がわずかに見られるため,
図には青の破線で想定される半円を描き込んである.
実際,100KHZ以上の周波数範囲まで計測を行うと,この部分にもはっきりとした半円が現れることを
別途実験を行い確認している.ただしここでは,物質移動に関係する長い緩和時間を持つ現象に興味があるため,
低周波側の半円(i)の挙動を表す等価回路を推定する.\\

円弧状の半円を示す等価回路の一つはRC並列回路である.
ただしRC並列回路のナイキスト線図は完全な半円を描く.
一方,図\ref{fig:fig5_1}を始めとする本研究での計測結果は,
横に扁平な半円となっているためRC並列回路でのフィッティグはできない.
そのため,Cole-ColeプロットやCole-Davidsonプロットの使用がより適切と考えられる.
いずれのプロットを選択するかにあたり半円(i)の高周波側での挙動を見ると,
図\ref{fig:fig5_1}に緑の実線で示した箇所で直線的な変化をする箇所があることに気付く.
この部分の実数軸に対する傾き$\alpha$はおよそ60$\sim$70度になっている.
直線的なインピーダンススペクトルはCPE素子で表現することができるため,
直線部分の存在はCPE素子を含む等価回路を想定すべきであることを示唆する.
Cole-ColeプロットはCPEと抵抗の並列接続で与えられることからこの条件を満足する.
一方,Cole-Davidsonプロットは,Cole-Coleプロットと類似したナイキスト線図を与えるものの,
有限個のCPE素子では表現することができない.
Cole-Davidonプロットの特徴は半円の左側半分(高周波側)が右側に比べてより歪んだ形となることである.
しかしながら,Cole-Coleプロットとの差が明らかになるのは高周波側での傾きが図\ref{fig:fig5_1}
の$\alpha$よりもかなり小さくなってからである.
以上のことから,ここではCole-Coleプロットを用いて実験で得られたインピーダンス
スペクトルのフィッティングを行う.
なお,ナイキスト線図上の直線的な変化はワールブルグインピーダンスにも現れることを
第3章において既に述べた通りで,等価回路の候補となるように思われる.
しかしながら,ワールブルグインピーダンスの直線部分の傾きは45$^\circ$のため,
ワールブルグインピーダンスだけでは今回の実験結果を再現することはできない.

Cole-Coleプロットのインピーダンスは第3章で述べたように式(\ref{eqn:CC})で与えられる.
これに,抵抗を直列に接続すれば,実数軸上の任意の位置にある半円状のスペクトルを
フィッティングすることができる.従って,ここで用いる等価回路のインピーダンスは
\begin{equation}
	Z=R_0 +\frac{R_{ct}}{1+\left( i\omega R_{ct}C\right)^p}
	\label{eqn:CC_R0}
\end{equation}
と表すことができる.式(\ref{eqn:CC_R0})において$R_0$は直列接続された抵抗成分を,
$R_{ct}$と$C$はRC並列回路の抵抗とコンデンサ成分の静電容量をそれぞれ表す.
式(\ref{eqn:CC_R0})は,CPEのインピーダンス(\ref{eqn:Z_CPE})を用い,次のように
書き直すことができる.
\begin{equation}
	Z=R_0 +\frac{R_{ct}}{1+\left( i\omega \right)^pT_{CPE}}
	\label{eqn:R_CPE}
\end{equation}
ただし,
\begin{equation}
	T_{CPE}=R_{ct}^{p-1}C^{p}
	\label{eqn:T_CPE}
\end{equation}
である.式(\ref{eqn:R_CPE})は図\ref{fig:fig5_2}の回路の合成インピーダンス
であることから,この図に示した抵抗aと抵抗(b)-CPE(c)の直-並列回路が
本研究で用いる等価回路となる.なお,$T_{CPE}$はCPE素子の時定数で,その次元は
\begin{equation}
	[T_{CPE}] =[ {\rm F\cdot s}^{p-1}]
	\label{eqn:}
\end{equation}
で,指数$p$が1のときコンデンサーの静電容量と単位も含めて一致する.
これら回路素子に対応する電気化学的な現象は,$R_0$は試料内の
電荷移動抵抗,$R_{ct}$と$T_{CPE}$はそれぞれ電極や物質の界面における
電荷移動抵抗と電気二重層の形成による分極であると言われている.
%--------------------
\begin{figure}[h]
	\begin{center}
	\includegraphics[width=0.8\linewidth]{Figs/fig5_1.eps} 
	\end{center}
	\caption{
		計測で得られたインピーダンススペクトルの特徴.	
	} 
	\label{fig:fig5_1}
\end{figure}
%--------------------
\begin{figure}[h]
	\begin{center}
	\includegraphics[width=0.4\linewidth]{Figs/fig5_2.eps} 
	\end{center}
	\caption{
		インピーダンススペクトルのフィッティングに用いる等価回路.
		$R_0,R_{ct}$および$T_{CPE}$はそれぞれの回路素子の素子定数を表す.
	} 
	\label{fig:fig5_2}
\end{figure}
%--------------------
\section{回路定数の推定方法}
図\ref{fig:fig5_2}に示した等価回路の素子定数$R_0,R_{ct},C$および$p$を
最小二乗法は用いて決定する.そこで,実験で得られたインピーダンスを$Z(\omega)$,
回帰式として用いる式(\ref{eqn:R_CPE})のインピーダンスを$\tilde{Z}(\omega)$
と書き,残差$r$を次のように定義する.
\begin{equation}
	r\left(R_0,R_{ct},C,p \right):= \frac{1}{2}\int _{W} \left| Z(\omega)-\tilde{Z}(\omega)\right|^2 d\omega
	\label{eqn:}
\end{equation}
残差$r$を最小化することで素子定数を決定する.すなわち
\begin{equation}
	\left( R_0, R_{ct}, C,p \right)= {\rm argmin} \left\{ r\left(R_0,R_{ct},C,p \right) \right\}
	\label{eqn:LSprb}
\end{equation}
で$R_0,R_{ct},C$および$p$を決定する.ここで,回帰式$\tilde Z(\omega)$は,
\begin{equation}
	\tilde Z (\omega) =R_0+\frac{R_{ct}}{g(C,p)}
	\label{eqn:}
\end{equation}
の形をしていることから,$R_0$と$R_{ct}$について線形である. 一方,$g(C,p)$の項は
\begin{equation}
	g(C,p)= 1+(i\omega)^pT_{CPE} 
	\label{eqn:}
\end{equation}
だから,$p$と$C$あるいは$T_{CPE}$について$\tilde {Z}(\omega)$は
非線形な複素数値関数となる.
以上より$(p,C)$が与えられたときであれば,$r$を最小化する$R_0$と$R_{ct}$を厳密に求めることができる.
この点を利用し本研究では,次のような繰り返し計算により,式(\ref{eqn:LSprb})の
最小二乗問題の近似解を求めた.
\begin{enumerate}
\item
	$(p,C)$の初期値を設定する.
\item
	与えられた$(p,C)$に対して$r$を最小化する$(R_0, R_{ct})$を求める.
\item
	ステップ(2)で求めた$(R_0,R_{ct})$と$C$に対し,$r$を最小化する$p$を探索して
	$p$の値を更新する.
\item
	ステップ(2)で求めた$(R_0,R_{ct})$とステップ(3)で更新した$p$に対し,$r$を最小化する$T_{CPE}$を探索して
	$T_{CPE}$の値を更新する.
\item
	現在の$(R_0,R_{ct},p,C)$に対する残差$r$が十分小さくなるか,今以上に減少しなくなるまでステップ(2)から(4)の更新を繰り返す.
\end{enumerate}
なお,ステップ(3)と(4)の探索には1次元最小化のどのようなアルゴリズムを用いてもよい.
ここでは最も単純な方法として,指定された区間内部を一定の間隔で
探索することで最小値の近似値を求めた.
ただし,探索区間は(4)のステップが終了するたびに縮小し,探索の解像度が次第に高くなるようにしている.
探索区間の初期値は$p$については$[0.4,1.0]$として区間内を50分割し,
$T_{CPE}$については$ [10^{-6}, 10^{6}]$において
対数軸上で等間隔に区間分割して最小値を求めた.
%いずれも,反復計算1回毎に区間を0.95倍に縮小した.
\section{回路定数の推定結果と考察}
実験結果から推定した等価回路の素子定数を,図\ref{fig:fig5_3}−図\ref{fig:fig5_7}に示す.
各々の図に示した2つのグラフは,(a)横軸を乾燥密度,(b)横軸を含水比とした
ものの二通りの形式で同じ量の変化を示している.
例えば,図\ref{fig:fig5_3}では,直列接続された抵抗$R_0$を
単位面積あたりの抵抗として(a)では乾燥密度との関係を,
(b)では含水比との関係を示している.
なお,同一色のドットは同じ試験体に対する結果を示している.
一つの試験体に対し,各圧縮段階で3回以上のインピーダンス計測を
行っているため,それぞれの供試体に対し15から20点程度の推定値が示されている.
以下,これらの結果について順に検討する.\\

はじめに,図\ref{fig:fig5_1}の$R_0$についてみると,直列抵抗成分は含水比の増加につれ
抵抗値を下げることが分かる.これは,水分中の荷電物質が水分量の増加につれて移動し易くなる
ことを示すものである.なお,供試体作成には純水を用い,粘土の含有量も一定であるため,
供試体ごとに電荷を持つ物質の総量は同じであることから,この結果は固体粘土の物性としての
電気伝導性を示していると考えて良い.乾燥密度との関係について言えば,含水比の低いw15やw17の
供試体では,乾燥密度が増加するについれて抵抗は下がっている.一方,含水比が相対的に高い
w20,w22,w24の場合,乾燥密度の増加による抵抗の減少幅は小さい.
このことは,電気伝導経路となる水分のネットワークが低含水比の供試体では圧縮によって
連結性が増すが,高含水比のものは当初から水分の連結性がよく,密度の増加が伝導率の増加に
大きく寄与しないことを示している.\\

%--------------------
\begin{figure}[h]
	\begin{center}
	\includegraphics[width=0.8\linewidth]{Figs/fig5_3.eps} 
	\end{center}
	\caption{
		等価回路に含まれる直列抵抗成分$R_0$.
	} 
	\label{fig:fig5_3}
\end{figure}
%--------------------
図\ref{fig:fig5_4}は,CPEと並列接続された抵抗$R_{ct}$の推定結果を示したものである.
$R_{ct}$は乾燥密度や含水比との明瞭な相関がみられない.
特徴的な点を強いてあげるとすれば,水分量の少ないw15供試体で他よりも大きな値を
取るという点がある.なお,値の大小はあるものの水分量にも密度にも$R_{ct}$
はっきりした相関を示さないということは,物質界面における電荷移動抵抗は
水分量や密度に影響を受けにくいという解釈が成り立つ.
\begin{figure}[h]
	\begin{center}
	\includegraphics[width=0.8\linewidth]{Figs/fig5_4.eps} 
	\end{center}
	\caption{
		等価回路に含まれる並列抵抗成分$R_{ct}$.
	} 
	\label{fig:fig5_4}
\end{figure}
これに対して,$R_{ct}$と対をなす量としてインピーダンスに現れる容量成分$C$の
推定結果は図\ref{fig:fig5_5}のようでなる.
こちらも水分,密度との相関は明瞭では無いが,相対的に水分の多いw22とw25供試体では
乾燥密度に対して明らかに増加する傾向がある.
これは,容量成分の発現には水分で満たされた間隙のサイズが関係することを示唆している.
ただし,$C$は$p=1$の場合を除き単一のコンデンサを表現するものでなく,
RC並列回路からのずれが大きいとき,$C$自体に回路素子としての意味をもたせることや,
具体的な電気化学的プロセスに対応付けることは難しい.\\
%--------------------
\begin{figure}[h]
	\begin{center}
	\includegraphics[width=0.8\linewidth]{Figs/fig5_5.eps} 
	\end{center}
	\caption{
		Cole-Coleプロットにおけるコンデンサの静電容量$C$.
	} 
	\label{fig:fig5_5}
\end{figure}

図\ref{fig:fig5_7}に$T_{CPE}$の推定結果を示す.
この図の(a)は横軸を乾燥密度に,(b)は含水比にとって$T_{CPE}$を
プロットしたものである.
この図から明らかなように,同じ含水比であれば乾燥密度が高い程,
$T_{CPE}$も大きな値となる.また,同じ乾燥密度でみたときには,
含水比が高い程$T_{CPE}$の値も大きい.
$T_{CPE}$はコンデンサの静電容量とは異なる次元を持つ量であるが,
蓄電量と電圧の関係を表す実数値の係数であることから,
蓄電容量を表す定数という点は同じである.
従って,図\ref{fig:fig5_7}の結果は,水分が多い程,また,
乾燥密度が大きい程,蓄電容量が増すことを意味している.
蓄電は粘土内の局所的な分極(正負電荷の分離)が生じて電気二重層を形成することによる.
このように考えると,水分の増加に伴い$T_{CPE}$の値が大きくなることは,
局所的な分極が間隙水の内部あるいは表面で発生していると判断できる.
一方,乾燥密度の増加による$T_{CPE}$の増加は,局所的な分極が
より狭い領域内に密集して生じるためと解釈できる.
同じことは,乾燥密度の増加に伴い,分極電荷の空間的な密度も高まり,蓄電容量の増加となって
現れたという言い方もできる.\\
\begin{figure}[h]
	\begin{center}
	\includegraphics[width=0.8\linewidth]{Figs/fig5_7.eps} 
	\end{center}
	\caption{
		等価回路に含まれるCPE素子の時定数$T_{CPE}$.
	} 
	\label{fig:fig5_7}
\end{figure}

最後に,CPEの指数$p$と含水比,乾燥密度との関係を図\ref{fig:fig5_6}に示す.
この図に示した含水比と$p$の関係(b)を見ると,水分の増加に対し$p$の変化は
単調でなく,含水比20\%程度で極大となっている.
次に,乾燥密度との関係(a)を含水比毎に見ると,乾燥密度に対して増加
するケースと低下するケースが混在している.
含水比が20\%以下のw15,w17およびw20では,乾燥密度の増加に応じて$p$も増加するのに対し,
高含水比側(w22とw24)では,これと逆の傾向を示している.
この結果は,含水比20\%前後では,指数$p$を決める要因となる
電気化学的現象において特別な状態が発生していることを示唆する.
そこで,指数$p$の意味についてより詳しく検討する.
CPEの電流と電圧の関係は
\begin{equation}
	V=\frac{I}{(i\omega )^pT_{CPE}}
	\label{eqn:}
\end{equation}
である.帯電量$Q$は電流を時間に関して一回積分することで得られるため,交流回路では
\begin{equation}
	Q=\frac{I}{i \omega}=(i\omega)^{p-1}T_{CPE}V=T_{CPE}Ve^{-i\frac{\pi}{2}(1-p)}
	\label{eqn:}
\end{equation}
となる.これは,帯電量$Q$は印加電圧に対して位相が
$\phi_p=\frac{\pi}{2}(1-p)$だけ遅れることを意味する.
$p=1$の場合は理想的なコンデンサを意味し,このとき$\phi_p=0$で
帯電は電圧と完全に同期する.一方,$p<1$の場合,印加した電圧よりも帯電が遅れて生じ,
そのラグは指数$p$が小さい程大きくなる.
なお,分極を生じさせる電荷移動が拡散に支配されるときは$p=0.5$となり,
このときのCPEはワールブルグインピーダンスと呼ばれている.
つまり,$0.5<p<1$では,拡散支配のもとで生じる場合分極程に遅くは無いが,
電圧変化に追従できるほど速やかなものではないことを意味する.
このことを踏まえれば,図\ref{fig:fig5_6}において$p$の下限値が概ね0.5程度で拡散
に支配され,上限値でも0.75程度で明らかな位相遅れを伴うということを示しており,
このような$p$の推定結果は合理的と考えられる.
なお,低含水比の場合に,乾燥密度に対して$p$が増加することは,
乾燥密度の上昇に伴い分極が生じやすくなると言い換えることができる.
分極が生じるためには電荷の移動が必要で,電荷の移動は間隙水を経路として起きる.
そのため,間隙水量と,間隙水ネットワークの連結性や屈曲は,分極の位相遅れ,
すなわち指数$p$の値に反映される.以上のことを踏まえれば,低含水比側での乾燥密度に
対する$p$の増加は,乾燥密度が増すことによって間隙水の連結性がよくなることを示すと解釈できる.
一方,高含水比側での乾燥密度に対する$p$の減少は,間隙水の連結性が向上する効果を,
圧縮によって生じる間隙水ネットワークの屈曲や閉塞の効果が上回り,
結果的には電荷移動が抑制されるためと考えることができる.
図\ref{fig:fig5_6}はこれら相反する効果が,含水比20\%程度でバランスすることで,
乾燥密度に対する$p$の変化挙動が切り替わったことを示している.
%--------------------
\begin{figure}[h]
	\begin{center}
	\includegraphics[width=0.8\linewidth]{Figs/fig5_6.eps} 
	\end{center}
	\caption{
		等価回路に含まれるCPE素子の指数$p$.
	} 
	\label{fig:fig5_6}
\end{figure}
%図\ref{fig:fig5_6}はCPEの指数$p$の推定結果を示したもので,最も興味深い挙動が現れている.
%まず,含水比との関係を見ると,水分の増加に対し$p$の応答は単調でなく,w20で極大となる結果を示している.
%含水比が同じ(すなわち同じ供試体)で指数$p$には変動が有り,この点について乾燥密度の関係をみれば,w15,w17,w20の
%低含水比側では乾燥密度の増加に応じて$p$も増加するが,高含水比ではその反対の傾向となることが示されている.
%この結果は,含水比20\%前後が,指数$p$を決める要因の特別な状態が現れていることを意味する.
%指数$p$は,理想的コンデンサーからのずれを意味し,$p=1$に近いほど理想的なコンデンサーとみなしうる.
%逆に,$p$が小さい場合は,単一のコンデンサーで挙動を説明できないことを意味する.特に$p=0.5$の場合はワールブルグインピーダンスの場合に
%相当し,荷電物質の拡散が微視的な電気二重層の形成を律速する.つまり,$p$が大きいときには一定の外部電場に対して充電が速やかかつ
%効率的に行われるが,$p$が小さい場合には,電荷の移動が拡散に支配される結果,理想的なコンデンサーのようには充電
%すなわち分極が怒らない.分極を妨げる要素には,水分が少なく伝導経路が十分に確保されないことと,
%間隙が複雑に屈曲して電場へ追従した電荷の移動が制限を受けることに2つが考えられる.
%w22やw25の指数が小さいことは,間隙の屈曲によるものと考えられる.このように考えると,乾燥密度が大きくなっても,屈曲率が増すために,
%分極率には寄与しないことも説明がつく.一方,w15やw17では水分が少なくもともと水分ネットワークの繋がりが
%良くないことが$p$が小さくなることが一つの理由と考えられる.そのため,乾燥密度が増すと,ネットワークの連結性は向上し,
%指数$p$は若干大きな値を持つようになる.しかしながら,間隙の屈曲は次第に大きくなるため,指数$p$の増加は
%頭打ちになり,水分が多い場合ほどの値にはならない.これに対して,w20は適切な水分量で,無理なく水和粘土が
%締め固められたために,元々間隙の屈曲が小さく圧縮されることによって荷電物質の移動する一定の直線距離を移動するための経路長が
%より短くなることで,分極が起こりやすくなると考えられる.これは巨視的には最適含水比に近い状態で締め固められたことによると言うこともできる.
%--------------------
%最後に,CPEの時定数$T_{CPE}$の推定結果を図\ref{fig:fig5_9}に示す.この結果は,水分,密度と最も相関の高い挙動を示す.
%乾燥密度,含水比のいずれに対しても$T_{CPE}$はほぼ単調に増加している.すなわち,乾燥密度が同じであれば,水分が多い方が,
%水分量が同じならば乾燥密度が高い方が$T_{CPE}$の値が大きくなっている.$T_{CPE}$は一般化されたコンデンサーの容量を表す.
%従って,水分と密度に応じて,コンデンサー成分が支配的になることを意味する・また,微視的なコンデンサーの総量が,密度や水分といったマクロ量で
%スケールし,微視的な構造に依らないことを示すという点で非常に興味深い結果と言える.
%なお,$p$もCPEの性質を決めるパラメータだが,$T_{CPE}$は誘電体の分極率にあたり,$p$は分極の緩和時間に関する
%パラメータであるため,両者が水分量と密度に対して同じように変化する必然性は無い.ここでの結果は,$T_{CPE}$は微視構造をあまり反映せず,
%一方$p$は間隙ネットワークの屈曲率に影響されることを示唆するという点でで微視的構造を反映するパラメータということができる.

