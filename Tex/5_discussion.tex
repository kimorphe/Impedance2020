\section{等価回路の推定}
図\ref{fig:fig5_1}に,本研究で得られた典型的なナイキスト線図を示す。
黒の実線で示した計測結果には,低周波側に大きな容量性の半円(i)のおよそ半分が現れている.
高周波側にも容量性半円の一部と思われる箇所がわずかに見られ,図ではこれに青の破線で
想定される半円を描き込んである。実際、100KHZ以上の周波数範囲まで計測を行うと,
この部分にもはっきりとした半円が現れることは,別途実験を行い確認している.
ここでは,物質移動に関係する長い緩和時間を持つ現象に興味があるため,
低周波側の半円(i)について調べる.
円弧状の半円を示す等価回路の一つはRC並列回路である.
ただし,RC並列回路は完全な半円でナイキスト線図を描く.
一方,ここでの計測結果では、縦方向に扁平な半円となっているため,
RC並列回路でのフィッティグはできず,Cole-ColeプロットやCole-Davidsonプロット
の使用がより適切と考えられる。
いずれのプロットを選択するかにあたり,半円(i)の高周波側での挙動を見ると
図\ref{fig:fig5_1}に緑の実線で示した箇所で直線的に変化をしている箇所があり,
この部分の実数軸に対する傾き$\alpha$は,およそ60$\sim$70度になっている.
直線的なインピーダンススペクトルはCPE素子で表現することができるため,
このことはCPE素子を含む等価回路を想定すべきであることを示唆している.
Cole-ColeプロットはCPEと抵抗の並列接続で与えられることからこの条件を満足する.
一方,Cole-Davidsonプロットは,Cole-Coleプロットと類似したナイキスト線図を与えるが,
有限個のCPE素子では表現することができない.Cole-Davidonプロットは,半円の左側半分(高周波側)が
右側に比べてより歪んだ形を表すことが可能だが,Cole-Coleプロットとの差が明らかになるのは
高周波側での傾きが図\ref{fig:fig5_1}の$\alpha$よりもかなり小さくなってからである.
以上のことから,ここではCole-Coleプロットを用いて,実験で得られたインピーダンス
スペクトルのフィッティングを行う.
なお,ナイキスト線図上の直線的な変化はワールブルグインピーダンスにも現れる.
しかしながら,ワールブルグインピーダンスの直線部分の傾きは45$^\circ$のため,
ワールブルグインピーダンスだけでは,今回の実験結果を再現することはできない.
この点でも,Cole-Coleプロットの利用が適切であるということができる.

Cole-Coleプロットのインピーダンスは,第3章で述べたように式(\ref{eqn:CC})で与えられる.
これに,抵抗を直列に接続すれば,実数軸上の任意の位置になる半円状のスペクトルへの
フィッティングを行うことができる.従って,ここで用いるインピーダンスは
\begin{equation}
	Z=R_0 +\frac{R_{ct}}{1+\left( i\omega R_{ct}C\right)^p}
	\label{eqn:CC_R0}
\end{equation}
と表すことができる.
ここに,$R_0$は直列接続した抵抗成分の抵抗を,$R_{ct}$と$C$は
RC並列回路の抵抗とコンデンサ成分の静電容量をそれぞれ表す.
式(\ref{eq:CC_R0})は次のように書き直すことができる.
\begin{equation}
	Z=R_0 +\frac{R_{ct}}{1+\left( i\omega \right)^pT_{CPE}}
	\label{eqn:R_CPE}
\end{equation}
ただし,
\begin{equation}
	T_{CPE}=R_{ct}^{p-1}C^{p}
	\label{eqn:T_CPE}
\end{equation}
である.式(\ref{eqn:R_CPE})は図\ref{fig:fig5_2}の回路の合成インピーダンス
であることから,この図に示した抵抗aとCPE(b)-抵抗(c)の直-並列回路が
本研究で用いる等価回路となる.
なお,$T_{CPE}$はCPE素子の時定数で,その次元は
\begin{equation}
	[T_{CPE}] ={\rm [ F\cdot s^{p-1}]}
	\label{eqn:}
\end{equation}
で,指数$p$が1のとき,コンデンサーの静電容量と単位も含めて一致する.
これら回路素子に対応する電気化学的な現象は,$R_0$は試料内の
電荷移動抵抗,$R_{ct}$と$T_{CPE}$はそれぞれ電極や物質の界面における
電荷移動抵抗と電気二重層の形成による分極であると言われている.
%--------------------
\begin{figure}[h]
	\begin{center}
	\includegraphics[width=0.9\linewidth]{Figs/fig5_1.eps} 
	\end{center}
	\caption{
		計測で得られたインピーダンススペクトルの特徴.	
	} 
	\label{fig:fig5_1}
\end{figure}
%--------------------
\begin{figure}[h]
	\begin{center}
	\includegraphics[width=0.5\linewidth]{Figs/fig5_2.eps} 
	\end{center}
	\caption{
		インピーダンススペクトルのフィッティングに用いる等価回路.
		$R_0,R_{ct}$および$T_{CPE}$はそれぞれの回路素子の素子定数を表す.
	} 
	\label{fig:fig5_2}
\end{figure}
%--------------------
\section{回路定数の推定方法}
図\ref{fig:fig5_2}に示した等価回路の素子定数$R_0,R_{ct},C$および$p$を
最小二乗法を用いいて決定する.
ここで,実験で得られたインピーダンスを$Z(\omega)$,
回帰式として用いる式(\ref{eqn:R_CPE})のインピーダンスを$\tilde{Z}(\omega)$
と書き,残差$r$を次のように定義する.
\begin{equation}
	r\left(R_0,R_{ct},C,p \right):= \frac{1}{2}\int _{W} \left| Z(\omega)-\tilde{Z}(\omega)\right|^2 d\omega
	\label{eqn:}
\end{equation}
残差$r$を最小化することで,素子定数を決定する.すなわち
\begin{equation}
	\left( R_0, R_{ct}, C,p \right)= {\rm argmin} \left\{ r\left(R_0,R_{ct},C,p \right) \right\}
	\label{eqn:}
\end{equation}
とする.回帰式$\tilde Z(\omega)$は,
\begin{equation}
	\tilde Z (\omega) =R_0+\frac{R_{ct}}{g(C,p)}
	\label{eqn:}
\end{equation}
の形をしており,$R_0$と$R_{ct}$について線形である. 一方,
\begin{equation}
	g(C,p)= 1+(i\omega)^pT_{CPE} 
	\label{eqn:}
\end{equation}
だから,$p$と$C$あるいは$T_{CPE}$について$\tilde {Z}(\omega)$は
非線形な複素数値関数となっている.
従って,$(p,C)$が与えられたときには$r$を最小化する$R_0$と$R_{ct}$は厳密に求めることができる.
以上のことを踏まえ,本研究では,次に示すような繰り返し計算を行うことで,
非線形最小二乗問題の近似解を求めた.
\begin{enumerate}
\item
	$(p,C)$の初期値を設定する.
\item
	与えられた$(p,C)$において,$r$を最小化する$(R_0, R_{ct})$を求める.
\item
	ステップ(2)で求めた$(R_0,R_{ct})$と$C$に対し,$r$を最小化する$p$を探索して
	$p$の値を更新する.
\item
	ステップ(2)で求めた$(R_0,R_{ct})$とステップ(3)で更新した$p$に対し,$r$を最小化する$T_{CPE}$を探索して
	$T_{CPE}$の値を更新する.
\item
	現在の$(R_0,R_{ct},p,C)$に対する残差$r$が十分小さくなるか,今以上に減少しなくなるまでステプ(2)から(4)の更新を繰り返す.
\end{enumerate}
なお,ステップ(3)と(4)の探索には,1次元最小化のどのようなアルゴリズムを用いてもよい.
ここでは,最も単純な方法として,探索区間を定め,その区間内部を一定の間隔で
探索することで最小値の近似値を求めた.
ただし,探索区間は(4)のステップが終了するたびに縮小し,探索の解像度が次第に高くなるようにしている.
探索区間の初期値は$p$については$[0.4,1.0]$として区間内を50分割して
最小値を求めた.一方,$T_{CPE}$については$ [10^{-6}, 10^{6}]$において
対数軸上で等間隔に区間を分割して最小値を求めた.
%いずれも、反復計算1回毎に区間を0.95倍に縮小した.
\section{回路定数の推定結果と考察}
実験結果から推定した等価回路の素子定数を図\ref{fig:fig5_3}−図\ref{fig:fig5_7}に示す.
これら各々の図に示した2つのグラフは,横軸が乾燥密度のものと含水比として
同じ量の変化を示したものである.
例えば,図\ref{fig:fig5_3}では,直列接続された抵抗の抵抗値$R_0$を
単位面積あたりの抵抗として(a)では乾燥密度との関係を,
(b)では含水比との関係を示している.
なお,同一色のドットは同じ試験体に対する推定値を示している.
一つの試験体に対し,段階的に圧縮を行っているため,圧縮の進行に応じて
回路定数の推定値が変化することから,一つの試験体に対し15から20点程度の推定値が示されている.
以下,これらの結果について順に検討する.

はじめに,図\ref{fig:fig5_1}の$R_0$についてみると,直列抵抗成分は含水比の増加につれ
抵抗値を下げることが分かる.これは、水分中の荷電物質が水分量の増加につれて移動し易くなる
ことを示すものである.
なお,試料混合には純水を用い,粘土の量も一定であるため,供試体ごとに電荷を持つ物質の
総量は同じと考えてよく,この結果は固体粘土の物性としての電気伝導性を示していると考えて良い.
乾燥密度との関係について言えば,含水比の低いw15やw17の供試体では,乾燥密度が増加するに
ついれて抵抗は下がっている.一方,含水比が相対的に高いw20,w22,w25の場合,乾燥密度の増加
による抵抗の減少は小さい.このことは,電気伝導経路となる水分のネットワークが,
低含水比の供試体では圧縮によって連結性が増すが,高含水比のものでは,当初から
水分の連結性がよく,密度の増加がほとんど伝導率の増加に寄与しないことを示している.

%--------------------
\begin{figure}[h]
	\begin{center}
	\includegraphics[width=0.9\linewidth]{Figs/fig5_3.eps} 
	\end{center}
	\caption{
		等価回路に含まれる直列抵抗成分$R_0$.
	} 
	\label{fig:fig5_3}
\end{figure}
%--------------------
図\ref{fig:fig5_4}は,CPEと並列接続された抵抗$R_{ct}$の推定結果を示したものである.
$R_{ct}$は乾燥密度や含水比との明瞭な相関がみられない。
特徴的な点を強いてあげるとすれば,水分量の少ないw15供試体で他よりも大きな値を
取るという点がある.
しかしながら,値の大小はあるものの水分量にも密度に対してもはっきりとした傾向を
示さないということは,物質界面における電荷移動抵抗は,水分量や密度に影響を
受けにくいという解釈がより自然と考えられる.
\begin{figure}[h]
	\begin{center}
	\includegraphics[width=0.9\linewidth]{Figs/fig5_4.eps} 
	\end{center}
	\caption{
		等価回路に含まれる並列抵抗成分$R_{ct}$.
	} 
	\label{fig:fig5_4}
\end{figure}
一方,$R_{ct}$と対をなす量としてインピーダンスに入る容量成分$C$の推定結果は
図\ref{fig:fig5_5}のようになる.
こちらも,水分,密度との相関はあまり明瞭では無いが,
相対的に水分の多いw22とw25供試体では,乾燥密度に対して明らかに増加する傾向の
あることが分かる.
これは、容量成分の発現には水分で満たされた間隙のサイズが関係することを示唆している。
ただし,$C$は$p=1$の場合を除き、単一のコンデンサを表現するものでなく,
RC並列回路からのずれが大きいときには,$C$に電気化学的なプロセスを具体的に
対応させることは難しい.
%--------------------
\begin{figure}[h]
	\begin{center}
	\includegraphics[width=0.9\linewidth]{Figs/fig5_5.eps} 
	\end{center}
	\caption{
		等価回路に含まれるコンデンサの静電容量$C$.
	} 
	\label{fig:fig5_5}
\end{figure}
図\ref{fig:fig5_6}はCPEの指数$p$の推定結果を示したもので、
最も興味深い挙動が現れている.
まず、含水比との関係を見ると,水分の増加に対し$p$の応答は
単調でなく,w20で極大となる結果を示している.
含水比が同じ(すなわち同じ供試体)で指数$p$には変動が有り,
この点について乾燥密度の関係をみれば,w15,w17,w20の
低含水比側では乾燥密度の増加に応じて$p$も増加するが,
高含水比ではその反対の傾向となることが示されている.
この結果は,含水比20\%前後が,指数$p$を決める要因の
特別な状態が現れていることを意味する.
指数$p$は、理想的コンデンサーからのずれを意味し,
$p=1$に近いほど理想的なコンデンサーとみなしうる。
逆に,$p$が小さい場合は、単一のコンデンサーで挙動を説明できないことを
意味する。特に$p=0.5$の場合はワールブルグインピーダンスの場合に
相当し,荷電物質の拡散が微視的な電気二重層の形成を律速する。
つまり,$p$が大きいときには一定の外部電場に対して充電が速やかかつ
効率的に行われるが,$p$が小さい場合には,電荷の移動が
拡散に支配される結果,理想的なコンデンサーのようには充電
すなわち分極が怒らない.
分極を妨げる要素には、水分が少なく伝導経路が十分に確保されないことと、
間隙が複雑に屈曲して電場へ追従した電荷の移動が制限を受けることに
2つが考えられる。
w22やw25の指数が小さいことは、間隙の屈曲によるものと考えられる。
このように考えると、乾燥密度が大きくなっても、屈曲率が増すために、
分極率には寄与しないことも説明がつく。
一方、w15やw17では水分が少なくもともと水分ネットワークの繋がりが
良くないことが$p$が小さくなることが一つの理由と考えられる。
そのため、乾燥密度が増すと、ネットワークの連結性は向上し、
指数$p$は若干大きな値を持つようになる。
しかしながら、間隙の屈曲は次第に大きくなるため、指数$p$の増加は
頭打ちになり、水分が多い場合ほどの値にはならない。
これに対して、w20は適切な水分量で、無理なく水和粘土が
締め固められたために、元々間隙の屈曲が小さく圧縮されることに
よって荷電物質の移動する一定の直線距離を移動するための経路長が
より短くなることで、分極が起こりやすくなると考えられる。
これは巨視的には最適含水比に近い状態で締め固められたことによると言うこともできる。
%--------------------
\begin{figure}[h]
	\begin{center}
	\includegraphics[width=0.9\linewidth]{Figs/fig5_6.eps} 
	\end{center}
	\caption{
		等価回路に含まれるCPE素子の指数$p$.
	} 
	\label{fig:fig5_6}
\end{figure}
%--------------------
最後に,CPEの時定数$T_{CPE}$の推定結果を図\ref{fig:fig5_9}に示す。
この結果は、水分、密度と最も相関の高い挙動を示す。
乾燥密度、含水比のいずれに対しても$T_{CPE}$はほぼ単調に増加している。
すなわち、乾燥密度が同じであれば、水分が多い方が、
水分量が同じならば乾燥密度が高い方が$T_{CPE}$の値が大きくなっている。
$T_{CPE}$は一般化されたコンデンサーの容量を表す。
従って、水分と密度に応じて、コンデンサー成分が支配的になることを意味する。
また、微視的なコンデンサーの総量が、密度や水分といったマクロ量で
スケールし、微視的な構造に依らないことを示すという点で非常に興味深い結果と言える。
なお、$p$もCPEの性質を決めるパラメータだが、
$T_{CPE}$は誘電体の分極率にあたり、$p$は分極の緩和時間に関する
パラメータであるため、両者が水分量と密度に対して同じように変化する必然性は
無い。ここでの結果は,$T_{CPE}$は微視構造をあまり反映せず、
一方$p$は間隙ネットワークの屈曲率に影響されることを示唆するという
点でで微視的構造を反映するパラメータということができる。
\begin{figure}[h]
	\begin{center}
	\includegraphics[width=0.9\linewidth]{Figs/fig5_7.eps} 
	\end{center}
	\caption{
		等価回路に含まれるCPE素子の時定数$T_{CPE}$.
	} 
	\label{fig:fig5_7}
\end{figure}
