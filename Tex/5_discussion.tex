\section{等価回路の推定}
図\ref{fig:fig5_1}に,本研究で得られた典型的なナイキスト線図を示す。
黒の実線で示した計測結果には,低周波側に大きな容量性の半円(i)のおよそ半分が現れている.
高周波側にも容量性半円の一部と思われる箇所がわずかに見られ,図ではこれに青の破線で
想定される半円を描き込んである。実際、100KHZ以上の周波数範囲まで計測を行うと,
この部分にもはっきりとした半円が現れることは,別途実験を行い確認している.
ここでは,物質移動に関係する長い緩和時間を持つ現象に興味があるため,
低周波側の半円(i)について調べる.
円弧状の半円を示す等価回路の一つはRC並列回路である.
ただし,RC並列回路は完全な半円でナイキスト線図を描く.
一方,ここでの計測結果では、縦方向に扁平な半円となっているため,
RC並列回路でのフィッティグはできず,Cole-ColeプロットやCole-Davidsonプロット
の使用がより適切と考えられる。
いずれのプロットを選択するかにあたり,半円(i)の高周波側での挙動を見ると
図\ref{fig:fig5_1}に緑の実線で示した箇所で直線的に変化をしている箇所があり,
この部分の実数軸に対する傾き$\alpha$は,およそ60$\sim$70度になっている.
直線的なインピーダンススペクトルはCPE素子で表現することができるため,
このことはCPE素子を含む等価回路を想定すべきであることを示唆している.
Cole-ColeプロットはCPEと抵抗の並列接続で与えられることからこの条件を満足する.
一方,Cole-Davidsonプロットは,Cole-Coleプロットと類似したナイキスト線図を与えるが,
有限個のCPE素子では表現することができない.Cole-Davidonプロットは,半円の左側半分(高周波側)が
右側に比べてより歪んだ形を表すことが可能だが,Cole-Coleプロットとの差が明らかになるのは
高周波側での傾きが図\ref{fig:fig5_1}の$\alpha$よりもかなり小さくなってからである.
以上のことから,ここではCole-Coleプロットを用いて,実験で得られたインピーダンス
スペクトルのフィッティングを行う.
なお,ナイキスト線図上の直線的な変化はワールブルグインピーダンスにも現れる.
しかしながら,ワールブルグインピーダンスの直線部分の傾きは45$^\circ$のため,
ワールブルグインピーダンスだけでは,今回の実験結果を再現することはできない.
この点でも,Cole-Coleプロットの利用が適切であるということができる.

Cole-Coleプロットのインピーダンスは,第3章で述べたように式(\ref{eqn:CC})で与えられる.
これに,抵抗を直列に接続すれば,実数軸上の任意の位置になる半円状のスペクトルへの
フィッティングを行うことができる.従って,ここで用いるインピーダンスは
\begin{equation}
	Z=R_0 +\frac{R_{ct}}{1+\left( i\omega R_{ct}C\right)^p}
	\label{eqn:CC_R0}
\end{equation}
と表すことができる.
ここに,$R_0$は直列接続した抵抗成分の抵抗を,$R_{ct}$と$C$は
RC並列回路の抵抗とコンデンサ成分の静電容量をそれぞれ表す.
式(\ref{eq:CC_R0})は次のように書き直すことができる.
\begin{equation}
	Z=R_0 +\frac{R_{ct}}{1+\left( i\omega \right)^pT_{CPE}}
	\label{eqn:R_CPE}
\end{equation}
ただし,
\begin{equation}
	T_{CPE}=R_{ct}^{p-1}C^{p}
	\label{eqn:T_CPE}
\end{equation}
である.式(\ref{eqn:R_CPE})は図\ref{fig:fig5_2}の回路の合成インピーダンス
であることから,この図に示した抵抗aとCPE(b)-抵抗(c)の直-並列回路が
本研究で用いる等価回路となる.
なお,$T_{CPE}$はCPE素子の時定数で,その次元は
\begin{equation}
	[T_{CPE}] ={\rm [ F\cdot s^{p-1}]}
	\label{eqn:}
\end{equation}
で,指数$p$が1のとき,コンデンサーの静電容量と単位も含めて一致する.
%--------------------
\begin{figure}[h]
	\begin{center}
	\includegraphics[width=0.9\linewidth]{Figs/fig5_1.eps} 
	\end{center}
	\caption{
		計測で得られたインピーダンススペクトルの特徴.	
	} 
	\label{fig:fig5_1}
\end{figure}
%--------------------
\begin{figure}[h]
	\begin{center}
	\includegraphics[width=0.5\linewidth]{Figs/fig5_2.eps} 
	\end{center}
	\caption{
		インピーダンススペクトルのフィッティングに用いる等価回路.
		$R_0,R_{ct}$および$T_{CPE}$はそれぞれの回路素子の素子定数を表す.
	} 
	\label{fig:fig5_2}
\end{figure}
%--------------------
%--------------------
\section{回路定数の推定}
\section{結果の解釈}
%--------------------
\begin{figure}[h]
	\begin{center}
	\includegraphics[width=0.9\linewidth]{Figs/fig5_3.eps} 
	\end{center}
	\caption{
		等価回路に含まれる直列抵抗成分$R_0$.
	} 
	\label{fig:fig5_3}
\end{figure}
%--------------------
\begin{figure}[h]
	\begin{center}
	\includegraphics[width=0.9\linewidth]{Figs/fig5_4.eps} 
	\end{center}
	\caption{
		等価回路に含まれる並列抵抗成分$R_{ct}$.
	} 
	\label{fig:fig5_4}
\end{figure}
%--------------------
\begin{figure}[h]
	\begin{center}
	\includegraphics[width=0.9\linewidth]{Figs/fig5_5.eps} 
	\end{center}
	\caption{
		等価回路に含まれるコンデンサの静電容量$C$.
	} 
	\label{fig:fig5_5}
\end{figure}
%--------------------
\begin{figure}[h]
	\begin{center}
	\includegraphics[width=0.9\linewidth]{Figs/fig5_6.eps} 
	\end{center}
	\caption{
		等価回路に含まれるCPE素子の指数$p$.
	} 
	\label{fig:fig5_6}
\end{figure}
%--------------------
\begin{figure}[h]
	\begin{center}
	\includegraphics[width=0.9\linewidth]{Figs/fig5_7.eps} 
	\end{center}
	\caption{
		等価回路に含まれるCPE素子の時定数$T_{CPE}$.
	} 
	\label{fig:fig5_7}
\end{figure}
